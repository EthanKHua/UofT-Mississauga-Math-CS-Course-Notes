\documentclass{eh-homework}

\begin{document}
    \begin{question}{1}
        \begin{enumerate}[label=(\alph*)]
            \item Prove that if $n > k$ and $\gcd(n,k) = 1$, then $n \mid \binom{n}{k}$.
            
            Recall that the chairperson identity: for integers \(n > k\),
            \[
                k \binom{n}{k} = n \binom{n-1}{k-1}.
            \]
            By definition, \(k \mid n \binom{n-1}{k-1}\), but since \(n\) and \(k\) are coprime, it must be true that \(k \mid \binom{n-1}{k-1}\). Thus \(\frac{1}{k}\binom{n-1}{k-1} \in \mathbb{N}\) and
            \[
                \binom{n}{k} = n \cdot \frac{1}{k}\binom{n-1}{k-1},
            \]
            so \(n \mid \binom{n}{k}\).

            \item Then, show that $(a+b)^n \equiv a^n + b^n \pmod{n}$ when $n$ is prime. 
            
            Using the binomial theorem,
            \[
                (a+b)^n = \sum_{k=0}^{n} \binom{n}{k} a^k b^{n-k}
            \]
            But notice that for all \(k\) with \(0 < k < n\), we have that \(\mathrm{gcd} (n,k) = 1\), so by the previous part, \(n \mid \binom{n}{k}\) and
            \[
                \binom{n}{k}a^k b^{n-k} \equiv 0 \pmod n.
            \]
            Thus
            \[
                (a+b)^n = a^n + b^n + \sum_{k=1}^{n-1} \binom{n}{k}a^k b^{n-k} \equiv a^n + b^n \pmod n
            \]
            as needed.

            \item Find two examples (that have different $a, b, n$) that show that if $n$ is composite, then the statement in part (b) may or may not hold.
            
            For the first example, let \(n = 4, a = b = 1\). We have that
            \[
                (1 + 1)^4 = 16 \equiv 0 \pmod 4
            \]
            but
            \[
                1^4 + 1^4 \equiv 2 \pmod 4
            \]
            which shows that the statement does not hold.

            Next, let \(n = 6, a = 0, b = 2\). It is easy to see that
            \[
                (a+b)^n = 2^6 = b^n
            \]
            so the statement will hold, even though \(n\) is not prime.
        \end{enumerate}
    \end{question}
\end{document}