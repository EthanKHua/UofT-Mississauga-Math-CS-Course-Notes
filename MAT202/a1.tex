\documentclass{article}
\usepackage[margin=1.0in]{geometry}
\usepackage{amssymb,amsmath,amsthm,amsfonts}
\usepackage{enumitem}
\usepackage{xcolor}
\usepackage{mathtools}
\usepackage{systeme}

% My boxes
\usepackage[breakable]{tcolorbox}

% \RequirePackage{background}
% \backgroundsetup{
%     scale=1,
%     color=black,
%     opacity=1,
%     angle=0,
%     contents={
%         \includegraphics[width=\paperwidth,height=\paperheight]{\nightmodebackground}
%     }
% }

\definecolor{pastelblue}{RGB}{96, 145, 245}
\definecolor{pastelgreen}{RGB}{106, 235, 135}
\definecolor{darkgray}{RGB}{60, 60, 60}
\definecolor{lightgray}{RGB}{180, 180, 180}
\definecolor{offwhite}{RGB}{225, 225, 245}


\pagecolor{darkgray}
\color{offwhite}

\newcommand{\Z}{\mathbf{Z}}
\newcommand{\N}{\mathbf{N}}
\newcommand{\R}{\mathbf{R}}
\newcommand{\Q}{\mathbf{Q}}
\newcommand{\C}{\mathbf{C}}

\newcommand{\id}{\mathrm{id}}
\newcommand{\op}{\mathrm{op}}
\newcommand{\diam}{\mathrm{diam}}
\newcommand{\GL}{\mathrm{GL}}
\newcommand{\Tr}{\mathrm{Tr}}
\newcommand{\im}{\mathrm{im}}
\newcommand{\rank}{\mathrm{rank}}

\newcommand{\cl}[1]{\overline{#1}}

\swapnumbers % places numbers before thm names

\theoremstyle{plain} % The "plain" style italicizes all body text.
	\newtheorem{thm}{Theorem}
		\numberwithin{thm}{section} % Theorem numbers are determined by section.
	\newtheorem{lemma}[thm]{Lemma}
	\newtheorem{prop}[thm]{Proposition}
	\newtheorem{cor}[thm]{Corollary}

\theoremstyle{definition}
    \newtheorem{defn}[thm]{Definition}
	\newtheorem{example}[thm]{Example}
	\newtheorem{exercise}[thm]{Exercise} %Exercise

\begin{document}
    \newtcolorbox{question}[2][]{fonttitle=\large, fontupper=\large, fontlower=\large, title=Question {#2}., oversize, arc=3mm, outer arc=2mm, opacityback=0.9, coltitle=offwhite, colframe=pastelblue, colback=darkgray, colupper=lightgray, collower=lightgray, leftrule=1mm, rightrule=1mm, toprule=1.5mm, titlerule=1mm, bottomrule=1mm, valign=center, add to natural height=5mm, lower separated=false, before lower=\begin{proof}, after lower= \smallbreak \end{proof}, #1, breakable=true}

    \begin{question}{1}
        Count the number of five-card hands that can be formed from a standard deck such that:
        \begin{itemize}
            \item the hand contains more red cards than black cards, \textbf{and}
            \item there are no repeated ranks in the hand.
        \end{itemize}
        
        \textit{Solution.}

        We will handle the first condition by separating the problem into 3 cases: the hands with 3, 4, and 5 red cards respectively. We will also consider the second condition in every case.

        \smallbreak

        First, we pick our 5 distinct ranks, which there are \({13\choose 5}\) different ways to do so. After picking the 5 ranks, we want to assign each colored card a rank. For 3, 4, and 5 red cards, there are \({5 \choose 3},{5 \choose 4},{5 \choose 5}\) ways respectively to assign each red card to a different rank. For each rank, there are 2 ways to choose a red card and 2 ways to choose a black card. Since we are choosing a total of 5 cards, no matter if we are choosing 3, 4, or 5 red cards, the number of ways turns out to be \(2^5\). We can add all the cases together to get that the answer is
        \[
            {13 \choose 5} \cdot {5 \choose 3} \cdot 2^5 + {13 \choose 5} \cdot {5 \choose 4} \cdot 2^5 + {13 \choose 5} \cdot {5 \choose 5} \cdot 2^5
        \]
        \[
            = {13 \choose 5} \cdot 2^5 \cdot (10 + 5 + 1) = {13 \choose 5} \cdot 2^9
        \]
    \end{question}
    \newpage
    \begin{question}{2}
        In the expansion of $\left(x^3 + (2y - 3z)^9\right)^{202}$, determine the coefficients of the following terms:
        \begin{enumerate}[label=(\alph*)]
            \item $z^{1818}$ 
            \item $x^{303}y^{404}z^{505}$
        \end{enumerate}
        \textit{Solution.}
        
        We can solve this problem by applying the binomial theorem twice. We see that
        \[
            \left(x^3 + (2y - 3z)^9\right)^{202} = \sum_{k=0}^{202} {202\choose k}x^{3(202 - k)}(2y - 3z)^{9k}
        \]
        \[
            = \sum_{k=0}^{202} {202\choose k}x^{3(202 - k)}\sum_{i=0}^{9k} {9k \choose i}(2y)^{9k - i}(-3z)^{i}
        \]
        For part (a), we are interested in the term with \(k = 202,\ i = 9k = 1818\), as that is the only way we can get \(z^{1818}\). The coefficient for this is \({202 \choose 202}{1818 \choose 1818}(-3)^{1818} = 3^{1818}\).

        \medskip

        For part (b), to obtain the term with \(x^{303} y^{404} z^{505}\), we set \(k = 101\), \(i = 505\), and see that the coefficient is \({202\choose 101}{909\choose 505}2^{909 - 505}(-3)^{505}\).
    \end{question}
\end{document}