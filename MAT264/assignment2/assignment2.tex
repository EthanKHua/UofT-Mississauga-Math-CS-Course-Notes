\documentclass{eh-homework}

\begin{document}
\begin{question}{1}
    \textbf{(Based on 2.2, \#24)} Let \(g(x) = \left(\frac{1}{2}\right)^x + \left(\frac{1}{5}\right)^x - 10^{-5}\).
    
    \begin{enumerate}[label=\alph*.]
        \item Show that if \(g\) has a zero at \(p\), then the function \(f(x) = x + c g(x)\) has a fixed point at \(p\).
        
        Suppose that \(g\) has a zero at \(p\). Then \(g(p) = 0\). It follows immediately that \(f(p) = p + cg(p) = p\), so \(f\) has a fixed point at \(p\).
        
        \item Find a value of \(c\) for which fixed point iteration of \(f(x)\) will successfully converge for any starting value, \(p_0\), in the interval \([16, 17]\). (*Note: You don’t need to include the graphs.)
    
        To guarantee convergence, we will find \(c\) such that \(|f'(x)| < 1\) for all \(x \in [16,17]\). First, we rule out \(c=0\), as despite \(f(x) = x\) converging to a fixed point everywhere, it is unable to tell us about the roots of \(g\). Now, we compute that
        \[
            f'(x) = 1 + c\left( \left(\frac{1}{2}\right)^x \cdot \ln \left( \frac{1}{2} \right) + \left(\frac{1}{5}\right)^x \cdot \ln \left( \frac{1}{5} \right) \right) = 1 - c \left( 2^{-x} \cdot \ln 2 + 5^{-x} \cdot \ln 5 \right)
        \]
        We note that if \(c < 0\), then \(- c \left( 2^{-x} \cdot \ln 2 + 5^{-x} \cdot \ln 5 \right) > 0\), so \(f'(x) > 1\), which is not what we want. If \(c > 0\), \(f'\) is an increasing function. Since \(16 \leq x \leq 17\) we get that
        \[
            1 - c(2^{-16} \cdot \ln 2 + 5^{-16} \cdot \ln 5) \leq f'(x) \leq 1 - c(2^{-17} \cdot \ln 2 + 5^{-17} \cdot \ln 5)
        \]
        We solve for \(c\) in the following inequality:
        \begin{align*}
            1 - c(2^{-17} \cdot \ln 2 + 5^{-17} \cdot \ln 5) < 1 &\implies c(2^{-17} \cdot \ln 2 + 5^{-17} \cdot \ln 5) > 0 \\
            &\implies c > 0
        \end{align*}
        We also want the lower bound of \(f'(x)\) to be -1:
        \begin{align*}
            1 - c(2^{-16} \cdot \ln 2 + 5^{-16} \cdot \ln 5) > -1 &\implies c(2^{-16} \cdot \ln 2 + 5^{-16} \cdot \ln 5) < 2 \\
            &\implies c < \frac{2}{2^{-16} \cdot \ln 2 + 5^{-16} \cdot \ln 5}
        \end{align*}
        Thus any value of \(c\) between \(0\) and \(\dfrac{2}{2^{-16} \cdot \ln 2 + 5^{-16} \cdot \ln 5}\) will work, so we can just pick \(c = \dfrac{1}{2^{-16} \cdot \ln 2 + 5^{-16} \cdot \ln 5}\).

        \item Use the function from part (b) with the value of \(c\) you have determined to find a root of \(g(x)\) accurate to within \(10^{-4}\). State the value you used for \(p_0\) and show the last three iterations. How many iterations did it take?
        
        We will use fixed point iteration on \(f(x) = x + cg(x)\) with \(p_0 = 16.5\). Below is the Octave code, input, and output:
        \begin{lstlisting}
function [m] = fixedpoint(f,x,N,tol)
  for j=1:N
    m = f(x);
    disp(["Value at iteration number " num2str(j) ": " num2str(m)]);
    if abs(m - x) <= tol
    disp(["Fixed point within given tolerance found in " num2str(j) " iterations."])
    return;
    else
    x = m;
    end%if
  end%for
  disp("Method failed. Max iterations exceeded.")
end%function

>> [m] = fixedpoint(@(x) x + 1/(2^(-16)*log(2) + 5^(-16)*log(5))*(1/2^x + 1/5^x - 10^(-5)), 16.5, 1000, 10^(-4))
Value at iteration number 1: 16.5747
Value at iteration number 2: 16.5979
Value at iteration number 3: 16.6056
Value at iteration number 4: 16.6083
Value at iteration number 5: 16.6092
Value at iteration number 6: 16.6095
Value at iteration number 7: 16.6096
Value at iteration number 8: 16.6096
Fixed point within given tolerance found in 8 iterations.
m = 16.610
        \end{lstlisting}

        We found a fixed point for \(f\) around \(x = 16.610\), which implies that \(g\) has a root around that point as well.
    
        \item Now repeat part (c) and find a root of \(g\) accurate to within \(10^{-7}\), using potentially other values for \(c\) as necessary. Explain your process and how you picked an appropriate \(c\) and \(x_0\).
        
        We continue using fixed point iteration, keeping the value of \(c\) the same. We know that our fixed point is close to \(x = 16.610\), so that will be where we start the next fixed point iteration. Below is the Octave commands used and the output:
        \begin{lstlisting}
>> format long
>> [m] = fixedpoint(@(x) x + 1/(2^(-16)*log(2) + 5^(-16)*log(5))*(1/2^x + 1/5^x - 10^(-5)), 16.610, 1000, 10^(-7))
Value at iteration number 1: 16.6098
Value at iteration number 2: 16.6097
Value at iteration number 3: 16.6097
Value at iteration number 4: 16.6096
Value at iteration number 5: 16.6096
Value at iteration number 6: 16.6096
Value at iteration number 7: 16.6096
Value at iteration number 8: 16.6096
Value at iteration number 9: 16.6096
Fixed point within given tolerance found in 9 iterations.
m = 16.60964085351603
        \end{lstlisting}
    \end{enumerate}
    \end{question}
    \newpage
    \begin{question}{2}
    \textbf{(2.3, \#9)} The function \(g(x) = \sqrt[3]{5 - 3x}\) satisfies the hypotheses of Proposition 5 over the interval \([1, 1.3]\).
    
    Find a bound on the number of iterations required to find the fixed point to within \(10^{-5}\) accuracy starting with initial value \(x_0\) of your choice.

    Let's take the derivative of \(g\):
    \[
        g'(x) = \frac{-1}{(5-3x)^{\frac{2}{3}}}
    \]
    \end{question}
    
    \begin{question}{3}
    Consider the function \(g(x) = \ln(\sin x + 1.5)\).
    
    Find an initial value \(x_0\) (to four decimal places) so that Newton’s method fails at the second iteration. That is, Newton’s method finds \(x_1\) but cannot find \(x_2\).
    \end{question}
    
    \begin{question}{4}
    Let \(g(x) = \cos x - e^{-x/2} + 1.0005\), which has one negative root in \([-1, 0]\). Using \(x_0 = -1\) and \(x_1 = 0\), determine \(x_2\) and \(x_3\) when using:
    
    \begin{enumerate}[label=\alph*.]
        \item the bracketed Newton’s method, and
        \item the bracketed secant method.
    \end{enumerate}
    
    Show the results of your computation in a table and explain your steps.
    \end{question}
    
    \begin{question}{5}
    Let \(g(x) = \cos x - e^{-x/2} + 1.0005\).
    
    Using any of the root-finding methods discussed in Chapter 2, find all of its positive roots to within \(10^{-4}\). Explain how you know you’ve found all of them.
    \end{question}
    
    \begin{question}{6}
    \textbf{(3.2, \#11, 12)} A Lagrange interpolating polynomial is constructed for the function \(f(x) = (\sqrt{2})^x\) using \(x_0 = 0\), \(x_1 = 1\), \(x_2 = 2\), \(x_3 = 3\).
    
    \begin{enumerate}[label=\alph*.]
        \item If this polynomial is used to approximate \(f(1.5)\), find a bound on the error in this approximation.
    
        \item Find the Lagrange interpolating polynomial, and use it to approximate \(f(1.5)\). Then calculate the actual error in approximation.
    \end{enumerate}
    \end{question}
\end{document}