\documentclass{eh-homework}

\begin{document}
\begin{question}{1}
    \textbf{(Based on 2.2, \#24)} Let \(g(x) = \left(\frac{1}{2}\right)^x + \left(\frac{1}{5}\right)^x - 10^{-5}\).
    
    \begin{enumerate}[label=\alph*.]
        \item Show that if \(g\) has a zero at \(p\), then the function \(f(x) = x + c g(x)\) has a fixed point at \(p\).
        
        Suppose that \(g\) has a zero at \(p\). Then \(g(p) = 0\). It follows immediately that \(f(p) = p + cg(p) = p\), so \(f\) has a fixed point at \(p\).
        
        \item Find a value of \(c\) for which fixed point iteration of \(f(x)\) will successfully converge for any starting value, \(p_0\), in the interval \([16, 17]\). (*Note: You don’t need to include the graphs.)
    
        To guarantee convergence, we will find \(c\) such that \(|f'(x)| < 1\) for all \(x \in [16,17]\). First, we rule out \(c=0\), as despite \(f(x) = x\) converging to a fixed point everywhere, it is unable to tell us about the roots of \(g\). Now, we compute that
        \[
            f'(x) = 1 + c\left( \left(\frac{1}{2}\right)^x \cdot \ln \left( \frac{1}{2} \right) + \left(\frac{1}{5}\right)^x \cdot \ln \left( \frac{1}{5} \right) \right) = 1 - c \left( 2^{-x} \cdot \ln 2 + 5^{-x} \cdot \ln 5 \right)
        \]
        We note that if \(c < 0\), then \(- c \left( 2^{-x} \cdot \ln 2 + 5^{-x} \cdot \ln 5 \right) > 0\), so \(f'(x) > 1\), which is not what we want. If \(c > 0\), \(f'\) is an increasing function. Since \(16 \leq x \leq 17\) we get that
        \[
            1 - c(2^{-16} \cdot \ln 2 + 5^{-16} \cdot \ln 5) \leq f'(x) \leq 1 - c(2^{-17} \cdot \ln 2 + 5^{-17} \cdot \ln 5)
        \]
        We solve for \(c\) in the following inequality:
        \begin{align*}
            1 - c(2^{-17} \cdot \ln 2 + 5^{-17} \cdot \ln 5) < 1 &\implies c(2^{-17} \cdot \ln 2 + 5^{-17} \cdot \ln 5) > 0 \\
            &\implies c > 0
        \end{align*}
        We also want the lower bound of \(f'(x)\) to be -1:
        \begin{align*}
            1 - c(2^{-16} \cdot \ln 2 + 5^{-16} \cdot \ln 5) > -1 &\implies c(2^{-16} \cdot \ln 2 + 5^{-16} \cdot \ln 5) < 2 \\
            &\implies c < \frac{2}{2^{-16} \cdot \ln 2 + 5^{-16} \cdot \ln 5}
        \end{align*}
        Thus any value of \(c\) between \(0\) and \(\dfrac{2}{2^{-16} \cdot \ln 2 + 5^{-16} \cdot \ln 5}\) will work, so we can just pick \(c = \dfrac{1}{2^{-16} \cdot \ln 2 + 5^{-16} \cdot \ln 5}\).

        \item Use the function from part (b) with the value of \(c\) you have determined to find a root of \(g(x)\) accurate to within \(10^{-4}\). State the value you used for \(p_0\) and show the last three iterations. How many iterations did it take?
        
        We will use fixed point iteration on \(f(x) = x + cg(x)\) with \(p_0 = 16.5\). Below is the Octave code, input, and output:
        \begin{lstlisting}
function [m] = fixedpoint(f,x,N,tol)
  for j=1:N
    m = f(x);
    disp(["Value at iteration number " num2str(j) ": " num2str(m)]);
    if abs(m - x) <= tol
    disp(["Fixed point within given tolerance found in " num2str(j) " iterations."])
    return;
    else
    x = m;
    end%if
  end%for
  disp("Method failed. Max iterations exceeded.")
end%function

>> [m] = fixedpoint(@(x) x + 1/(2^(-16)*log(2) + 5^(-16)*log(5))*(1/2^x + 1/5^x - 10^(-5)), 16.5, 1000, 10^(-4))
Value at iteration number 1: 16.5747
Value at iteration number 2: 16.5979
Value at iteration number 3: 16.6056
Value at iteration number 4: 16.6083
Value at iteration number 5: 16.6092
Value at iteration number 6: 16.6095
Value at iteration number 7: 16.6096
Value at iteration number 8: 16.6096
Fixed point within given tolerance found in 8 iterations.
m = 16.610
        \end{lstlisting}

        We found a fixed point for \(f\) around \(x = 16.610\), which implies that \(g\) has a root around that point as well.
    
        \item Now repeat part (c) and find a root of \(g\) accurate to within \(10^{-7}\), using potentially other values for \(c\) as necessary. Explain your process and how you picked an appropriate \(c\) and \(x_0\).
        
        We continue using fixed point iteration, keeping the value of \(c\) the same. We know that our fixed point is close to \(x = 16.610\), so that will be where we start the next fixed point iteration. Below is the Octave commands used and the output:
        \begin{lstlisting}
>> format long
>> [m] = fixedpoint(@(x) x + 1/(2^(-16)*log(2) + 5^(-16)*log(5))*(1/2^x + 1/5^x - 10^(-5)), 16.610, 1000, 10^(-7))
Value at iteration number 1: 16.6098
Value at iteration number 2: 16.6097
Value at iteration number 3: 16.6097
Value at iteration number 4: 16.6096
Value at iteration number 5: 16.6096
Value at iteration number 6: 16.6096
Value at iteration number 7: 16.6096
Value at iteration number 8: 16.6096
Value at iteration number 9: 16.6096
Fixed point within given tolerance found in 9 iterations.
m = 16.60964085351603
        \end{lstlisting}
    \end{enumerate}
    \end{question}
    \newpage
    \begin{question}{2}
    \textbf{(2.3, \#9)} The function \(g(x) = \sqrt[3]{5 - 3x}\) satisfies the hypotheses of Proposition 5 over the interval \([1, 1.3]\).
    
    Find a bound on the number of iterations required to find the fixed point to within \(10^{-5}\) accuracy starting with initial value \(x_0\) of your choice.

    Let \(\hat{x}\) be the fixed point of \(g\) within the interval \([1, 1.3]\). To find the \(M\) value in proposition 5, we take the derivative of \(g\):
    \[
        g'(x) = \frac{-1}{(5-3x)^{\frac{2}{3}}}, \text{ where } x \in [1,1.3].
    \]
    Since \(1 \leq x \leq 1.3\), we have
    \[
        1.1 \leq 5 - 3x \leq 2 \implies \frac{1}{5-3x} \leq \frac{1}{1.1} < 1
    \]
    Since the function \(t^{\frac{2}{3}}\) is increasing for positive \(t\), we have that
    \[
        \frac{1}{(5-3x)^{\frac{2}{3}}} \leq \frac{1}{1.1^{\frac{2}{3}}} < 1
    \]
    Thus \(|g'(x)| \leq \dfrac{1}{1.1^{\frac{2}{3}}} < 1\), so \(M = \dfrac{1}{1.1^{\frac{2}{3}}} \approx 0.9384\) is our desired value.

    Let \(x_0 = 1\). Notice that \(1 \leq \hat{x} \leq 1.3\). By proposition 5, we have that
    \[
        |x_{165} - \hat{x}| \leq M^{165} |1 - \hat{x}| = M^{165} (\hat{x} - 1) \leq M^{165} (1.3 - 1) = 0.3 \cdot M^{165} \approx 8.393 \cdot 10^{-6} < 10^{-5}
    \]
    Thus the upper bound on the number of iterations is \(165\).

    \end{question}
    \newpage
    \begin{question}{3}
    Consider the function \(g(x) = \ln(\sin x + 1.5)\).
    
    Find an initial value \(x_0\) (to four decimal places) so that Newton’s method fails at the second iteration. That is, Newton’s method finds \(x_1\) but cannot find \(x_2\).

    \textit{Solution.}
    
    We know that Newton's method fails to find \(x_2\) if \(g'(x_1) = \dfrac{\cos x_1}{\sin x_1 + 1.5} = 0\). Let's pick \(x_1 = \dfrac{\pi}{2}\), which is a root of \(g'\). We want to try and obtain \(x_1\) using Newton's method, that is, we attempt to solve for \(x_0\) in the equation
    \[
        \frac{\pi}{2} = x_0 - \frac{g(x_0)}{g'(x_0)}.
    \]
    which is equivalent to trying to find a root of the function
    \[
        h(x) = x - \dfrac{g(x)}{g'(x)} - \dfrac{\pi}{2} = x - (\sin x + 1.5) \cdot \frac{\ln (\sin x + 1.5)}{\cos x} - \frac{\pi}{2}.
    \]
    Notice that \(h\) is defined for all values of \(x\) except for those of the form \(x = \frac{\pi}{2} + n \pi\), for \(n \in \mathbb{Z}\). Since \(-\frac{\pi}{2} < -1.5 < -1 < \frac{\pi}{2}\), we know that \(h\) is defined on the entire interval \([-1.3, -1]\). Moreover, \(h\) is continuous on this interval, and \(h(-1.5) = 1.81 > 0\) and \(h(-1) = -2.06 < 0\), so the conditions to use the bisection method are met. Below is the Octave code used to find the root:

    \begin{itemize}
        \item[]
        \begin{lstlisting}
function [m,M,i]=bisection(a,b,f,tol)
    N=ceil((log(b-a)-log(tol))/log(2));
    L=f(a);
    for i=1:N
        m=(a+b)/2;
        M=f(m);
        if (M==0)
        return;
        end%if
        if (L*M<0)
        b=m;
        else
        a=m;
        L=M;
        end%if
    end%for
    i=N;
end%function
>>f = @(x) x - (sin(x)+1.5)*log(sin(x)+1.5)/cos(x) - pi/2;
>> [m,M,i] = bisection(-1.5, -1, f, 0.00001)
m = -1.4567
M = 2.287e-05
i = 16
        \end{lstlisting}
    \end{itemize}
    Therefore, \(h\) has a root at around \(x = -1.4567\). We choose \(x_0\) to be this value, from which it follows that \(x_1 = \frac{\pi}{2}\), and thus since \(g'(x_1) = 0\), \(x_2\) cannot be found.
    \end{question}
    \newpage
    \begin{question}{4}
    Let \(g(x) = \cos x - e^{-x/2} + 1.0005\), which has one negative root in \([-1, 0]\). Using \(x_0 = -1\) and \(x_1 = 0\), determine \(x_2\) and \(x_3\) when using:
    
    \begin{enumerate}[label=\alph*.]
        \item the bracketed Newton’s method, and
        \item the bracketed secant method.
    \end{enumerate}
    
    Show the results of your computation in a table and explain your steps.

    \medskip

    (a):

    Let \(l\) denote the left endpoint of the current interval and \(r\) denote the right endpoint of the current interval. To begin, we set \(l = x_0\) and \(r = x_1\). For each row, we evaluate the candidate for the subsequent term in the sequence \(x_{k+1}\), which is \(c = x_{k-1} - \frac{g(x_k)}{g'(x_k)}\). If \(c \in [l,r]\), then we set \(x_{k+1} = c\). Otherwise, we take \(x_{k+1} = \frac{l+r}{2}\), the midpoint of the interval. For the next iteration, we choose the new left and right endpoints such that \(x_{k+1}\) is one of the new endpoints and we choose between the previous endpoints for our second endpoint such that the function evaluated at the chosen endpoint has opposite sign to \(f(x_{k+1})\).  Below is the table of values computed:

    \begin{center}
        \begin{tabular}{c|c|c|c|c|c|c|c|c}
            \(k\) & \(l\) & \(r\) & \(g(l)\) & \(g(r)\) & \(g'(x_{k})\) & Candidate & \(x_{k+1}\) & \(g(x_{k+1})\) \\
            \hline
            1 & -1 & 0 & -0.1079 & 1.0005 & 0.5 & -2.001 & -0.5 & 0.5941 \\
            2 & -1 & -0.5 & -0.1079 & 0.5941 & 1.121 & -1.030 & -0.75 & 0.2772
        \end{tabular}
    \end{center}

    \vspace{9pt}

    Thus, using the bracketed Newton's method, we have that \(x_2 = -0.5\) and \(x_3 = -0.75\).

    \medskip

    (b):

    The method for calculating the values is similar to the previous part, only that our candidate is now given by \(c_{k+1} = x_k - g(x_k) \dfrac{x_k - x_{k-1}}{g(x_k) - g(x_{k-1})}\). Below are the values calculated using the bracketed secant method:
    \begin{center}
        \begin{tabular}{c|c|c|c|c|c|c|c}
            \(k\) & \(l\) & \(r\) & \(g(l)\) & \(g(r)\) & Candidate & \(x_{k+1}\) & \(g(x_{k+1})\) \\
            \hline
            1 & -1 & 0 & -0.1079 & 1.0005 & -0.9026 & -0.9026 & 0.0497 \\
            2 & -1 & -0.9026 & -0.1079 & 0.0497 & -0.9333 & 0.9333 & 0.0010
        \end{tabular}
    \end{center}
    Thus \(x_2 = -0.9026\) and \(x_3 = 0.9333\).
    \end{question}
    
    \begin{question}{5}
    Let \(g(x) = \cos x - e^{-x/2} + 1.0005\).
    
    Using any of the root-finding methods discussed in Chapter 2, find all of its positive roots to within \(10^{-4}\). Explain how you know you’ve found all of them.

    \textit{Solution.}

    First, we notice that for \(x > \ln 1000\) 

    \end{question}
    
    \begin{question}{6}
    \textbf{(3.2, \#11, 12)} A Lagrange interpolating polynomial is constructed for the function \(f(x) = (\sqrt{2})^x\) using \(x_0 = 0\), \(x_1 = 1\), \(x_2 = 2\), \(x_3 = 3\).
    
    \begin{enumerate}[label=\alph*.]
        \item If this polynomial is used to approximate \(f(1.5)\), find a bound on the error in this approximation.
        
        Let \(P_3\) be the interpolating polynomial of least degree that passes through the points \((0, 1), (1, \sqrt{2}), (2,2)\) and \((3, 2\sqrt{2})\). We know that there exists \(\xi \in (0,3)\) such that the absolute error
        \[
            |f(1.5) - P(1.5)| = \left\vert \frac{f^{(4)}(\xi)}{4!} 1.5(1.5-1)(1.5-2)(1.5-3) \right\vert = \frac{5625}{24000}\left\vert \left(\frac{\ln 2}{2}\right)^4(\sqrt{2})^{\xi} \right\vert 
        \]
        \(f\) is increasing, so it follows that
        \[
            |f(1.5) - P(1.5)| \leq \frac{5625\ln 2}{24000 \cdot 2^4}(\sqrt{2})^3 = \frac{5625\ln 2}{24000 \cdot 2^3}\sqrt{2} \approx 0.2871
        \]
    
        \item Find the Lagrange interpolating polynomial, and use it to approximate \(f(1.5)\). Then calculate the actual error in approximation.
        
        First, we will find \(p_1, p_2, p_3\), and \(p_4\):
        \[
            p_1(x) = f(0)\cdot \frac{(x - 1)(x-2)(x-3)}{(0 - 1)(0 - 2)(0 - 3)} = -\frac{1}{6}(x-1)(x-2)(x-3)
        \]
        \[
            p_2(x) = f(1) \cdot \frac{x(x-2)(x-3)}{(1-0)(1-2)(1-3)} = \sqrt{2}x(x-2)(x-3)
        \]
        \[
            p_3(x) = f(2) \cdot \frac{x(x-1)(x-3)}{(2-0)(2-1)(2-3)} = -x(x-1)(x-3)
        \]
        \[
            p_4(x) = f(3) \cdot \frac{x(x-1)(x-2)}{(3-0)(3-1)(3-2)} = \frac{\sqrt{2}}{3}x(x-1)(x-2)
        \]
        Thus the Lagrange interpolating polynomial is
        \[
            P_3(x) = p_1(x) + p_2(x) + p_3(x) + p_4(x)
        \]
        \[
            = -\frac{1}{6}(x-1)(x-2)(x-3) + \sqrt{2}x(x-2)(x-3) - x(x-1)(x-3) + \frac{\sqrt{2}}{3}x(x-1)(x-2)
        \]
    \end{enumerate}
    \end{question}
\end{document}