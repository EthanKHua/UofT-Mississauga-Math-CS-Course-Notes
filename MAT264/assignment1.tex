\documentclass{article}
\usepackage[margin=1.0in]{geometry}
\usepackage{amssymb,amsmath,amsthm,amsfonts}
\usepackage{listings}
\usepackage{enumitem}
\usepackage{xcolor}
\usepackage{mathtools}

\lstset{
      language=Octave,
      basicstyle=\color{offwhite},
      breaklines=true,%
    morekeywords={matlab2tikz},
    keywordstyle=\color{pastelblue},%
    morekeywords=[2]{1}, keywordstyle=[2]{\color{black}},
    identifierstyle=\color{offwhite},%
    stringstyle=\color{mylilas},
    commentstyle=\color{pastelgreen},%
    showstringspaces=false,%without this there will be a symbol in the places where there is a space
    numbers=left,%
    numberstyle={\tiny \color{offwhite}},% size of the numbers
    numbersep=9pt, % this defines how far the numbers are from the text
   }

% My boxes
\usepackage[breakable]{tcolorbox}

% \RequirePackage{background}
% \backgroundsetup{
%     scale=1,
%     color=black,
%     opacity=1,
%     angle=0,
%     contents={
%         \includegraphics[width=\paperwidth,height=\paperheight]{\nightmodebackground}
%     }
% }

\definecolor{pastelblue}{RGB}{96, 145, 245}
\definecolor{pastelgreen}{RGB}{106, 235, 135}
\definecolor{darkgray}{RGB}{60, 60, 60}
\definecolor{lightgray}{RGB}{180, 180, 180}
\definecolor{offwhite}{RGB}{225, 225, 245}
\definecolor{mygreen}{RGB}{28,172,0} % color values Red, Green, Blue
\definecolor{mylilas}{RGB}{170,55,241}


\pagecolor{darkgray}
\color{offwhite}

\newcommand{\Z}{\mathbf{Z}}
\newcommand{\N}{\mathbf{N}}
\newcommand{\R}{\mathbf{R}}
\newcommand{\Q}{\mathbf{Q}}
\newcommand{\C}{\mathbf{C}}

\newcommand{\id}{\mathrm{id}}
\newcommand{\op}{\mathrm{op}}
\newcommand{\diam}{\mathrm{diam}}
\newcommand{\GL}{\mathrm{GL}}
\newcommand{\Tr}{\mathrm{Tr}}
\newcommand{\im}{\mathrm{im}}
\newcommand{\rank}{\mathrm{rank}}

\newcommand{\cl}[1]{\overline{#1}}

\swapnumbers % places numbers before thm names

\theoremstyle{plain} % The "plain" style italicizes all body text.
	\newtheorem{thm}{Theorem}
		\numberwithin{thm}{section} % Theorem numbers are determined by section.
	\newtheorem{lemma}[thm]{Lemma}
	\newtheorem{prop}[thm]{Proposition}
	\newtheorem{cor}[thm]{Corollary}

\theoremstyle{definition}
    \newtheorem{defn}[thm]{Definition}
	\newtheorem{example}[thm]{Example}
	\newtheorem{exercise}[thm]{Exercise} %Exercise

\begin{document}
    \newtcolorbox{question}[2][]{fonttitle=\large, fontupper=\large, fontlower=\large, title=Question {#2}., oversize, arc=3mm, outer arc=2mm, opacityback=0.9, coltitle=offwhite, colframe=pastelblue, colback=darkgray, colupper=lightgray, collower=lightgray, leftrule=1mm, rightrule=1mm, toprule=1.5mm, titlerule=1mm, bottomrule=1mm, valign=center, add to natural height=5mm, lower separated=false, before lower=\begin{proof}, after lower= \smallbreak \end{proof}, #1, breakable=true}
    \begin{question}{1}
        Suppose \(\widetilde{p}\) must approximate \(p\) with relative error at most \(10^{-3}\). Find the largest interval in which \(\widetilde{p}\) must lie if \(p = 900\).

        \tcblower

        Since we want the relative error to be at most \(10^{-3}\), we set
        \[
            \frac{|\widetilde{p} - p|}{|p|} \leq 10^{-3}
        \]
        Substitute \(p = 900\) to get
        \[
            \frac{|\widetilde{p} - 900|}{900} \leq 10^{-3} \implies |\widetilde{p} - 900| \leq \frac{9}{10} \implies 900-\frac{9}{10} \leq \widetilde{p} \leq 900 + \frac{9}{10}.
        \]
        Thus \(\widetilde{p}\) lies within the interval \(\left[ 900 - \dfrac{9}{10}, 900 + \dfrac{9}{10}\right]\).
    \end{question}
    \newpage
    \begin{question}{2}
        Compute the absolute error and relative error of the following approximation of \(e\):
        \[
            \sum_{n=0}^{5} \frac{1}{n!}
        \]
        Using Octave, we find that the absolute error is
        \[
            \left\vert e - \sum_{n=0}^{5} \frac{1}{n!} \right\vert = \left\vert e - \left(1 + 1 + \frac{1}{2} + \frac{1}{6} + \frac{1}{24} + \frac{1}{120}\right) \right\vert = \left\vert e - \frac{326}{120} \right\vert \approx \texttt{1.615161792378306e-03}
        \]
        and the relative error is
        \[
            \left\vert \frac{e - \sum_{n=0}^{5} \frac{1}{n!}}{e} \right\vert = \left\vert \frac{e - \frac{326}{120}}{e} \right\vert \approx \texttt{5.941848175815963e-04}
        \]
    \end{question}
    \newpage
    \begin{question}{3}
        Find the second Taylor polynomial, \(P_2(x)\), \(f(x) = e^x \cos (x)\) about \(x_0 = 0\).
        \begin{enumerate}[label=(\alph*)]
            \item Use \(P_2(0.5)\) to approximate \(f(0.5)\). Find an upper bound on the error \(|f(0.5) - P_2(0.5)|\) using the remainder term and compare it to the actual error.
            
            First, we find \(P_2(x)\). We calculate that
            \begin{align*}
                f(x_0) &= 1 \\
                f'(x_0) &= e^{x_0}(\cos (x_0) - \sin (x_0)) = 1 \\
                f''(x_0) &= -2e^{x_0}\sin (x_0) = 0 \\
                f^{(3)}(x) &= -2e^{x} (\sin (x) - \cos (x))
            \end{align*}
            Thus \(P_2(x) = 1 + x\), so \(f(0.5) \approx P_2(0.5) = 1.5\). The error term is
            \[
                |R_2(0.5)| = \left\vert \dfrac{f^{(3)}(\xi)}{3!}(0.5)^{3} \right\vert = \left\vert \dfrac{e^{\xi } (\sin (\xi ) - \cos (\xi ))}{24} \right\vert \text{, for } \xi \in (0, 0.5).
            \]
            Since \(e^{\xi} < e^{0.5}\), \(\sin (\xi), \cos (\xi) \leq 1\), we get that
            \[
                |R_2(0.5)| < \frac{e^{0.5}}{12} \approx 0.1374.
            \]
            and the actual absolute error is
            \[
                \left\vert e^{0.5} \cos (0.5) - 1.5 \right\vert \approx 0.053111
            \]

            \item Find a bound on the error \(|f(x) - P_2(x)|\) good on the interval \([0,1]\).
            
            Similar to the previous part, the error term is
            \[
                |R_2(x)| = \left\vert \frac{e^{\xi}(\sin (\xi ) - \cos (\xi ))}{3}x^3 \right\vert \text{, for } \xi  \in (0, x)
            \]
            We know that \(e^{\xi} < e^x\), so
            \[
                |R_2(x)| < \frac{2}{3} x^3e^x.
            \]

            \item Approximate \(\int _0^1 f(x)\ dx\) by calculating \(\int _0^1 P_2(x)\ dx\) instead.
            
            We have that
            \[
                \int _0^1 f(x)\ dx \approx \int _0^1 P_2(x)\ dx = \int _0^1 1 + x\ dx = x + \frac{x^2}{2} \Big|_0^1 = 1.5
            \]

            \item Find an upper bound for the error in (c) using \(\int _0^1 |R(x)|\ dx\) and compare the bound to the actual error.
            
            From part (b), the error term \(|R_2(x)\) is bounded above by \(\dfrac{2}{3}x^3 e^x\). Thus the error for the computation is
            \[
                \int _0^1 |R_2(x)|\ dx < \int _0^1 \dfrac{2}{3}x^3 e^x\ dx = \frac{2}{3}\left( x^3e^x - 3x^2e^x + 6xe^x - 6e^x \right) \Big|_0^1
            \]
            \[
                = \frac{2}{3}(-2e + 6) \approx 0.3756.
            \]
            and the actual error is
            \[
                \left\vert \int _0^1 f(x)\ dx - 1.5 \right\vert = \left\vert \int _0^1 e^x \cos (x)\ dx - 1.5 \right\vert = \left\vert \frac{1}{2}e^x(\cos (x) + \sin (x))\Big|_0^1 - 1.5 \right\vert
            \]
            \[
                = \left\vert \frac{e(\cos (1) + \sin (1))}{2} - \frac{1}{2} - 1.5 \right\vert \approx 0.1220.
            \]
        \end{enumerate}
    \end{question}
    \newpage
    \begin{question}{4}
        Find a theoretical upper bound, as a function of \(x\), for the absolute error in using \(T_4(x)\) to approximate \(f(x) = \frac{10}{x} + \sin(10x)\); \(x_0 = \pi\).

        We can find the series expansion for \(\sin (10x)\) quite easily by noticing that
        \[
            \sin (10x) = \sin (10x - 10\pi) = \sin (10(x - \pi)) = \sum_{i=0}^{\infty} 
        \]
    \end{question}
    \newpage
    \begin{question}{5}
        Let \((p_n) = \left\langle \frac{3n^5 - 5n}{1 - n^5} \right\rangle \to -3\) Find the (fastest) rate of convergence of the form \(\mathcal{O}\left(\frac{1}{n^p}\right)\) or \(\mathcal{O}\left(\frac{1}{a^n}\right)\) for each. If this is not possible, suggest a reasonable rate of convergence.

        We claim that \((p_n)\) converges to \(p = 3\) with rate of convergence \(\mathcal{O} \left(\frac{1}{n^4}\right)\). To prove this, first let \(n_0 = 1\) and \(\lambda = 7\). We see that for all \(n > n_0\),
        \[
            |p_n - p| = \left\vert \frac{3n^5 - 5n}{1 - n^5} + 3 \right\vert = \left\vert \frac{3 - 5n}{(1 - n)(1 + n + n^2 + n^3 + n^4)} \right\vert
        \]
        \[
            = \left\vert \frac{5 - 5n - 2}{(1 - n)(1 + n + n^2 + n^3 + n^4)} \right\vert = \left\vert \frac{5}{1 + n + n^2 + n^3 + n^4} + \frac{2}{(n - 1)(1 + n + n^2 + n^3 + n^4)} \right\vert
        \]
        We can remove the absolute values because the term inside is positive. Since \(n - 1 > 1\), we get
        \[
            \frac{5}{1 + n + n^2 + n^3 + n^4} + \frac{2}{(n - 1)(1 + n + n^2 + n^3 + n^4)} \leq \frac{5}{n^4} + \frac{2}{n^4} = \frac{7}{n^4}
        \]
        which is what we wanted.

        To show that this is indeed the fastest rate of convergence, we can show that the sequence \(\left( \frac{1}{n^4} \right)\) converges to 0 with rate of convergence \(\mathcal{O} \left( p_n + 3 \right)\). To do this, let \(n_0 = 1\), \(\lambda = 1\). Let \(n > n_0\). From the work we did in the previous part, we know that
        \[
            |p_n - p| = \frac{5}{1 + n + n^2 + n^3 + n^4} + \frac{2}{(n - 1)(1 + n + n^2 + n^3 + n^4)}
        \]
        Since \(\dfrac{2}{(n-1)(1 + n + n^2 + n^3 + n^4)} > 0\) and \(1 < n < n^2 < n^3 < n^4\), we get
        \[
            |p_n - p| > \frac{5}{1 + n + n^2 + n^3 + n^4} > \frac{5}{n^4 + n^4 + n^4 + n^4 + n^4} = \frac{1}{n^4}
        \]
        as needed.
    \end{question}
    \newpage
    \begin{question}{6}
        \begin{enumerate}[label=(\alph*)]
            \item Suppose you are trying to find the root of \(f(x) = x - e^{-x}\) using the bisection method. Find an integer \(a\) such that the interval \([a, a+2]\) is an appropriate one in which to start the search.
            
            Let \(a = 0\). We see that \(f(0) = -1 < 0\), \(f(2) = 2 - \frac{1}{e^2} > 0\), so the interval [0,2] satisfies the conditions to use the bisection method.

            \item Use the bisection method to find the root in your interval in (a), accurate to \(10^{-4}\). Provide the Octave code you used to produce your result.
            
            Using the method, the root was found at \(x \approx 0.5672\).

            Below is the Octave code used, along with the command used in command line:
            \begin{lstlisting}
function [m,M,i]=bisection(a,b,f,tol)
N=ceil((log(b-a)-log(tol))/log(2));
L=f(a);
for i=1:N
m=(a+b)/2;
M=f(m);
if (M==0)
return;
end%if
if (L*M<0)
b=m;
else
a=m;
L=M;
end%if
end%for
i=N;
end%function
            \end{lstlisting}
            \begin{lstlisting}[numbers=none]
>> [m,M,i] = bisection(0, 2, @(x) x-e^(-x), 0.0001)
            \end{lstlisting}
        \end{enumerate}
    \end{question}
\end{document}