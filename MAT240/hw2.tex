\documentclass{article}
\usepackage[margin=1.0in]{geometry}
\usepackage{amssymb,amsmath,amsthm,amsfonts}
\usepackage{enumitem}
\usepackage{xcolor}
\usepackage{mathtools}
\usepackage{systeme}

% My boxes
\usepackage[breakable]{tcolorbox}

% \RequirePackage{background}
% \backgroundsetup{
%     scale=1,
%     color=black,
%     opacity=1,
%     angle=0,
%     contents={
%         \includegraphics[width=\paperwidth,height=\paperheight]{\nightmodebackground}
%     }
% }

\definecolor{pastelblue}{RGB}{96, 145, 245}
\definecolor{pastelgreen}{RGB}{106, 235, 135}
\definecolor{darkgray}{RGB}{60, 60, 60}
\definecolor{lightgray}{RGB}{180, 180, 180}
\definecolor{offwhite}{RGB}{225, 225, 245}


\pagecolor{darkgray}
\color{offwhite}

\newcommand{\Z}{\mathbf{Z}}
\newcommand{\N}{\mathbf{N}}
\newcommand{\R}{\mathbf{R}}
\newcommand{\Q}{\mathbf{Q}}
\newcommand{\C}{\mathbf{C}}

\newcommand{\id}{\mathrm{id}}
\newcommand{\op}{\mathrm{op}}
\newcommand{\diam}{\mathrm{diam}}
\newcommand{\GL}{\mathrm{GL}}
\newcommand{\Tr}{\mathrm{Tr}}
\newcommand{\im}{\mathrm{im}}
\newcommand{\rank}{\mathrm{rank}}

\newcommand{\cl}[1]{\overline{#1}}

\newenvironment{amatrix}[1]{%
  \left(\begin{array}{@{}*{#1}{c}|c@{}}
}{%
  \end{array}\right)
}

\swapnumbers % places numbers before thm names

\theoremstyle{plain} % The "plain" style italicizes all body text.
	\newtheorem{thm}{Theorem}
		\numberwithin{thm}{section} % Theorem numbers are determined by section.
	\newtheorem{lemma}[thm]{Lemma}
	\newtheorem{prop}[thm]{Proposition}
	\newtheorem{cor}[thm]{Corollary}

\theoremstyle{definition}
    \newtheorem{defn}[thm]{Definition}
	\newtheorem{example}[thm]{Example}
	\newtheorem{exercise}[thm]{Exercise} %Exercise

\begin{document}
    \newtcolorbox{question}[2][]{fonttitle=\large, fontupper=\large, fontlower=\large, title=Question {#2}., oversize, arc=3mm, outer arc=2mm, opacityback=0.9, coltitle=offwhite, colframe=pastelblue, colback=darkgray, colupper=lightgray, collower=lightgray, leftrule=1mm, rightrule=1mm, toprule=1.5mm, titlerule=1mm, bottomrule=1mm, valign=center, add to natural height=5mm, lower separated=false, before lower=\begin{proof}, after lower= \smallbreak \end{proof}, #1, breakable=true}

    \begin{question}{1}
        Let \( V \) be a vector space over the field \( \mathbb{F} \), and \( S \) a (non-empty) set. Let \( \mathcal{F}(S,V) = \{ f : S \to V \} \) be the set of \( V \)-valued functions.
        
        We define addition and scaling on \( \mathcal{F}(S,V) \) pointwise:
        \[
        (f + g)(s) = f(s) + g(s)
        \]
        \[
        (cf)(s) = cf(s)
        \]
        
        We will verify some of the vector space axioms required to prove that \( \mathcal{F}(S,V) \) is a vector space over \( \mathbb{F} \).
        
        \begin{enumerate}[label=(\alph*)]
            \item Why do these operations make sense?
            \item Prove (using only the definitions above, and the fact that \( V \) is a vector space) that \( c(f + g) = cf + cg \) for all \( f,g \in \mathcal{F}(S,V) \) and \( c \in \mathbb{F} \).
            \item Prove that for all \( f \in \mathcal{F}(S,V) \) there exists \( g \in \mathcal{F}(S,V) \), so that \( f + g = 0 \). (Here \( 0 : S \to V \) is the constant function defined by \( 0(s) = 0_V \) for \( s \in S \).)
        \end{enumerate}

        \tcblower
        \ 

        (a):

        These operations make sense because they allow us to use the properties of the sets underlying \(\mathcal{F} (S,V)\).

        \medskip

        (b):

        We will do this by showing that for all \(s \in S\), we have \(c(f(s) + g(s)) = cf(s) + cg(s)\).

        Fix \(s \in S\). It follows that \(f(s), g(s) \in V\), so by the axiom of distributivity in \(V\), we have that
        \[
            c(f(s) + g(s)) = cf(s) + cg(s)
        \]

        \medskip

        (c):

        Let \(f \in \mathcal{F} (S,V)\). Choose \(g = (-1 \cdot f)\). Then for all \(s \in S\),
        \[
            f(s) + g(s) = f(s) + (-f(s)) = 0
        \]
        as needed.
    \end{question}
    \newpage
    \begin{question}{2}
        Let \( W = \left\{ (x,y,z,w) \in \mathbb{Q}^4 \middle| \begin{array}{l}
        x + 5w = y + 5z \\
        y = 4w - 3z \\
        x + y + z = 3w
        \end{array} \right\} \).
        
        Do not use Q3 to solve this problem. This problem is a “warm up” for Q3.
        
        \begin{enumerate}[label=(\alph*)]
            \item Rearrange the equations defining \( W \) to show that \( W \) is the set of solutions to a homogeneous system of equations.
            \item Solve the system using row-reduction and express the general solution as a linear combination of the “basic solutions”.
            \item Show that \( W = \text{span} \, S \), for some set \( S \subseteq \mathbb{Q}^4 \).
            \item Deduce that \( W \) is a subspace of \( \mathbb{Q}^4 \).
        \end{enumerate}

        \tcblower
        \ 

        (a):

        Rearranging, the equations become
        \begin{equation*}
            \systeme[xyzw]{
                x - y - 5z + 5w = 0,
                y + 3z -4w = 0,
                x + y + z -3w = 0
            }
        \end{equation*}

        \medskip

        (b):

        The augmented matrix associated with this system of equations is
        \[
            \begin{amatrix}{4}
                1 & -1 & -5 & 5 & 0 \\
                0 & 1 & 3 & -4 & 0 \\
                1 & 1 & 1 & -3 & 0
            \end{amatrix}
        \]
        Row reducing this, we get
        \begin{align*}
            \begin{amatrix}{4}
                1 & -1 & -5 & 5 & 0 \\
                0 & 1 & 3 & -4 & 0 \\
                1 & 1 & 1 & -3 & 0
            \end{amatrix}
            \xrightarrow{r_3\,\to\,r_3 - r_1}
            \begin{amatrix}{4}
                1 & -1 & -5 & 5 & 0 \\
                0 & 1 & 3 & -4 & 0 \\
                0 & 2 & 6 & -8 & 0
            \end{amatrix}
            \xrightarrow[r_3\,\to\,r_3 - 2r_2]{r_1\,\to\,r_1 + r_2}
            \begin{amatrix}{4}
                1 & 0 & -2 & 1 & 0 \\
                0 & 1 & 3 & -4 & 0 \\
                0 & 0 & 0 & 0 & 0
            \end{amatrix}
        \end{align*}
        We parameterize \(z\) and \(w\) to obtain that
        \begin{align*}
            x &= 2s - t \\
            y &= -3s + 4t \\
            z &= s \\
            w &= t
        \end{align*}
        so the general solution of this system of equations is given by
        \[
            \begin{pmatrix*}
                 x \\
                 y \\
                 z \\
                 w \\
            \end{pmatrix*}
            =
            s\begin{pmatrix*}[r]
                 2 \\
                 -3 \\
                 1 \\
                 0 \\
            \end{pmatrix*}
            +t
            \begin{pmatrix*}[r]
                 -1 \\
                 4 \\
                 0 \\
                 1 \\
            \end{pmatrix*}
        \]

        \medskip

        (c):

        Let
        \(S = \left\{ \begin{pmatrix*}[r]
                        2 \\
                        -3 \\
                        1 \\
                        0 \\
                      \end{pmatrix*},
                      \begin{pmatrix*}[r]
                        -1 \\
                        4 \\
                        0 \\
                        1 \\
                      \end{pmatrix*}\right\} \).
        Let \(\vec{v} \in \mathrm{span} S\). It follows that
        \[
            \vec{v} = s\begin{pmatrix*}[r]
                2 \\
                -3 \\
                1 \\
                0 \\
           \end{pmatrix*}
           +t
           \begin{pmatrix*}[r]
                -1 \\
                4 \\
                0 \\
                1 \\
           \end{pmatrix*}
        \]
        for some \(s,t \in \mathbb{Q}\). But notice that this is actually a solution to the system in \(W\), so \(\vec{v} \in W\).

        Now let \(\vec{w} \in W\), so \(\vec{w}\) solves the system of equations in \(W\), but this means that we can write \(\vec{w}\) as a linear combination of the vectors in \(S\), so \(\vec{w} \in \mathrm{span} S\), so \(W = \mathrm{span} S\).

        \medskip

        (d):

        By the previous part, \(W\) is actually a spanning set, and we know that all spanning sets are subspaces, so we conclude that \(W\) is a subspace of \(\mathbb{Q}^4\).
    \end{question}
    \newpage
    \begin{question}{3}
        We now generalize Q2. Consider a linear system with \( m \) equations and \( n \) unknowns:
        
        \[
        \begin{aligned}
        a_{11}x_1 + a_{12}x_2 + \cdots + a_{1n}x_n &= 0 \\
        a_{21}x_1 + a_{22}x_2 + \cdots + a_{2n}x_n &= 0 \\
        &\vdots \\
        a_{m1}x_1 + a_{m2}x_2 + \cdots + a_{mn}x_n &= 0.
        \end{aligned}
        \]
        
        We saw in Week 3 that any solution \( x = (x_1, x_2, \ldots, x_n) \in \mathbb{F}^n \) can be expressed as \( x = \sum_{i=1}^{k} t_i x_i \), where \( t_i \in \mathbb{F} \) are the parameters, and \( x_i \in \mathbb{F}^n \) are the “basic solutions”.
        
        Let \( W \) be the set of solutions to this system.
        
        \begin{enumerate}[label=(\alph*)]
            \item Prove that \( W = \text{span} \, S \) for some set \( S \), and hence that \( W \) is a subspace of \( \mathbb{F}^n \).
            \item Prove that the set \(\{x_1, x_2, \ldots, x_k\}\) is linearly independent.
            
            (Hint: Think about the variables which correspond to the choice of parameters. There is exactly one vector for each such parameter. Use the corresponding entry to show that if \( t_1 x_1 + t_2 x_2 + \cdots + t_k x_k = 0 \) then \( t_i = 0 \) for each \( i \).)
            
            \item Find a basis for \( W \).
        \end{enumerate}

        \tcblower
        \ 

        (a):

        Let \(S = \{ x_1, ..., x_k \}\). We note that any linear conbination of the vectors in \(S\) form a solution to the system, but not only that, any solution to the system can be written as a linear combination of these vectors, so \(\mathrm{span} S = W\).

        \medskip

        (b):

        Suppose that \(\sum_{i=1}^{k} t_i x_i = 0\).

        \medskip

        (c):

        From part (a) and part (b), the set \(S = \{ x_1, ..., x_k \}\) spans \(W\) and is linearly independent, which by definition forms a basis of \(W\).
    \end{question}
    \newpage
    \begin{question}{4}
        Is the set \( S = \{e_1 + 2 e_2 - 3 e_3, e_1 + e_2 - e_3, e_2 - e_3\} \subseteq \mathbb{Q}^3 \) a basis for \( \mathbb{Q}^3 \)? Justify your answer.

        \tcblower

        We claim that \(S\) is indeed a basis for \(\mathbb{Q}^3\). For convenience, denote the vectors in \(S\) by \(v_1, v_2, v_3\) respectively. Notice that
        \[
            e_1 = v_2 - v_3,\ e_2 = -v_1 + v_2 + 2v_3,\ e_3 = -v_1 + v_2 + v_3.
        \]
        Thus, for any \(x \in \mathbb{Q}^3\), since \(\{ e_1, e_2, e_3 \}\) is a basis for \(\mathbb{Q}^3\), for some \(a,b,c \in \mathbb{Q}\), we have that
        \[
            x = ae_1 + be_2 + ce_3 = a(v_2 - v_3) + b(-v_1 + v_2 + 2v_3) + c(-v_1 + v_2 + v_3)
        \]
        \[
            \implies x = (-b - c)v_1 + (a + b + c)v_2 + (-a + c)v_3
        \]
        which shows that \(S\) spans \(\mathbb{Q}^3\).

        Now, for constants \(p,q,r \in \mathbb{Q}\), suppose that
        \[
            0 = pv_1 + qv_2 + rv_3
        \]
        SUbstituting back our values, we get that
        \begin{align*}
            0 &= p(e_1 + 2e_2 - 3e_3) + q(e_1 + e_2 - e_3) + r(e_2 - e_3) \\
            &= (p + q)e_1 + (2p + q + r)e_2 + (-3p -q -r)e_3
        \end{align*}
        By the linear independence of the standard vectors, we have that
        \begin{equation*}
            \sysdelim..
            \systeme{
                p+q = 0,
                2p + q + r = 0,
                -3p -q - r = 0
            }
        \end{equation*}
        We can solve for \(p,q,r\) to get that \(p = q = r = 0\).

        Thus we can conclude that \(S\) is a basis for \(\mathbb{Q}^3\).
    \end{question}
    \newpage
    \begin{question}{5}
        Let \( V \) be a finite dimensional vector space over a field \( \mathbb{F} \).
        
        \begin{enumerate}
            \item Prove that if \( W \subseteq V \) is a subspace with basis \( \beta_W \), then there exists a linearly independent set \( \alpha \) so that \( \beta = \beta_W \cup \alpha \) is a basis for \( V \). (We say that \( \beta \) “extends” \( \beta_W \). So you are proving that “every basis of a subspace \( W \) can be extended to a basis of \( V \)”.)
            \item Prove that for any linearly independent set \( I \) and spanning set \( S \), we have \( |I| \leq \dim V \leq |S| \).
        \end{enumerate}
    \end{question}
    \newpage
    \begin{question}{6}
        Consider a matrix \( M \in \mathcal{M}_{n \times n}(\mathbb{F}) \). Given \( p \in \{1, \ldots, n\} \) we can split \( M \) into “blocks”: 
        
        \[
        M = \left( \begin{array}{c|c}
            A & B \\
            \hline
            C & D
        \end{array} \right) 
        \]
        
        where \( A \) is \( k \times k \), \( B \) is \( k \times (n-k) \), \( C \) is \( (n-k) \times k \) and \( D \) is \( (n-k) \times (n-k) \).
        
        For example, if \( n = 5 \) and \( k = 2 \), then such a block matrix would be of the form
        
        \[
        M = \left(\begin{array}{cc|ccc} 
        1 & 2 & 3 & 2 & 3 \\ 
        -5 & 3 & 3 & 1 & 1 \\
        \hline 
        1 & 2 & 0 & -1 & 1 \\ 
        3 & 1 & 3 & -1 & 7 \\ 
        1 & 0 & -1 & 3 & 5 
        \end{array}\right)
        \]
        
        where \( A = \begin{pmatrix} 
        1 & 2 \\
        -5 & 3 
        \end{pmatrix}, B = \begin{pmatrix} 
        3 & 2 & 3 \\ 
        3 & 1 & 1 
        \end{pmatrix}, C = \begin{pmatrix} 
        1 & 2 \\ 
        3 & 1 \\ 
        1 & 0 
        \end{pmatrix}, D = \begin{pmatrix} 
        0 & -1 & 1 \\ 
        3 & -1 & 7 \\ 
        -1 & 3 & 5 
        \end{pmatrix} \).
        
        Prove that if \( M = \left(\begin{array}{c|c} 
        A & B \\ 
        \hline
        C & D 
        \end{array}\right) \) and \( N = \left(\begin{array}{c|c} 
        A' & B' \\
        \hline
        C' & D' 
        \end{array}\right) \), then
        \[ \alpha M + N = \left(\begin{array}{c|c}
        \alpha A + A' & \alpha B + B' \\
        \hline
        \alpha C + C' & \alpha D + D' 
        \end{array}\right) \]
    \end{question}
    \newpage
    \begin{question}{7}
        Let \( W = \left\{ A \in \mathcal{M}_{2n \times 2n}(\mathbb{F}) \mid A = \left( \frac{X - X^t}{O_n} \middle| \frac{O_n}{X + X^t} \right) \text{ with } X \in \mathcal{M}_{n \times n}(\mathbb{F}) \right\} \).
        
        (Assume char\((\mathbb{F})\neq 2\).)
        
        \begin{enumerate}
            \item Let \( n = 2 \). Find a basis for \( W \).
            \item Now generalize to arbitrary \( n \). Find a basis for \( W \), and use it to compute dim \( W \).
        \end{enumerate}
    \end{question}
    \newpage
    \begin{question}{8}
        \begin{enumerate}[label=(\alph*)]
            \item Prove that if \( W_1, W_2 \subseteq V \) are subspaces, then \( W_1 + W_2 \) is a subspace.
            \item Let \( W_1 = \{(x, y, x + y) \in \mathbb{F}^3 \mid x, y \in \mathbb{F} \} \). Find two subspaces \( W_2, W_3 \) so that:
            
            \begin{itemize}
                \item \( W_1 + W_2 = \mathbb{F}^3 \) but \( \mathbb{F}^3 \neq W_1 \oplus W_2 \).
                \item \( W_1 \oplus W_3 = \mathbb{F}^3 \).
            \end{itemize}
            
            \item Find another subspace \( U \subseteq \mathbb{F}^3 \) so that \( W_1 \oplus U = \mathbb{F}^3 \).
        \end{enumerate}

        \tcblower
        \ 

        (a):

        Let \(W_1, W_2\) be subspaces of \(V\). We verify that \(W_1 + W_2\) is also a subspace of \(V\).

        First, note that \(0 \in W_1, W_2\), so \(0 + 0 = 0 \in W_1 + W_2\). Next, let \(c \in \mathbb{F}\), \(\vec{v} , \vec{w} \in W_1 + W_2\). Then \(\vec{v} = \vec{v}_1 + \vec{v}_2\) and \(\vec{w} = \vec{w}_1 + \vec{w}_2\), for some \(\vec{v}_1, \vec{w}_1 \in W_1\) and \(\vec{v}_2, \vec{w}_2 \in W_2\). Since \(W_1, W_2\) are subspaces, it is true that
        \[
            c\vec{v}_1 + \vec{w}_1 \in W_1 \text{ and } c\vec{v}_2 + \vec{w}_2 \in W_2
        \]
        which implies that
        \[
            \vec{v} + \vec{w} = (c\vec{v}_1 + \vec{w}_1) + (c\vec{v}_2 + \vec{w}_2) \in W_1 + W_2,
        \]
        verifying that \(W_1 + W_2\) is indeed a subspace.

        \medskip

        (b):

        Let \(W_2 = \mathbb{F}^3\), \(W_3 = \mathrm{span} \{e_3\}\).
    \end{question}
    \newpage
    \begin{question}{9}
        Let \( V \) be a finite dimensional vector space over \( \mathbb{F} \), and \( W_1, W_2 \subseteq V \) subspaces with bases \( \beta_1, \beta_2 \) respectively. Prove that \( V = W_1 \oplus W_2 \) if and only if \( \beta = \beta_1 \cup \beta_2 \) is a basis for \( V \).

        \tcblower

        Let \(m = |\beta_1|\), \(k = |\beta _2|\).

        Suppose that \(V = W_1 \oplus W_2\). We will show that \(\beta = \beta _1 \cup \beta _2\) is a basis for \(V\).

        Let \(x \in V\). By our assumption, \(x = w_1 + w_2\), for some \(w_1 \in W_1\) and \(w_2 \in W_2\). These vectors can in turn be written as
        \[
            w_1 = \sum_{i=1}^m a_i v_i \text{ and } w_2 = \sum_{i=1} ^k b_i w_i
        \]
        where \(v_i \in \beta _1\) and \(w_i \in \beta _2\). Thus \(x\) can be written as a linear combination of vectors in \(\beta\):
        \[
            x = \sum_{i=1}^m a_i v_i + \sum_{i=1} ^k b_i w_i
        \]
        so \(\beta\) spans \(V\).

        To show that \(\beta\) is linearly independent, suppose that
        \[
            \sum_{i=1}^{m} a_i v_i + \sum_{i=1}^{k} b_i w_i = 0
        \]
        We put the vectors of each subspace on each side to get
        \[
            \sum_{i=1}^{m} a_i v_i = - \sum_{i=1}^{k} b_i w_i
        \]
        By the closure property of subspaces, \(\sum_{i=1}^{m} a_i v_i \in W_1\) and \(\sum_{i=1}^{k} b_i w_i \in W_2\), but since they are equal, it must be true that \(\sum_{i=1}^{m} a_i v_i = \sum_{i=1}^{k} b_i w_i \in W_1 \cap W_2 = \{ 0 \}\), so \(\sum_{i=1}^{m} a_i v_i = \sum_{i=1}^{k} b_i w_i = 0\). Since \(\beta _1, \beta _2\) are linearly independent, it must be true that \(a_i = 0\) and \(b_i = 0\), which was what we wanted to show. Therefore \(\beta\) is indeed a basis for \(V\).

        Conversely, suppose that \(\beta\) is a basis for \(V\). We want to show that \(V = W_1 \oplus W_2\). It is obvious that \(W_1 + W_2 \subseteq V\), so it suffices to prove that \(V \subseteq W_1 \oplus W_2\) and \(W_1 \cap W_2 = \{ 0 \}\).

        Let \(x \in V\). Then since \(\beta\) is a basis, we have that
        \[
            x = \sum_{j=1}^{m} a_i v_i + \sum_{j=1}^{k} b_i w_i, \text{ for } a_i, b_i \in \mathbb{F}, v_i \in \beta _1, \text{ and } w_i \in \beta _2.
        \]
        By closure, we have that \(\sum_{j=1}^{m} a_i v_i \in W_1\) and \(\sum_{j=1}^{k} b_i w_i \in W_2\), so we see that \(x \in W_1 + W_2\). Thus \(V = W_1 + W_2\).

        To show that \(W_1 \cap W_2 = \{ 0 \}\), it suffices to show that if \(x \in W_1 \cap W_2\), then it must be true that \(x = 0\). Indeed, if \(x \in W_1 \cap W_2\), we can write it as a two linear combinations of vectors in either \(\beta _1\) or \(\beta _2\):
        \[
            x = \sum_{i=1}^{m} a_i v_i = \sum_{i=1}^{k} b_i w_i
        \]
        \[
            \implies \sum_{i=1}^{m} a_i v_i - \sum_{i=1}^{k} b_i w_i = 0
        \]
        By the linear independence of \(\beta\), we have that \(a_i = 0\), for all \(i\), which means that \(x = 0\), as desired, and the proof is complete.
    \end{question}
    \newpage
    \begin{question}{10}
        Let \( J = \left( \begin{array}{c|c} O & -I_2 \\ \hline I_2 & O \end{array} \right) \) and \( \mathbb{F} = \mathbb{C} \).
        
        \begin{enumerate}
            \item Verify that \( J^2 = -I_4 \).
            \item Find all \( X \in \mathcal{M}_{4\times 4}(\mathbb{F}) \) so that \( XJ = JX \).
            \item Show that \( sp_4 = \{ X \in \mathcal{M}_{4\times 4}(\mathbb{F}) | XJ = JX \} \) is a subspace of \( \mathcal{M}_{4\times 4}(\mathbb{F}) \).
            \item Find dim \( sp_4 \) by finding a basis for \( sp_4 \).
        \end{enumerate}
    \end{question}
    \newpage
    \begin{question}{11}
        Determine if the statements below are true or false. If true, give a proof. If false, explain why, and/or provide a counterexample.
        
        \begin{enumerate}[label=(\alph*)]
            \item Let \( V \) be a finite dimensional vector space over \( \mathbb{F} \). If \( I \subseteq V \) is a linearly independent set so that for any \( x \in V \setminus I \), the set \( I \cup \{ x \} \) is linearly dependent, then \( I \) is a basis for \( V \).
            \item Let \( V \) be a finite dimensional vector space over \( \mathbb{F} \). If \( S \subseteq V \) is a spanning set so that \( |S| = \dim V \), then \( S \) is a basis for \( V \).
            \item Let \( V \) be a finite dimensional vector space over \( \mathbb{F} \). If \( W \subseteq V \) a subspace, then there exists a unique subspace \( U \subseteq V \) so that \( V = W \oplus V \).
        \end{enumerate}
    \end{question}
\end{document}