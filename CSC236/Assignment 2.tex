\documentclass[11pt]{article}
\usepackage{amsmath,amssymb,amsthm,enumerate,nicefrac,fancyhdr,hyperref,graphicx,adj
ustbox}
\hypersetup{colorlinks=true,urlcolor=blue,citecolor=blue,linkcolor=blue}
\usepackage[left=2.6cm, right=2.6cm, top=1.5cm, includehead, includefoot]{geometry}
\usepackage[dvipsnames]{xcolor}
\usepackage[d]{esvect}
\usepackage{listings}
\usepackage{enumitem} % To allow for alph in enumerate
\usepackage{braket}
\usepackage{float} % To allow for H setting in figures.
%% header
\pagestyle{fancy}
\fancyhead[L]{\bf\large CSC236 UTM \\ Assignment 2}
\fancyhead[R]{\bf\large Fall 2024 \\Due Oct 28}
%\fancyfoot[C]{Page \thepage\ of 2}
\setlength{\headheight}{35pt}

\begin{document}
    \textbf{Q1.} On pages 55 and 56 of the textbook there is a proof for the correctness of the program \verb|avg|. The author used the invariant $$Inv(\verb|i|,\verb|sum|)\colon 0\leq \verb|i| < len(A) \wedge \verb|sum|=\sum_{k=0}^{\verb|i|-1}A[k].$$
    As you can see, the invariant is a predicate in two variables: $\verb|i|$, $\verb|sum|$. These two variables are used in the program \verb|avg|, but neither is the variable on which the induction proof is based.
    The author is using simple induction, but it is not very clear what the induction variable is (also the induction predicate itself is ambiguous). We want to make sure you understand what is going on there by having you re-write the proof by yourself in a style similar to the one we use in lectures. Here is the predicate
    you should be proving:
    $$Q(j): \text{At the beginning of the } j^{th} \text{ iteration}, \verb|sum|=\sum_{k=0}^{\verb|i|-1}A[k].$$
    Remark 1: By the program's design, the variables $\verb|i|,\verb|sum|$ may change with each iteration (in other words, both are functions of $j$). This is why it might be more appropriate to write
    
    $$Q(j): \verb|sum|_j=\sum_{k=0}^{\verb|i|_j-1}A[k] \quad \text{or} \quad Q(j): \verb|sum|(j)=\sum_{k=0}^{\verb|i|(j)-1}A[k].$$

    where $\verb|i|_j$ (or $\verb|i|(j)$) means the value of program variable $\verb|i|$ at the beginning of the $j^{th}$ iteration (the same with $\verb|sum|$).
    That said, we believe that the version above Remark 1 is the best to work with as long as one understands that $\verb|i|,\verb|sum|$ are iteration dependent.

    Remark 2: Using $j$ as an index in Remark 1 has a different meaning from that which is intended by the author. The author is using indices to differentiate between the values of $\verb|i|$, $\verb|sum|$ at the beginning of an (arbitrary) iteration and
    their values after the iteration is run.
    \medskip
    The end goal is to prove $Q(len(A))$. A proof by induction will show that $$\forall
    j\in\{1,\ldots,len(A)\}, Q(j).$$ Your proof must follow the style used in lectures.

    \begin{proof}
        It will be shown using simple induction that \(\forall j \in \{1,..., \texttt{len}(A) \},Q(j)\).

        \textbf{Base Case.} Let \(j = 1\). At the beginning of the first iteration of the loop, \(i=0\) and \(sum=0\). Notice that in the expression \(\sum_{k=0} ^{-1} A[k]\), the lower bound is greater than the upper bound. This results in an empty sum that evaluates to \(0\). Therefore the base case holds.

        \textbf{Induction Hypothesis.} Suppose that for some \(j \in \{1,..., \texttt{len}(A)\}\), \(Q(j)\) is true.

        \textbf{Induction Step.} \(i\) increases by 1 at the end of each iteration, so at the beginning of the \(j\)th iteration, \(i = j-1<\texttt{len}(A)\), which means that the loop body is entered. From the induction hypothesis,
        \[
            sum = \sum_{k=0}^{j-2} A[k]
        \]
        Running through the loop, on line 9, \(A[j-1]\) is added to \(sum\). After this line is executed, the new value of sum becomes
        \[
            sum = \sum_{k=0} ^{j-2} A[k] + A[j-1] = \sum_{k=0} ^{j-1} A[k]
        \]
        Then, line 10 executes and the value of \(i\) is incremented and becomes \(j\). After the line is executed, the program returns to the top of loop, signifying the beginning of the \(j+1\)th iteration. \(sum\) remains unchanged since the execution of line 9, and is equal to
        \[
            \sum_{k=0}^{j-1} A[k] = \sum_{k=0}^{i-1} A[k]
        \]
        as desired.

        By the principle of simple induction, at the beginning of the \(j\)th iteration, \(Q(j)\) holds for any \(j \in \{1,2,..., \texttt{len}(A), \texttt{len}(A) +1\}\).

        At the beginning of the \(\texttt{len}(A) +1\)th iteration, \(i = \texttt{len}(A)\), so the loop terminates. The program returns \(sum / \texttt{len}(A)\) on line 11, which is exactly the average of the numbers in \(A\).

    \end{proof}
    \pagebreak
    \textbf{Q2.} --Following up on the Q1-- We mentioned this in class, but it is good to remind you again of the fact that, proving $\forall j\in\{1,\ldots,len(A)\}, Q(j)$ is equivalent to proving $\forall n\in\mathbb{N}, Q'(n)$ where $$Q'(n): 0\leq n<len(A) \Rightarrow Q(n+1).$$ Explain the equivalence.
    \begin{proof}
        It will be proven that the two statements imply each other, that is,
        \[
            \forall j \in \{1,..., \texttt{len}(A)\}, Q(j) \iff \forall n \in \mathbb{N}, 0 \leq n < \texttt{len}(A) \implies Q(n+1)
        \]
        To prove the forward direction, suppose that \(Q(j)\) holds true for all \(j \in \{1,..., \texttt{len}(A)\}\). Let \(n\) be a natural number such that \(0 \leq n < \texttt{len}(A)\). It follows that \(n \in \{0, 1,..., \texttt{len}(A) -1\}\), which implies that \(n+1 \in \{1,..., \texttt{len}(A)\}\). By the original assumption, \(Q(n+1)\) is true.

        Conversely, suppose that for all natural \(n\), if \(0 \leq n <\texttt{len}(A)\) then \(Q(n+1)\) is true. Let \(j \in \{1,..., \texttt{len}(A)\}\). Then \(j-1 \in \{0,...,\texttt{len}(A)-1\}\).  It is clear that \(j-1\) is a natural number and that \(0\leq j <\texttt{len}(A)\). By the assumption in the beginning, \(Q(j)\) is true.

        Thus both statements are equivalent to each other.

    \end{proof}
    \pagebreak
    \textbf{Q3.} Solve questions 6 to 10 on pages 64 to 66 of the textbook. There is a typo in Q8 line 12 (it should be \verb|c = 1| instead of \verb|c == 1|). Proofs must follow the lecture's format and level of clarity. Only \textbf{one} of the five questions will be selected for marking. A serious attempt of all the mentioned questions is a necessary condition for receiving more than 0 points.

    \textit{Question 6.}\begin{enumerate}
        \item Give a loop invariant that characterizes the values of \(a\) and \(y\).
        \begin{proof}
            Define the loop invariant to be
            \[
                Inv(x, y) : a = x + \frac{y(y+1)}{2} - 55
            \]
            It will be proven that at the beginning of the \(i\)th iteration of the loop, \(Inv(x,y)\) is true.

            \textbf{Base Case.} At the beginning of the first iteration, \(a=x\) and \(y=10\). Thus
            \[
                x + \frac{y(y+1)}{2} - 55 = x - \frac{10(11)}{2} - 55 = x - 55 + 55 = x = a
            \]
            The base case has been proven to be true.

            \textbf{Induction Hypothesis.} Now suppose that for some positive natural \(j\), \(Inv(a,y)\) holds true at the beginning of the \(j\)th iteration. \(y\) is initially 10 and is decremmented at the end of every iteration, so at the beginning of the \(j\)th iteration, \(y=10-(j-1) = 11 - j\). Then the loop invariant states that 
            \[
                a = x +\frac{(11-j)(12-j)}{2}-55
            \]

            \textbf{Induction Step.} If \(a \leq 0\), the loop terminates and never reaches the beginning of the \(j+1\)th iteration. Otherwise, on line 6 in the \(j\)th iteration, \(a\) is subtracted by \(y=11-j\). Using the induction hypothesis, the value of \(a\) is now
            \[
                a = x + \frac{(11-j)(12-j)}{2} - 55 - (11-j) = x + \left(\frac{(11-j)(12-j)}{2} - (11 - j)\right)
            \]
            \[
                = x+\frac{(11-j)(12-j-2)}{2}-55 = x+\frac{(10-j)(11-j)}{2}-55
            \]
            Finally, on line 7, the value of \(y\) is decremented by 1 and becomes \(y = 10-j\). The program then arrives at the beginning of the \(j+1\)th iteration. Indeed, it is true that
            \[
                a = x+\frac{(10-j)(11-j)}{2}-55 = x+\frac{y(y+1)}{2}-55
            \]
            Thus \(Inv(a,y)\) is true at the beginning of the \(j+1\)th iteration. Verifying the validity of the loop invariant.

        \end{proof}
        
        \item Show that sometimes this code fails to terminate
        \begin{proof}
            Let \(x=56\). It wil be shown using simple induction that at the beginning of every iteration, \(a > 0\), which implies that the loop can never terminate.

            Consider the beginning of an arbitrary \(i\)th iteration. It is known that \(y = 11 - j\). From the loop invariant found in part (a),
            \[
                a = x + \frac{y(y+1)}{2} - 55 = 56 + \frac{(11-j)(12-j)}{2} - 55 = 1 + \frac{(11-j)(12-j)}{2}
            \]
            Note that \(\frac{(11-j)(12-j)}{2} < 0\) if and only if \(11 < j\) and \(j < 12\) at the same time, which is impossible for \(j \in \mathbb{N}\). Thus \(\frac{(11-j)(12-j)}{2} \geq 0\) so
            \[
                a = 1 + \frac{(11-j)(12-j)}{2} \geq 1 > 0
            \]
            Therefore, \(a >0\) at the beginning of all iterations, which means that the loop never terminates.

        \end{proof}
    \end{enumerate}
    \noindent\textit{Question 7.} State the pre- and post-conditions that the algorithms must satisfy, then prove that each algorithm is correct.
    \begin{enumerate}[label=(\alph*)]
        \item 
        \begin{proof}
            Pre-condition: \(a\) is a non-zero number and \(b\) is a natural number.
            
            Post-condition: Returns the value of \(a^b\).

            The correctness of the program will be proved using complete induction on \(b\). For the entirety of the proof, \(a\) is fixed to be an arbitrary non-zero number.

            \textbf{Base Case.} When \(b=0\), the method returns 1, which is exactly \(a^b\).

            \textbf{Induction Hypothesis.} Suppose that the program returns the correct value of \(a^b\) for all \(b \leq n\), for some \(n \in \mathbb{N}\).

            \textbf{Induction Step.} When \(b = n+1\), \(b \geq 1\), so line 3 is not reached. Consider two possible cases:
            
            If \(b\) is even, line 5 and 6 will be executed. A new variable \(x\) is initialized to be equal to \verb|exp_rec(a, b / 2)|. Since \(\frac{b}{2}\) is natural and less than or equal to \(n\), By the induction hypthesis, the value of \(x\) is \(a^{\frac{b}{2}}\). The program returns \verb|x * x|, which is equal to \(a^{\frac{b}{2}} \cdot a^{\frac{b}{2}} = a^b\) on line 6. 

            If \(b\) is odd, line 8 and 9 will be executed. By a similar argument as the case above, a variable \(x\) is initialized to the value \(a^{\frac{b-1}{2}}\) and \verb|x * x * a| is returned, which has the value of \(a^{\frac{b-1}{2}} \cdot a^{\frac{b-1}{2}} \cdot a = a^b\)

            Therefore, it can be concluded that the program satisfies the postconditions for all natural \(b\).

        \end{proof}

        \item 
        \begin{proof}
            Let \(a\) be a non-zero number and \(b\) be a natural number. For a positive natural \(i\), define the loop invariant
            \[
                P(i) : \texttt{mult}^\texttt{exp} \cdot \texttt{ans} = a^b
            \]
            It will be shown using induction on \(i\) that for \(i \in \{1, 2,..., \lfloor \log _2 (b) \rfloor + 1\}\), \(P(i)\) is true at the beginning of the \(i\)th iteration.

            \textbf{Base Case.} Let \(i=1\). At the start of the very first iteration, \verb|mult = a|, \verb|ans = 1|, and \verb|exp = b|, so it follows that
            \[
                \texttt{mult}^\texttt{exp} \cdot \text{ans} = a^b \cdot 1 = a^b
            \]
            Thus the base case holds.

            \textbf{Induction Hypothesis.} Let \(j \in \{1,2,..., \lfloor \log _2(b) \rfloor\}\). Suppose that at the beginning of the \(j\)th iteration, \(P(j)\) is true. That is,
            \[
                \texttt{mult}^\texttt{exp} \cdot \texttt{ans} = a^b
            \]
            \textbf{Induction Step.} Let \(\texttt{mult}_{0},\texttt{exp}_{0},\texttt{ans}_{0}\) be the value of  \(\texttt{mult},\texttt{exp},\texttt{ans}\) at the beginning of the \(j\)th iteration and let \(\texttt{mult}_{1},\texttt{exp}_{1},\texttt{ans}_{1}\) denote the value of the variables at the beginning of the \(j+1\)th iteration.

            Now, two cases will be considered for the analysis of the loop execution.

            If \verb|exp| is odd, then line 7 will run and multiply \verb|ans| by \verb|mult|, which implies that \(\texttt{ans}_1 = \texttt{ans}_0 \cdot \texttt{mult}_0\).

            Line 8 will run and will multiply \verb|mult| by itself, so \(\texttt{mult}_1 = \texttt{mult}_0^{\, 2}\).
            
            As well, \verb|exp| can be rewritten as \(2n+1\), for some natural \(n\), so line 9 will result in \(\texttt{exp}_1 = \texttt{exp}_0 // 2 = \frac{\texttt{exp}_0 - 1}{2}\)

            After line 9, the current iteration ends and the \(j+1\)th iteration begins. Using algebriac manipulation and the induction hypothesis, \(P(j+1)\) can be verified to be true:
            \[
                \texttt{mult}_1^{\texttt{exp}_1} \cdot \texttt{ans}_1 = (\texttt{mult}_0^{\, 2})^\frac{\texttt{exp}_0 - 1}{2} \cdot (\texttt{ans}_0 \cdot \texttt{mult}_0) = (\texttt{mult}_0)^{\texttt{exp}_0 - 1}(\texttt{mult}_0 \cdot \texttt{ans}_0)
            \]
            \[
                = \texttt{mult}_0 ^{\texttt{exp}_0} \cdot \texttt{ans}_0 = a^b
            \]

            Otherwise if \verb|exp| is even, then line 7 will not run, so \(\texttt{ans}_1 = \texttt{ans}_0\).

            Finally, from line 8, it is seen that \(\texttt{mult}_1 = \texttt{mult}_0^2\)

            \verb|exp| can be written as \(2k\) for some natural \(k\), so line 9 produces \(\texttt{exp}_1 = \texttt{exp}_0 // 2 = \frac{\texttt{exp}_0}{2}\).

            The execution of line 9 denotes the end of the \(j\)th iteration and the start of the \(j+1\)th iteraton. Similar to the previous case, it is seen that 
            \[
                \texttt{mult}_1^{\texttt{exp}_1} \cdot \texttt{ans}_1 = (\texttt{mult}_0^{\, 2})^\frac{\texttt{exp}_0}{2} \cdot (\texttt{ans}_0) = \texttt{mult}_0^{\texttt{exp}_0} \cdot \texttt{ans}_0 = a^b
            \]
            Thus, in either case, \(P(j+1)\) has been shown to hold.

            By the principle of simple induction, it has been shown that \(P(\lfloor \log _2(b) \rfloor + 1)\) is true at the beginning of iteration number \(\lfloor \log _2(b) \rfloor + 1\). Since \verb|exp| was initialized as \(b\) and has been halved \(\lfloor \log _2(b) \rfloor + 1\) times, the value of \verb|exp| at the beginning of the iteration is 0, which means that the loop terminates. Using the fact that \(P(\lfloor \log _2(b) \rfloor + 1)\) holds,
            \[
                \texttt{mult}^\texttt{exp} \cdot \texttt{ans} = \texttt{mult}^0 \cdot \texttt{ans} = \texttt{ans} = a^b
            \]
            The program returns \verb|ans| on line 10, which satisfies its postconditions.
        \end{proof}
    \end{enumerate}
    \textit{Question 8.} Prove that \verb|majority()| is correct.
    \begin{proof}
        
    \end{proof}
    \textit{Question 9.} Studying bubblesort.
    \begin{enumerate}[label=(\alph*)]
        \item State and prove an invariant for the inner loop
        \item State and prove an invariant for the outer loop
        \item Prove that \verb|bubblesort| is correct, according to its specifications
    \end{enumerate}
    \textit{Question 10.} Consider the following generalization of the \verb|min| function.
    \begin{enumerate}[label=(\alph*)]
        \item Prove that this algorithm is correct
        \item Analyze the worse-case running time of this algorithm.
    \end{enumerate}
    \pagebreak
    \textbf{Q4.} Solve questions 6,7,10,12, and 14 on pages 46 to 48 of the textbook. Only \textbf{two} of the five questions will be selected for marking. A serious attempt of all the mentioned questions is a necessary condition for receiving more than 0 points.
    \textit{Question 7.} Let \(T(n)\) denote the number of distinct full binary trees with \(n\) nodes. Give a recurrence for \(T(n)\). Then use induction to prove that \(T(n)\geq \frac{1}{n}2^{(n-1) / 2}\).
\end{document}