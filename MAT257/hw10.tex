\documentclass{article}
\usepackage[margin=1.0in]{geometry}
\usepackage{amssymb,amsmath,amsthm,amsfonts}
\usepackage{enumitem}
\usepackage{xcolor}
\usepackage{mathtools}

% My boxes
\usepackage[breakable]{tcolorbox}

% \RequirePackage{background}
% \backgroundsetup{
%     scale=1,
%     color=black,
%     opacity=1,
%     angle=0,
%     contents={
%         \includegraphics[width=\paperwidth,height=\paperheight]{\nightmodebackground}
%     }
% }

\definecolor{pastelblue}{RGB}{96, 145, 245}
\definecolor{pastelgreen}{RGB}{106, 235, 135}
\definecolor{darkgray}{RGB}{60, 60, 60}
\definecolor{lightgray}{RGB}{180, 180, 180}
\definecolor{offwhite}{RGB}{225, 225, 245}


\pagecolor{darkgray}
\color{offwhite}

\newcommand{\Z}{\mathbf{Z}}
\newcommand{\N}{\mathbf{N}}
\newcommand{\R}{\mathbf{R}}
\newcommand{\Q}{\mathbf{Q}}
\newcommand{\C}{\mathbf{C}}

\newcommand{\id}{\mathrm{id}}
\newcommand{\op}{\mathrm{op}}
\newcommand{\diam}{\mathrm{diam}}
\newcommand{\GL}{\mathrm{GL}}
\newcommand{\Tr}{\mathrm{Tr}}
\newcommand{\im}{\mathrm{im}}
\newcommand{\rank}{\mathrm{rank}}

\newcommand{\cl}[1]{\overline{#1}}

\swapnumbers % places numbers before thm names

\theoremstyle{plain} % The "plain" style italicizes all body text.
	\newtheorem{thm}{Theorem}
		\numberwithin{thm}{section} % Theorem numbers are determined by section.
	\newtheorem{lemma}[thm]{Lemma}
	\newtheorem{prop}[thm]{Proposition}
	\newtheorem{cor}[thm]{Corollary}

\theoremstyle{definition}
    \newtheorem{defn}[thm]{Definition}
	\newtheorem{example}[thm]{Example}
	\newtheorem{exercise}[thm]{Exercise} %Exercise

\begin{document}
    \newtcolorbox{question}[2][]{fonttitle=\large, fontupper=\large, fontlower=\large, title=Question {#2}., oversize, arc=3mm, outer arc=2mm, opacityback=0.9, coltitle=offwhite, colframe=pastelblue, colback=darkgray, colupper=lightgray, collower=lightgray, leftrule=1mm, rightrule=1mm, toprule=1.5mm, titlerule=1mm, bottomrule=1mm, valign=center, add to natural height=5mm, lower separated=false, before lower=\begin{proof}, after lower= \\ \end{proof}, #1, breakable=true}

    \begin{question}{27}
        Let $f:\R^2\rightarrow \R$ be a continuously differentiable function.
    \begin{enumerate}[label=(\alph*)]
        \item  Show that the partial function $\R\rightarrow \R$, $t\mapsto f(x,t)$ is integrable (over any bounded interval in $\R$).
        
        \item By (a), we can define a function $\varphi:\R\rightarrow \R$ by
            \[ \varphi(x) = \int_a^b f(x,t) \ dt. \]
        Show that $\varphi$ is differentiable, and that its (classical) derivative is given by
            \[ \dfrac{d\varphi}{dx}(x_0) = \int_a^b \dfrac{\partial f}{\partial x}(x_0,t) \ dt. \]
        This formula is known as \textbf{differentiation under the integral sign}, or \textbf{Feynmann's trick}.
    
        \item Use Feynmann's trick to solve the single-variable integral:
            \[ \int_0^\infty e^{- t^2} \ dt \]
    \end{enumerate}
    \tcblower
    (a):

    Since \(f\) is continuous, it follows immediately that its partial function is continuous, which implies that it is integrable.

    (b):

    Let \(x_0 \in \mathbb{R}\). We will show that as \(h\to 0\), \[\frac{1}{h} \left(\int _a^b f(x_0 + h,t)dt - \int _a^b f(x_0, t)dt - h\int _a^b \frac{\partial f}{\partial x} (x_0,t)dt\right) \to 0,\] which is equivalent to saying
    \[
        \dfrac{d\varphi}{dx}(x_0) = \int_a^b \dfrac{\partial f}{\partial x}(x_0,t) \ dt.
    \]
    Let \(\varepsilon > 0\). By the partial differentiability of \(f\), we obtain a \(\delta\) so that \[\left\vert f(x_0 + h, t) - f(x_0, t) - h\frac{\partial f}{\partial x}(x_0, t) \right\vert < \dfrac{|h|\varepsilon}{b-a}\] for all \(0 < |h| <\delta\).
    
    Fix \(h \in \mathbb{R}\) so that \(0 < |h| <\delta\). By the linearity of the integral,
    \[
        \left\vert \frac{1}{h} \left(\int _a^b f(x_0 + h,t)dt - \int _a^b f(x_0, t)dt - h\int _a^b \frac{\partial f}{\partial x} (x_0,t)dt\right) \right\vert
    \]
    \[
        = \left\vert\frac{1}{h}\int _a^b (f(x_0 + h, t) - f(x_0, t) - h\frac{\partial f}{\partial x} (x_0,t))dt\right\vert 
    \]
    \[
        \leq \frac{1}{|h|} \int _a^b \left\vert f(x_0 + h, t) - f(x_0, t) - h\frac{\partial f}{\partial x} (x_0,t) \right\vert dt
    \]
    \[
        < \frac{1}{|h|}\int _a^b \frac{|h|\varepsilon}{b-a}dt <\varepsilon
    \]
    as desired. Thus we have found an expression for \(\varphi ' (x_0)\).

    (c):

    Let \(I = \int _0^{\infty} e^{-t^2}\). Define \(\varphi : \mathbb{R} \to  \mathbb{R}\) by
    \[
        \varphi (x) = \int _0^{\infty} \frac{e^{-x^2(t^2 + 1)}}{t^{2} +1}dt
    \]
    From part (b), we see that
    \[
        \varphi '(x) = -2x\int _0^{\infty} e^{-x^2 (t^2 + 1)}dt = -2xe^{-x^2}\int _0^{\infty} e^{-x^2 t^2}dt
    \]
    We perform the substitution \(u = xt\). Changing \(u\) back to \(t\) gives us
    \[
        \varphi '(x) = -2e^{-x^2}\int _0^{\infty} e^{-t^2}dt
    \]
    Thus we have that
    \[
        \varphi '(x) = -2e^{-x^2}I
    \]
    We can take the definite integral of both sides with respect to \(x\):
    \[
        \int _0 ^{\infty} \varphi '(x) dx = \int _0^{\infty} -2e^{-x^2}I dx \implies \lim_{n \to \infty} \varphi (n) - \varphi (0) = -2I^2
    \]
    Now we will analyse \(\lim_{n \to \infty} \varphi (n)\) and \(\varphi (0)\) separately.

    First, we will show that \(\lim_{n \to \infty} \varphi (n) = 0\). Let \(\varepsilon > 0\). For \(n > N\), where \(N\) is a fixed number, we have that \(e^{-n^2(t^2 +1)} < \dfrac{2\varepsilon}{\pi}\), so
    \[
        |\varphi (n)| = \left\vert \int _0^{\infty} \frac{e^{-n^2(t^2 + 1)}}{t^{2} +1}\ dt \right\vert \leq \int _0^{\infty} \left\vert \frac{e^{-n^2(t^2 + 1)}}{t^{2} +1}\ dt \right\vert < \frac{2}{\pi} \int _0^{\infty} \frac{\varepsilon}{t^2 + 1}\ dt
    \]
    \[
        \implies \frac{2\varepsilon}{\pi} \arctan (t) \Big| _0^{\infty} = \varepsilon.
    \]
    Thus we can conclude that \(\lim_{n \to \infty} \varphi (n) = 0\).

    Now, notice that
    \[
        \varphi (0) = \int _0^{\infty} \frac{1}{t^2 + 1} \ dt = \arctan (t) \ \Big|_0^{\infty} = \frac{\pi}{2}
    \]

    Applying this to our original equation,
    \[
        0 - \frac{\pi}{2} = -2I^2 \implies 2I^2 = \frac{\pi}{2} \implies I^2 = \frac{\pi}{4}
    \]
    Finally, taking the squareroot of both sides gives us the result:
    \[
        \int_0^\infty e^{- t^2} \ dt = I = \frac{\sqrt{\pi}}{2}
    \]
    \end{question}
    \newpage
    \begin{question}{28}
        Let $U$ be an open set in a normed vector space $X$ and let $f:U\rightarrow Y$ be a \textbf{twice continuously differentiable} function, meaning that the second derivative $f'':U\rightarrow B(X,B(X,Y))$ exists and is continuous on $U$. We also say that $f$ is a \textbf{$C^2$-function}.
        \begin{enumerate}[label=(\alph*)]
            \item Let $f:\R^2\rightarrow \R$ be given by $f(x,y)=x^2-xy+y^2$. Find, with proof, an explicit formula for the linear mapping $f''(2,-1)$. Also, write down the matrix that represents this linear mapping with respect to a suitable ``standard'' basis.
            
            \item Now we investigate the case $X=\R^n$ and $Y=\R$, and let $f:U\rightarrow \R$ be some function defined on an open set $U\subseteq \R^n$. We use the notation $\dfrac{\partial^2 f}{\partial x_i\partial x_j}$ to refer to the \textbf{$(i,j)$th second partial derivative} of $f$: this is the $i$th partial derivative of the $j$th partial derivative $\dfrac{\partial f}{\partial x_j}$.
    

            \begin{enumerate}[label=(\roman*)]
                \item Show that $f$ is twice continuously differentiable if and only if all second partial derivatives exist and are continuous.

                \item Let $f$ be twice continuously differentiable Let $v\in \R^n$ and let $D_vf:U\rightarrow \R$ be the directional derivative of $f$ along $v$. Show that $D_vf$ is continuously differentiable.

                \item Let $f$ be twice continuously differentiable and let $v\in \R^n$. By (ii), we know that $D_vf$ is $C^1$, hence differentiable in every direction $\in \R^n$. Show that the directional derivatives commute:
                    \[ D_v(D_wf) = D_w(D_vf) \quad \text{for all $v,w\in \R^n$.} \]
                \item Deduce \textbf{Clairaut's Theorem}: that the second partial derivatives commute.
                \[ \dfrac{\partial^2 f}{\partial x_i\partial x_j} = \dfrac{\partial^2 f}{\partial x_j\partial x_i} \quad \text{for all $i,j\in \{1,\ldots,n\}$.} \]
            \end{enumerate}
        \end{enumerate}
        \tcblower
        (a):

        We claim that \(f''(2,-1) : \mathbb{R}^2 \to B(\mathbb{R}^2,\mathbb{R})\) is a bounded linear map which is represented by the matrix
        \[
            [f''] = \begin{bmatrix}
                2 & -1 \\
                -1 & 2 \\
            \end{bmatrix}
        \]

        (b)(i):

        (b)(ii):

        (b)(iii):

        (b)(iv):
    \end{question}
\end{document}