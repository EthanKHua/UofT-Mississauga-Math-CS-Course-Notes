\documentclass{article}
\usepackage[margin=1.0in]{geometry}
\usepackage{amssymb,amsmath,amsthm,amsfonts}
\usepackage{enumitem}
\usepackage{xcolor}
\usepackage{mathtools}

% My boxes
\usepackage[breakable]{tcolorbox}

% \RequirePackage{background}
% \backgroundsetup{
%     scale=1,
%     color=black,
%     opacity=1,
%     angle=0,
%     contents={
%         \includegraphics[width=\paperwidth,height=\paperheight]{\nightmodebackground}
%     }
% }

\definecolor{pastelblue}{RGB}{96, 145, 245}
\definecolor{pastelgreen}{RGB}{106, 235, 135}
\definecolor{darkgray}{RGB}{60, 60, 60}
\definecolor{lightgray}{RGB}{180, 180, 180}
\definecolor{offwhite}{RGB}{225, 225, 245}


\pagecolor{darkgray}
\color{offwhite}

\newcommand{\Z}{\mathbf{Z}}
\newcommand{\N}{\mathbf{N}}
\newcommand{\R}{\mathbf{R}}
\newcommand{\Q}{\mathbf{Q}}
\newcommand{\C}{\mathbf{C}}

\newcommand{\id}{\mathrm{id}}
\newcommand{\op}{\mathrm{op}}
\newcommand{\diam}{\mathrm{diam}}
\newcommand{\GL}{\mathrm{GL}}
\newcommand{\Tr}{\mathrm{Tr}}
\newcommand{\im}{\mathrm{im}}
\newcommand{\rank}{\mathrm{rank}}

\newcommand{\cl}[1]{\overline{#1}}

\swapnumbers % places numbers before thm names

\theoremstyle{plain} % The "plain" style italicizes all body text.
	\newtheorem{thm}{Theorem}
		\numberwithin{thm}{section} % Theorem numbers are determined by section.
	\newtheorem{lemma}[thm]{Lemma}
	\newtheorem{prop}[thm]{Proposition}
	\newtheorem{cor}[thm]{Corollary}

\theoremstyle{definition}
    \newtheorem{defn}[thm]{Definition}
	\newtheorem{example}[thm]{Example}
	\newtheorem{exercise}[thm]{Exercise} %Exercise

\begin{document}
    \newtcolorbox{question}[2][]{fonttitle=\large, fontupper=\large, fontlower=\large, title=Question {#2}., oversize, arc=3mm, outer arc=2mm, opacityback=0.9, coltitle=offwhite, colframe=pastelblue, colback=darkgray, colupper=lightgray, collower=lightgray, leftrule=1mm, rightrule=1mm, toprule=1.5mm, titlerule=1mm, bottomrule=1mm, valign=center, add to natural height=5mm, lower separated=false, before lower=\begin{proof}, after lower= \smallbreak \end{proof}, #1, breakable=true}

    \begin{question}{27}
        Let $f:\R^2\rightarrow \R$ be a continuously differentiable function.
    \begin{enumerate}[label=(\alph*)]
        \item  Show that the partial function $\R\rightarrow \R$, $t\mapsto f(x,t)$ is integrable (over any bounded interval in $\R$).
        
        \item By (a), we can define a function $\varphi:\R\rightarrow \R$ by
            \[ \varphi(x) = \int_a^b f(x,t) \ dt. \]
        Show that $\varphi$ is differentiable, and that its (classical) derivative is given by
            \[ \dfrac{d\varphi}{dx}(x_0) = \int_a^b \dfrac{\partial f}{\partial x}(x_0,t) \ dt. \]
        This formula is known as \textbf{differentiation under the integral sign}, or \textbf{Feynmann's trick}.
    
        \item Use Feynmann's trick to solve the single-variable integral:
            \[ \int_0^\infty e^{- t^2} \ dt \]
    \end{enumerate}
    \tcblower
    (a):

    Since \(f\) is continuous, it follows immediately that its partial function is continuous, which implies that it is integrable.

    (b):

    Let \(x_0 \in \mathbb{R}\). We will show that as \(h\to 0\), \[\frac{1}{h} \left(\int _a^b f(x_0 + h,t)dt - \int _a^b f(x_0, t)dt - h\int _a^b \frac{\partial f}{\partial x} (x_0,t)dt\right) \to 0,\] which is equivalent to saying
    \[
        \dfrac{d\varphi}{dx}(x_0) = \int_a^b \dfrac{\partial f}{\partial x}(x_0,t) \ dt.
    \]
    Let \(\varepsilon > 0\). By the partial differentiability of \(f\), we obtain a \(\delta\) so that \[\left\vert f(x_0 + h, t) - f(x_0, t) - h\frac{\partial f}{\partial x}(x_0, t) \right\vert < \dfrac{|h|\varepsilon}{b-a}\] for all \(0 < |h| <\delta\).
    
    Fix \(h \in \mathbb{R}\) so that \(0 < |h| <\delta\). By the linearity of the integral,
    \[
        \left\vert \frac{1}{h} \left(\int _a^b f(x_0 + h,t)dt - \int _a^b f(x_0, t)dt - h\int _a^b \frac{\partial f}{\partial x} (x_0,t)dt\right) \right\vert
    \]
    \[
        = \left\vert\frac{1}{h}\int _a^b (f(x_0 + h, t) - f(x_0, t) - h\frac{\partial f}{\partial x} (x_0,t))dt\right\vert 
    \]
    \[
        \leq \frac{1}{|h|} \int _a^b \left\vert f(x_0 + h, t) - f(x_0, t) - h\frac{\partial f}{\partial x} (x_0,t) \right\vert dt
    \]
    \[
        < \frac{1}{|h|}\int _a^b \frac{|h|\varepsilon}{b-a}dt =\varepsilon
    \]
    as desired. Thus we have found an expression for \(\varphi ' (x_0)\).

    (c):

    Let \(I = \int _0^{\infty} e^{-t^2}\ dt\). We will first show that this integral converges. We make the observation that the integrand is continuous and positive. Splitting the integral in two, we see that
    \[
        \int _0^{\infty} e^{-t^2}\ dt = \int _0^{1} e^{-t^2}\ dt + \int _1^{\infty} e^{-t^2}\ dt
    \]
    The first integral has finite bounds so it is convergent. For \(t \geq 1\), \(-t^2 \leq -t\). Since \(\exp (t)\) is an increasing function,
    \[
        e^{-t^2} \leq e^{-t} \implies \int _1^{\infty} e^{-t}\ dt \leq \int _1^{\infty} e^{-t}\ dt = 1
    \]
    Therefore \(I\) converges by the basic comparison test.
    
    Define \(\varphi : [0,\infty) \to  \mathbb{R}\) by
    \[
        \varphi (x) = \int _0^{\infty} \frac{e^{-x^2(t^2 + 1)}}{t^{2} +1}\ dt
    \]
    Notice that \(\int _0^{\infty} \frac{e^{-x^2(t^2 + 1)}}{t^{2} +1}\ dt \leq \int _0^{\infty} e^{-x^2(t^2 + 1)}\ dt\), so \(\varphi\) always converges for similar reasons.

    From part (b), we see that
    \[
        \varphi '(x) = -2x\int _0^{\infty} e^{-x^2 (t^2 + 1)}dt = -2xe^{-x^2}\int _0^{\infty} e^{-x^2 t^2}dt
    \]
    We perform the substitution \(u = xt\). Changing \(u\) back to \(t\) gives us
    \[
        \varphi '(x) = -2e^{-x^2}\int _0^{\infty} e^{-t^2}dt
    \]
    Thus we have that
    \[
        \varphi '(x) = -2e^{-x^2}I
    \]
    We can take the definite integral of both sides with respect to \(x\):
    \[
        \int _0 ^{\infty} \varphi '(x) dx = \int _0^{\infty} -2e^{-x^2}I dx \implies \lim_{n \to \infty} \varphi (n) - \varphi (0) = -2I^2
    \]
    Now we will analyse \(\lim_{n \to \infty} \varphi (n)\) and \(\varphi (0)\) separately.

    First, we will show that \(\lim_{n \to \infty} \varphi (n) = 0\). Let \(\varepsilon > 0\). For \(n > N\), where \(N\) is a fixed number, we have that \(e^{-n^2(t^2 +1)} < \dfrac{2\varepsilon}{\pi}\), so
    \[
        |\varphi (n)| = \left\vert \int _0^{\infty} \frac{e^{-n^2(t^2 + 1)}}{t^{2} +1}\ dt \right\vert \leq \int _0^{\infty} \left\vert \frac{e^{-n^2(t^2 + 1)}}{t^{2} +1}\ dt \right\vert < \frac{2}{\pi} \int _0^{\infty} \frac{\varepsilon}{t^2 + 1}\ dt
    \]
    \[
        \implies \frac{2\varepsilon}{\pi} \arctan (t) \Big| _0^{\infty} = \varepsilon.
    \]
    Thus we can conclude that \(\lim_{n \to \infty} \varphi (n) = 0\).

    Now, notice that
    \[
        \varphi (0) = \int _0^{\infty} \frac{1}{t^2 + 1} \ dt = \arctan (t) \ \Big|_0^{\infty} = \frac{\pi}{2}
    \]

    Applying this to our original equation,
    \[
        0 - \frac{\pi}{2} = -2I^2 \implies 2I^2 = \frac{\pi}{2} \implies I^2 = \frac{\pi}{4}
    \]
    Finally, taking the squareroot of both sides gives us the result:
    \[
        \int_0^\infty e^{- t^2} \ dt = I = \frac{\sqrt{\pi}}{2}
    \]
    \end{question}
    \newpage
    \begin{question}{28}
        Let $U$ be an open set in a normed vector space $X$ and let $f:U\rightarrow Y$ be a \textbf{twice continuously differentiable} function, meaning that the second derivative $f'':U\rightarrow B(X,B(X,Y))$ exists and is continuous on $U$. We also say that $f$ is a \textbf{$C^2$-function}.
        \begin{enumerate}[label=(\alph*)]
            \item Let $f:\R^2\rightarrow \R$ be given by $f(x,y)=x^2-xy+y^2$. Find, with proof, an explicit formula for the linear mapping $f''(2,-1)$. Also, write down the matrix that represents this linear mapping with respect to a suitable ``standard'' basis.
            
            \item Now we investigate the case $X=\R^n$ and $Y=\R$, and let $f:U\rightarrow \R$ be some function defined on an open set $U\subseteq \R^n$. We use the notation $\dfrac{\partial^2 f}{\partial x_i\partial x_j}$ to refer to the \textbf{$(i,j)$th second partial derivative} of $f$: this is the $i$th partial derivative of the $j$th partial derivative $\dfrac{\partial f}{\partial x_j}$.
    

            \begin{enumerate}[label=(\roman*)]
                \item Show that $f$ is twice continuously differentiable if and only if all second partial derivatives exist and are continuous.

                \item Let $f$ be twice continuously differentiable Let $v\in \R^n$ and let $D_vf:U\rightarrow \R$ be the directional derivative of $f$ along $v$. Show that $D_vf$ is continuously differentiable.

                \item Let $f$ be twice continuously differentiable and let $v\in \R^n$. By (ii), we know that $D_vf$ is $C^1$, hence differentiable in every direction $\in \R^n$. Show that the directional derivatives commute:
                    \[ D_v(D_wf) = D_w(D_vf) \quad \text{for all $v,w\in \R^n$.} \]
                \item Deduce \textbf{Clairaut's Theorem}: that the second partial derivatives commute.
                \[ \dfrac{\partial^2 f}{\partial x_i\partial x_j} = \dfrac{\partial^2 f}{\partial x_j\partial x_i} \quad \text{for all $i,j\in \{1,\ldots,n\}$.} \]
            \end{enumerate}
        \end{enumerate}
        \tcblower
        (a):

        We claim that \(f''(2,-1) : \mathbb{R}^2 \to B(\mathbb{R}^2,\mathbb{R})\) is a bounded linear map given by
        \[
            f''(2,-1)(p,q)(x,y) = (2p - q)x - (p - 2q)y
        \]
        Let \((p,q), (r,s) \in \mathbb{R}^2\) and \(c \in \mathbb{R}\). Then for all \((x,y) \in \mathbb{R}^2\),
        \[
            f''(2,-1)(cp + r, cq + s)(x,y) = [2(cp + r) - (cq + s)]x - [(cp + r) - 2(cq + s)]y
        \]
        \[
            = c[(2p - q)x - (p - 2q)y] + [(2r - s)x - (r - 2s)y] = cf''(2,-1)(p,q)(x,y) + f''(2,-1)(r,s)
        \]
        so \(f''(2,-1)\) is linear.

        Recall that \(f'(p,q)(x,y) = (2p - q)x - (p - 2q)y\). In particular, for \((p,q) = (2,-1)\), \(f'(2,-1)(x,y) = 5x - 4y\).
        
        Fix \((x,y) \in \mathbb{R}^2\). Then
        \[
            \lim_{h \to 0} \frac{f'(2 + h_1, -1 + h_2)(x,y) - f'(2,-1)(x,y) - L_p(h)}{\|h\|}
        \]
        \[
            = \lim_{h \to 0} \frac{[(5 + 2h_1 - h_2)x - (4 + h_1 - 2h_2)y] - [5x - 4y] - L_p(h)}{\|h\|}
        \]
        \[
            = \lim_{h \to 0} \frac{(2h_1 - h_2)x - (h_1 - 2h_2)y - L_p(h)}{\|h\|} = 0
        \]
        which verifies that \(f''(2,-1)(p,q)(x,y) = (2p - q)x - (p - 2q)y\).

        (b):

        Let \(f : U \to \mathbb{R}\) be a function, where \(U\) is open in \(\mathbb{R}^n\).

        (b)(i):

        Suppose that \(f\) is twice continuously differentiable. This implies that all its partial derivatives exist.

        Consider an arbitrary partial derivative \(\dfrac{\partial f}{\partial x_j}\) for \(i \in \{1, ..., n\}\). It will be shown that their directional derivatives all exist and are continuous.

        Let \(v \in \mathbb{R}^n\). We see that
        \[
            \lim_{t \to 0} \frac{1}{t}\left( \frac{\partial f}{\partial x_j} (p+tv) - \frac{\partial f}{\partial x_j} (p) \right) 
        \]
        Recall that \(f'(q)(v) = \sum_{i=1} ^n v_i \frac{\partial f}{\partial x_i} (q)\). Now, notice that
        \[
            \frac{\partial f}{\partial x_j} (p+tv) = f'(p+tv)(x_j) \text{ and } \frac{\partial f}{\partial x_j} (p) = f'(p)(x_j)
        \]
        We make this substitution in the limit expression, so it becomes
        \[
            \lim_{t \to 0} \frac{1}{t}(f'(p+tv)(x_j) - f'(p)(x_j)) = D_v f'(p)(x_j)
        \]
        by definition. Thus \(D_v \left( \frac{\partial f}{\partial x_j} (p)\right) = D_v f'(p)(x_j)\). This is continuous because every directional derivative of \(f'\) is continuous by our assumption. Taking \(v = x_i\), we conclude that all the second partial derivatives of \(f\) exist and are continuous.
        
        Conversely, suppose that all second partial derivatives of \(f\) exist and are continuous. Since all first partial derivatives will exist and are continuous, then \(f\) is continuously differentiable, meaning that \(f'\) exists.

        Let \(i \in \{1, ..., n\}\). We claim that \(\dfrac{\partial f'}{\partial x_i}\) exists and is continuous, where \(x_i\) represents the \(i\)th standard ordered basis vector of \(\mathbb{R}^n\).

        Fix \(p \in \mathbb{R}^n\). Recall that \(f'(p)(\vec{v}) = \sum_{j=1} ^n v_j \frac{\partial f}{\partial x_j}(p)\) from the differentiability theorem. For \(\vec{v} \in \mathbb{R}^n\), we have that
        \[
            \lim_{h \to 0} \frac{f'(p + hx_i)(\vec{v}) - f'(p)(\vec{v})}{h} = \lim_{h \to 0} \sum_{j=1} ^n \frac{v_i}{h}\left( \frac{\partial f}{\partial x_j} (p + hx_i) - \frac{\partial f}{\partial x_j} (p) \right)
        \]
        Since all second partial derivatives exist, the limit evaluates to
        \[
            \sum_{j=1} ^n v_i \frac{\partial ^2 f}{\partial x_i \partial x_j} (p)
        \]
        By definition, this is the \(i\)th partial derivative of \(f'\).

        Notice that it is also continuous, as it is a finite linear combination of second partial derivatives, which are continuous by assumption.
        
        Since we have that all partial derivatives of \(f'\) exist and are continuous, by the differentiability theorem, \(f'\) is continuously differentiable, which implies that \(f\) is twice continuously differentiable.

        Therefore we can conclude that twice continuously differentiable is equivalent to having all second partial derivatives exist and continuous.

        \medskip

        (b)(ii):

        Suppose that \(f\) is twice continuously differentiabe. Consider the directional derivative \(D_v f\). We will show that for all \(p\in U\), \(D_v f(p) = f'(p)(v)\).

        We can assume that \(p + v \in U\) by making \(\|v\|\) small enough. Define a function \(\alpha : [0,1] \to U\) by \(\alpha (t) = p + tv\). The derivative is given by \(\alpha '(t) = v\). Now, construct a new function \(g : \mathbb{R} \to \mathbb{R}\) defined by \(g(t) = (f \circ \alpha) (t)\). By the chain rule, we see that \(g'(t) = f'(\alpha (t)) (v)\).

        Thus we get that
        \[
            D_v f(p) = \lim_{h \to 0} \frac{f(p + hv) - f(p)}{h} = \lim_{h \to 0} \frac{g(h) - g(0)}{h} = g'(0) = f'(\alpha (0))(v) = f'(p)(v),
        \]
        Now, we can compute the derivative of \(D_{v} f\) directly and obtain that \((D_v f)'(p) = (f')'(p)(v)\). But we know by that \(f\) is twice continuously differentiable, so \((D_v f)'\) is continuous.

        Thus \(D_v f\) is continuously differentiable.
        \medskip

        (b)(iii):

        We will evaluate \(D_v (D_w f)\) from the definition and show that it is equal to \(D_w (D_v f)\).

        Fix \(p \in U\). There exists \(\delta > 0\) so that \(p + \delta v \in U\). For any such \(h\), construct a function \(g : [0,\delta] \to \mathbb{R}\) defined as
        \[
            g(t) = D_w f(p + tv)
        \]
        \(g\) is differentiable on its domain because it is a composition of differentiable functions. By the chain rule, we see that
        \[
            g'(t) = (D_w f)'(p+tv)(v)
        \]
        Using the fact that for all continuously differentiable functions, \(D_v h (q) = h'(q)(v)\), we get
        \[
            g'(t) = D_v (D_w f) (p+tv)
        \]
        Now we compute \(D_v (D_w f)\):
        \[
            \lim_{h \to 0} \frac{D_w f(p+hv) - D_w f(p)}{h}
        \]
        For \(|h| < \delta\), the limit becomes
        \[
            D_w (D_v f) (p) = \lim_{h \to 0} \frac{g(h) - g(0)}{h} = g'(0) = D_v (D_w f) (p)
        \]
        Therefore the directional derivatives commute.
        \medbreak
        (b)(iv):

        From the previous part, simply take \(v = x_i\) and \(w = x_j\) and the result follows immediately.
    \end{question}
    \newpage
    \begin{question}{29}
        Let $U\subseteq \R^2$ be an open set, and let $f:U\rightarrow \R$ be some differentiable function. This data defines an \textbf{explicit surface} in $\R^3$, also known as the \textbf{graph} of $f$:
        \[ S = S(f) = \{(x,y,z)\in \R^3 : (x,y)\in U, z=f(x,y)\}. \]
        Colloquially, we often say ``Let $z=f(x,y)$ be a surface in $\R^3$.''
        
        \begin{enumerate}[label=(\alph*)]
            \item Find the equation of the tangent plane to the surface $z=x+xy^2-y^3$ at the point $p=(2,1,3)$, as follows.
            \begin{enumerate}[label=(\roman*)]
                \item First, fix $x=2$ and set $y=1+t$; write $z$ as a function of $t$, and find $z'(0)$. This is the ``slope in the $x$ direction.''
                \item Similarly, find the ``slope in the $y$ direction.''
                \item Now you have two slopes in orthogonal directions. This gives you two vectors which span a plane. Shift this plane so that it becomes tangent at $p$. Write the equation of this plane in the form $Ax+By+Cz=D$, where $A,B,C,D\in \R$.
            \end{enumerate}
            
            \item Find --- with proof! --- the equations of all planes in $\R^3$ which (i) are tangent to the surface $z=x+xy^2-y^3$; (ii) are parallel to the vector $\vec{v}=\begin{bmatrix} 3 \\ 1 \\ 1\end{bmatrix}$; and (iii) pass through the point $p=(-1,-2,3)$.
            
            \item Let $f:\R^2\rightarrow \R$ be a continuously differentiable function, and define a new function $g:\R^2\rightarrow \R$ by
                \[ g(x,y) = f(f(xy,x),f(y,xy)). \]
            You are given the following information about $f$ and $g$:
                \begin{itemize}
                    \item $f(3,1)=5$ and $\nabla f(3,1)=(1,2)$
                    \item $f(3,3)=2$ and $\nabla f(3,3)=(0,-1)$
                    \item $g(1,3)=6$ and $\nabla g(1,3)=(3,4)$
                \end{itemize}
            Find $\nabla f(5,2)$.
        \end{enumerate}
        \tcblower
        (a)(i):

        Fix \(x = 2\) and set \(y = 1+t\). Then
        \[
            z_y(t) = 2 + 2(1+t)^2 - (1+t)^3 \implies z'(t) = 4(1+t) - 3(1+t)^2
        \]
        \[
            \implies z_y'(0) = 1
        \]

        \medskip

        (a)(ii):

        Fix \(y = 1\) and set \(x = 2 + t\). Then
        \[
            z_x(t) = (2+t) + (2+t) - 1 = 2t + 3 \implies z_x'(0) = 2
        \]

        \medskip

        (a)(iii):

        From the previous two parts gave us two vectors \((1,0,2)\) and \((0,1,1)\). This spans the plane given by the equation \(2x+y-z = 0\). Translate this plane by the vector \((2,1,3)\) which yields the equation
        \[
            2(x-2)+(y-1)-(z-3) = 0 \implies 2x+y-z=2
        \]
        \medbreak

        (b):

        We claim that there are only two planes that satisfy all three conditions:
        \[
            5x - 14y - z = 20 \text{ and } 5x - 11y - 4z = 5
        \]

        Let \((p,q) \in \mathbb{R}^n\) and suppose that the plane tangent to the surface \(z\) at this point has the properties above. We find the equation of this tangent plane like the previous part.

        Fix \(x = p\) and set \(y = q + t\). Then
        \[
            z_y(t) = p + p(q+t)^2 - (q+t)^3 \implies z_y'(t) = 2p(q+t) - 3(q+t)^2
        \]
        \[
            \implies z_y'(0) = 2pq - 3q^2
        \]

        Next, fix \(y = q\) and set \(x = p + t\). Then
        \[
            z_x(t) = (x+t) + (x+t)q^2 - q^3 \implies z_x'(t) = q^2 + 1
        \]

        This gives two vectors \((0, 1, 2pq - 3q^2)\) and \((1,0,q^2 + 1)\) that span a plane given by the equation \((q^2 + 1)x + (2pq - 3q^2)y - z = 0\). Translating this plane by \((p,q,p+pq^2 - q^3)\) gives another plane with the equation
        \[
            (q^2 + 1)(x-p) + (2pq - 3q^2)(y-q) - (z + q^3 - pq^2 - p) = 0
        \]
        \[
            \implies (q^2 + 1)x + (2pq - 3q^2)y - z = p(q^2 + 1) + q(2pq - 3q^2) + (q^3 - pq^2 - p)
        \]
        \[
            \implies (q^2 + 1)x + (2pq - 3q^2)y - z = 2pq^2 - 2q^3
        \]

        Now, take the initial vectors \((0, 1, 2pq - 3q^2)\) and \((1,0,q^2 + 1)\) and compute their cross product to find the normal vector \(\vec{n}\):
        \[
            \vec{n} = (0,1,2pq-3q^2) \times (1,0,q^2+1) = (q^2 + 1, 2pq - 3q^2, -1)
        \]
        This vector is normal to the plane. We only want planes that are parallel to \(\vec{v}\), which means that this normal vector should be normal to \(\vec{v}\) as well. We calculate the dot product of this normal vector and \(\vec{v}\):
        \[
            \vec{n} \cdot \vec{v} = 3(q^2 + 1) + (2pq - 3q^2) - 1
        \]
        Set this equal to 0. Simplifying, we have that
        \[
            3(q^2 + 1) + (2pq - 3q^2) - 1 = 0 \implies 2pq = -2
        \]
        \[
            \implies pq = -1
        \]
        We see that \(p,q \neq 0\) and get the relationship \(p = -\dfrac{1}{q}\). Substitute this into our plane equation and we get that
        \[
            (q^2 + 1)x + (2pq - 3q^2)y - z = 2pq^2 - 2q^3
        \]
        \[
            \implies (q^2 + 1)x + (-2 - 3q^2)y - z = -2q - 2q^3
        \]
        \[
            \implies (q^2 + 1)x - (3q^2 + 2)y - z = -2q(q^2 + 1)
        \]
        Finally, recall that our plane should pass through the point \(p = (-1,-2,3)\). Substituting \((x,y,z) = (-1,-2,3)\), we can solve for \(q\) and obtain our answer:
        \[
            -(q^2 + 1) + 2(3q^2 + 2) - 3 = -2q(q^2 + 1)
        \]
        \[
            \implies 2q^3 + 5q^2 + 2q = 0
        \]
        We disregard the case where \(q=0\) and end up with the quadratic equation
        \[
            2q^2 + 5q^2 + 2 = 0 \implies q = \frac{-5 \pm \sqrt{25 - 16}}{4} \implies q_1 = -2, q_2 = -\frac{1}{2}
        \]
        Substituting these values into our plane equation, we see that there are two planes that satisfy the above properties:
        \[
            5x - 14y - z = 20 \text{ and } 5x - 11y - 4z = 5
        \]

        (c):

        In general,
        \[
            \nabla g(p,q) = \left( \frac{\partial g}{\partial x} (p,q) , \frac{\partial g}{\partial y} (p,q) \right) 
        \]
        
        For \(a(x,y) = xy\), \(b(x,y) = x\), and \(c(x,y) = y\), we make the quick note that
        \[
            \begin{bmatrix}
                 a'(x,y) \\
            \end{bmatrix}
            =
            \begin{bmatrix}
                y &  x \\
            \end{bmatrix}
            \text{, } 
            \begin{bmatrix}
                 b'(x,y) \\
            \end{bmatrix}
            =
            \begin{bmatrix}
                1 &  0 \\
            \end{bmatrix}
            \text{, and } 
            \begin{bmatrix}
                 c'(x,y) \\
            \end{bmatrix}
            =
            \begin{bmatrix}
                0 &  1 \\
            \end{bmatrix}
        \]

        Differentiating \(g\) at \((1,3)\), by multiple applications of the chain rule, we get
        \[
            \begin{bmatrix}
                 g'(1,3) \\
            \end{bmatrix}
            = \begin{bmatrix}
                 f'(f(3,1),f(3,3)) \\
            \end{bmatrix}
            \times \begin{bmatrix}
                 \begin{bmatrix}
                     f'(3,1) \\
                 \end{bmatrix}
                 \times 
                 \begin{bmatrix}
                     \begin{bmatrix}
                         a'(1,3) \\
                     \end{bmatrix} \\
                     \begin{bmatrix}
                         b'(1,3) \\
                     \end{bmatrix}\\
                 \end{bmatrix} \\
                  \\
                \begin{bmatrix}
                    f'(3, 3) \\
                \end{bmatrix}
                \times 
                \begin{bmatrix}
                    \begin{bmatrix}
                        c'(1,3) \\
                    \end{bmatrix} \\
                    \begin{bmatrix}
                        a'(1,3) \\
                    \end{bmatrix} \\
                \end{bmatrix} \\
            \end{bmatrix}
        \]
        \[
            = \begin{bmatrix}
                 f'(5,2) \\
            \end{bmatrix}
            \times 
            \begin{bmatrix}
                \begin{bmatrix}
                    1 & 2 \\
                \end{bmatrix}
                \times 
                \begin{bmatrix}
                    3 & 1 \\
                    1 & 0 \\
                \end{bmatrix} \\
                 \\
               \begin{bmatrix}
                   0 & -1 \\
               \end{bmatrix}
               \times 
               \begin{bmatrix}
                    0 & 1 \\
                    3 & 1 \\
               \end{bmatrix} \\
           \end{bmatrix}
        \]
        \[
            = \begin{bmatrix}
                f'(5,2) \\
           \end{bmatrix}
           \times 
           \begin{bmatrix}
               5  & 1 \\
               -3  & -1 \\
          \end{bmatrix}
        \]
        Suppose that \(\begin{bmatrix}
             f'(5,2) \\
        \end{bmatrix} = \begin{bmatrix}
            r  &  s \\
        \end{bmatrix}\), for some \(r,s \in \mathbb{R}\). Then
        \[
            \begin{bmatrix}
                 g'(1,3) \\
            \end{bmatrix}
            =
            \begin{bmatrix}
                 f'(5,2) \\
            \end{bmatrix}
            \times 
            \begin{bmatrix}
                5 &  1 \\
                -3 &  -1 \\
            \end{bmatrix}
        \]
        \[
            \implies \begin{bmatrix}
                3 &  4 \\
            \end{bmatrix}
            = 
            \begin{bmatrix}
                r &  s \\
            \end{bmatrix}
            \times 
            \begin{bmatrix}
                5 &  1 \\
                -3 &  -1 \\
            \end{bmatrix}
        \]
        \[
            \implies \begin{bmatrix}
                3 &  4 \\
            \end{bmatrix}
            = 
            \begin{bmatrix}
                5r-3s &  r-s \\
            \end{bmatrix}
        \]
        \[
            \implies 5r - 3s = 3 \text{ and } r - s = 4
        \]
        Solving the system of equations gives
        \[
            r = -\frac{9}{2} \text{ and } s = -\frac{17}{2}
        \]
        Therefore \(\nabla f(5,2) = \left( -\dfrac{9}{2}, -\dfrac{17}{2} \right)\).
    \end{question}
\end{document}