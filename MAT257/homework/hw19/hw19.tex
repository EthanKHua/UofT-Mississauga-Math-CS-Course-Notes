\documentclass{../../../tex-setup/eh-homework}

\begin{document}
    \begin{question}{43}
        \textbf{Transverse intersction.} Let $M,N$ be two smooth surfaces in $\R^3$. We say that $M$ and $N$ \textbf{intersect transversally} if $T_pM\neq T_pN$ for all $p\in M\cap N$.
    
        \begin{enumerate}[label=(\alph*)]
            \item Prove that if $M,N$ intersect transversally, then $M\cap N$ is a smooth curve in $\R^3$.
            \item Show by example that the conclusion of (a) fails without the assumption of transverse intersection.
        \end{enumerate}
        \tcblower
        \ 

        (a):

        Suppose that \(M\) and \(N\) intersect transversally. We will show that \(M \cap N\) is a smooth \(1\)-manifold in \(\mathbb{R}^3\). Let \(p \in M \cap N\). Then there is some relatively open neighborhood \(U\) of \(M\) and \(V\) of \(N\) that is the zero set of some smooth functions \(f,g: \mathbb{R}^3 \to \mathbb{R}\), that is, \(U = Z(f)\) and \(V = Z(g)\). Let \(\Phi : \mathbb{R}^3 \to \mathbb{R}^2\) be defined by \(\Phi = (f,g)\). We claim that \((U \cap V, \Phi)\) is the desired chart containing \(p\). We first start by verifying that \(U \cap V\) is relatively open to \(M \cap N\). This is quick, as we know that \(U\) and \(V\) are relatively open to \(M\) and \(N\) respectively, for each point in \(U \cap V\), we can choose two open balls with radii \(r_1,r_2\) that stays within \(M\) and \(N\) respectively. We then take the lesser of the radii as our radius.

        Moving on, we see that \(\Phi\) is smooth, and \(Z(\Phi) = U \cap V\). It remains to show that \(J \Phi (q)\) has rank 2 for all \(q \in U \cap V\). The Jacobian of \(\Phi\) is a \(2 \times 3\) matrix given by
        \[
            J \Phi (q) = \left( \begin{array}{@{}c@{}}
                \nabla f(q) \\
                \hline
                \nabla g(q) \\
            \end{array} \right).
        \]
        It necessarily has rank at most 2 and at least 1 (because \(\nabla f, \nabla g\) have at least rank 1). Suppose for contradiction that \(\rank J \Phi (q) = 1\). Then \(\nabla f(q) = c \nabla g(q)\) for some non-zero constant \(c\). We know that the tangent space \(T_q M\) is given by the set of tagged vectors orthogonal to \(\nabla f(q)\). Likewise, \(T_q N\) consists of tagged vectors orthogonal to \(\nabla g(q)\). But notice that for \(v \in T_q M\), we have \(\nabla f(q) \cdot v = 0\) but also \(c \nabla f(q) \cdot v = \nabla g(q) \cdot v = 0\). If we additionally apply the same argument to \(u \in T_q N\), we can see that \(T_q M = T_q N\) which is a contradiction. Thus \(J \Phi (q)\) must have rank 2. From here, it follows that \(\Phi ^{-1}(\{ 0 \}) = U \cap V\) is a smooth manifold with dimension 1, so we can conclude that \(M \cap N\) is a smooth curve.

        \medskip

        (b):

        Let \(M\) be the \(xy\)-plane, and let \(N\) be the graph of \(f(x,y) = x^2 + y^2\). Then \(M \cap N\) is simply the origin, which is not a smooth curve.
    \end{question}
    \newpage
    \begin{question}{44}
        Suppose that $M$ is a smooth manifold, and let $\mathcal{A}$ be an open cover of $M$ by pairwise consistently oriented charts. Let $\mathcal{A}^+$ be the collection of all charts on $M$ which are positively oriented with $\mathcal{A}$; likewise, let $\mathcal{A}^-$ be the collection of all charts on $M$ which are negatively oriented with $\mathcal{A}$.

        Now suppose that $\mathcal{B}$ is some other open cover of $M$ by charts, such that any two (overlapping) charts in $\mathcal{B}$ are consistently oriented. Prove that if $M$ is connected, then either $\mathcal{B}$ is completely contained in $\mathcal{A}^+$, or else it is completely contained in $\mathcal{A}^-$.

        \tcblower

        First, we will prove a form of transitivity for manifold charts.

        \textbf{Lemma.} From a manifold \(M\), take two charts \((U, \varphi)\), \((V, \psi)\) that are consistently oriented with each other. Then \((U ,\varphi)\) is positively oriented with \(\mathcal{A}\) if \((V, \psi)\) is positively oriented with \(\mathcal{A}\).

        Suppose that \((U, \varphi)\) is positively oriented with \(\mathcal{A}\). Consider the chart \((U \cap V, \varphi | _{U \cap V})\) that is consistently oriented with some chart \((W, \gamma)\) in \(\mathcal{A}\). We will show that \((V, \psi)\) is consistently oriented with \((W, \gamma)\). Notice that
        \[
            \psi \circ \gamma^{-1} = (\psi \circ \varphi^{-1}) \circ (\varphi \circ \gamma^{-1})
        \]
        so
        \[
            J (\psi \circ \gamma^{-1}) = J (\psi \circ \varphi^{-1}) \cdot J (\varphi \circ \gamma^{-1}) \text{ and } \det J(\psi \circ \gamma^{-1}) > 0
        \]
        as needed.

        Next, we prove the main result. Let \(\mathcal{B}\) be an open cover of \(M\) by charts. Pick a chart \((U, \varphi)\) in \(\mathcal{B}\), and say for convenience that it is contained in \(\mathcal{A} ^+\). If the chart picked was in \(\mathcal{A} ^-\) the argument is analogous. We claim that all other charts are also contained in \(\mathcal{A} ^+\). Let \((V, \psi) \in \mathcal{B}\). Take any two points \(p \in U\) and \(q \in V\). Since \(M\) is connected, we can find a continuous function \(\gamma : [0,1] \to M\) such that \(\gamma (0) = p\) and \(\gamma (1) = q\). Consider \(C = \gamma [0,1]\). Since it is compact, we can find a finite number of charts \(\{ (U_n, \varphi_n) \} _{n \leq N}\) that cover \(C\), such that the open cover includes \((U, \varphi)\) and \((V, \psi)\). As well, we can always find a sequence of \(m\) charts \((U_{n_k})_{1 \leq k \leq m}\) such that
        \[
            U \cap U_{n_1} \neq \varnothing , U_{n_1} \cap U_{n_2} \neq \varnothing , \ldots , U_{n_{m}} \cap V \neq \varnothing.
        \]
        Suppose if not. Then consider \(\mathcal{C} \), the collection of charts in the finite subcover that can be reached from \(U\) using a chain of charts. Let \(\overline{\mathcal{C}}\) be the complement; the colleciton of charts that cannot be reached from \(V\). Then the charts of \(\mathcal{C}\) and \(\mathcal{D}\) are pairwise disjoint from one another, and \(V \in \overline{\mathcal{C}}\). We have that \(\mathcal{C} \cup \overline{\mathcal{C}}\) cover \(\gamma [0,1]\), but \(\bigcup_{C_i \in \mathcal{C}} C_i\) is disjoint from \(\bigcup_{D_i \in \overline{\mathcal{C}}} D_i\), which contradicts the fact that \(\gamma [0,1]\) is connected, so such a sequence of charts does indeed exist. Since \((U, \varphi)\) and \((U_{n_1}, \varphi_{n_1})\) are consistently oriented with each other and \((U, \varphi)\) is positively oriented with \(\mathcal{A}\), by the lemma, \((U_{n_1}, \varphi_{n_1})\) is positively oriented with \(\mathcal{A}\). We can apply this argument inductively to conclude that \(V\) is also positively oriented with \(\mathcal{A}\) and thus \(V \in \mathcal{A} ^+\).

        Therefore, if a single chart in \(\mathcal{B}\) is contained in \(\mathcal{A}^+\), then in fact every chart in \(\mathcal{B}\) is contained in \(\mathcal{A}^+\). The same applies for the case where there is a chart that is contained in \(\mathcal{A}^-\).
    \end{question}
\end{document}