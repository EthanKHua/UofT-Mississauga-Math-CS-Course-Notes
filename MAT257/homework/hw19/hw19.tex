\documentclass{../../../tex-setup/eh-homework}

\begin{document}
    \begin{question}{43}
        \textbf{Transverse intersction.} Let $M,N$ be two smooth surfaces in $\R^3$. We say that $M$ and $N$ \textbf{intersect transversally} if $T_pM\neq T_pN$ for all $p\in M\cap N$.
    
        \begin{enumerate}[label=(\alph*)]
            \item Prove that if $M,N$ intersect transversally, then $M\cap N$ is a smooth curve in $\R^3$.
            \item Show by example that the conclusion of (a) fails without the assumption of transverse intersection.
        \end{enumerate}
        \tcblower
        \ 

        (a):

        Suppose that \(M\) and \(N\) intersect transversally. We will show that \(M \cap N\) is a smooth \(1\)-manifold in \(\mathbb{R}^3\). Let \(p \in M \cap N\). Then there is some relatively open neighborhood \(U\) of \(M\) and \(V\) of \(N\) that is the zero set of some smooth functions \(f,g: \mathbb{R}^3 \to \mathbb{R}\), that is, \(U = Z(f)\) and \(V = Z(g)\). Let \(\Phi : \mathbb{R}^3 \to \mathbb{R}^2\) be defined by \(\Phi = (f,g)\). We claim that \((U \cap V, \Phi)\) is the desired chart containing \(p\). We first start by verifying that \(U \cap V\) is relatively open to \(M \cap N\). This is quick, as we know that \(U\) and \(V\) are relatively open to \(M\) and \(N\) respectively, for each point in \(U \cap V\), we can choose two open balls with radii \(r_1,r_2\) that stays within \(M\) and \(N\) respectively. We then take the lesser of the radii as our radius.

        Moving on, we see that \(\Phi\) is smooth, and \(Z(\Phi) = U \cap V\). It remains to show that \(J \Phi (q)\) has rank 2 for all \(q \in U \cap V\). The Jacobian of \(\Phi\) is a \(2 \times 3\) matrix given by
        \[
            J \Phi (q) = \left( \begin{array}{@{}c@{}}
                \nabla f(q) \\
                \hline
                \nabla g(q) \\
            \end{array} \right).
        \]
        It necessarily has rank at most 2 and at least 1 (because \(\nabla f, \nabla g\) have at least rank 1). Suppose for contradiction that \(\rank J \Phi (q) = 1\). Then \(\nabla f(q) = c \nabla g(q)\) for some non-zero constant \(c\). We know that the tangent space \(T_q M\) is given by the set of tagged vectors orthogonal to \(\nabla f(q)\). Likewise, \(T_q N\) consists of tagged vectors orthogonal to \(\nabla g(q)\). But notice that for \(v \in T_q M\), we have \(\nabla f(q) \cdot v = 0\) but also \(c \nabla f(q) \cdot v = \nabla g(q) \cdot v = 0\). If we additionally apply the same argument to \(u \in T_q N\), we can see that \(T_q M = T_q N\) which is a contradiction. Thus \(J \Phi (q)\) must have rank 2. From here, it follows that \(\Phi ^{-1}(\{ 0 \}) = U \cap V\) is a smooth manifold with dimension 1, so we can conclude that \(M \cap N\) is a smooth curve.

        \medskip

        (b):

        Let \(M\) be the \(xy\)-plane, and let \(N\) be the graph of \(f(x,y) = x^2 + y^2\). Then \(M \cap N\) is simply the origin, which is not a smooth curve.
    \end{question}
\end{document}