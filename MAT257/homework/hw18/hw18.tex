\documentclass{../../../tex-setup/eh-homework}

\begin{document}
    \begin{question}{42}
        Let $M\subseteq \R^N$ be a smooth $n$-manifold (with or without boundary!).
        \begin{enumerate}[label=(\alph*)]
            \item Show that if $n<N$, then $M$ is a \textit{Lebesgue null set}.

            \item Show that if $n=N$ and $M$ is closed and its boundary is nonempty, then $\partial M$ coincides with the usual topological boundary (as defined on Handout \#2).

            \item Show that if $M$ is compact and its boundary is nonempty, then $M$ is Jordan measurable.
        \end{enumerate}
        \tcblower
        \ 

        (a):

        We begin by proving a lemma:

        \textbf{Lemma:} A countable union of sets with Jordan measure 0 is a Lebesgue null set.

        Let \(E = \bigcup_{i \geq 0} E_i\), where \(\mu (E_i) = 0\).
    \end{question}
\end{document}