\documentclass{../../../tex-setup/eh-homework}

\begin{document}
    \begin{question}{42}
        Let $M\subseteq \R^N$ be a smooth $n$-manifold (with or without boundary!).
        \begin{enumerate}[label=(\alph*)]
            \item Show that if $n<N$, then $M$ is a \textit{Lebesgue null set}.

            \item Show that if $n=N$ and $M$ is closed and its boundary is nonempty, then $\partial M$ coincides with the usual topological boundary (as defined on Handout \#2).

            \item Show that if $M$ is compact and its boundary is nonempty, then $M$ is Jordan measurable.
        \end{enumerate}
        \tcblower
        \ 

        (a):

        We begin by proving a number of lemmas:

        \textbf{Lemma 1:} An open cover of any subset \(M \subseteq \mathbb{R}^n\) has a countable subcover.

        We know that \(\mathbb{R}^n\) is separable, so \(M\) is also separable. Let \(C\) be a countable dense subset of \(M\). Let \(\mathcal{U}\) be an open cover for \(M\). We construct the countable subcover \(\hat{U}\) as follows. For each \(q \in C\) and \(k \in \mathbb{Q}\), consider the open ball \(B(q, k)\). If there exists a \(U_{qk} \in \mathcal{U}\) such that \(B(q , k) \in U_{qk}\), include it in \(\hat{U}\). Notice that \(\hat{U}\) is at most countable. We claim that it is also an open cover.

        Let \(x \in M\). Then it is contained in some open set \(U \in \mathcal{U}\). As well, we can find an open ball such that \(B(x, \delta) \in U\). Since \(C\) is dense, we can find \(q \in C\) such that \(q \in B(x, \frac{\delta}{4})\). Let \(k \in \mathbb{Q}\) such that \(\frac{\delta}{4} < k < \frac{\delta}{2}\). Then \(x \in B(q, k) \subseteq B(x, \delta)\), because for all \(y \in B(q,k)\),
        \[
            \|x - y\| \leq \|x - q\| + \|q - y\| < \frac{\delta}{4} + \frac{\delta}{2} < \delta
        \]
        It follows that \(B(q,k) \in U\), so it is guaranteed that some \(U_{qk}\) from our construction exists. Thus \(x \in U_{qk} \in U\) so \(U\) is indeed an open cover and we are done.

        \medskip

        \textbf{Lemma 2:} A countable union of sets with Jordan measure 0 is a Lebesgue null set.

        Let \(E = \bigcup_{i \geq 1} E_i\), where \(\mu (E_i) = 0\). Let \(\varepsilon > 0\). For each \(E_i\), we can find a finite union of boxes \(B_i\) such that \(B_i \supseteq E_i\) and \(\vol (B_i) < \frac{\varepsilon}{2^i}\). We see that \(\bigcup_{i\geq1}B_i\) is a countable union of boxes, \(E \subseteq \bigcup_{i\geq1}B_i\), and
        \[
            \sum_{i = 1}^{\infty} \vol (B_i) < \sum_{i=1}^{\infty} \frac{\varepsilon}{2^i} = \frac{\varepsilon}{2(1 - \frac{1}{2})} = \varepsilon
        \]
        as desired.

        \medskip

        \textbf{Lemma 3:} If \(K\) is a Jordan measurable set with a compact exhaustion \(K_n\), then \(\mu (K) = \lim_{n \to \infty} \mu (K_n)\).

        Let \(\varepsilon > 0\). Since \(K\) is Jordan measurable, we can find a closed polybox \(I \subseteq K\) such that
        \[
            \mu (I) > \mu (K) - \varepsilon
        \]
        Notice that \(I\) is covered by \(\{ K_n \} _{n \in \mathbb{N}}\) and is compact, so there exists \(N \in \mathbb{N}\) such that \(I \subseteq K_n\) for all \(n > N\). Thus
        \[
            \mu (K_n) > \mu (I) > \mu (K) - \varepsilon \implies |\mu (K) - \mu (K_n)| < \varepsilon
        \]
        as needed.

        \medskip

        Now, we prove the problem at hand. Let \(M \subseteq \mathbb{R}^N\) be a smooth \(n\)-manifold with \(n < N\). Let \(\{ (U_i, \varphi)\}_{i \in I}\) be an atlas for \(M\). By Lemma 1, we can assume without loss of generality that the atlas is countable. We can also assume that each \(U_i\) is bounded, for if not, we can take a countable union of open balls that cover the unbounded \(U_i\), and restrict the embedding to each ball.

        For each chart \((U_i, \varphi_i)\), assume that the domain \(\hat{U}\) is \(\mathbb{R}^n\) or \(\overline{\mathbb{H}^n}\), depending on if the chart includes the manifold boundary. Regardless, we can take the compact exhaustion \(K_n = B(0, n)\) and see that \(\varphi _i (K_n)\) is a compact exhaustion for \(U_i\). Additionally, since \(\varphi _i\) maps from lower dimension to higher dimension, we know from a previous Big List question that \(\mu (\varphi _i (K_n))=0\), so by Lemma 3, \(\mu (U_i) = 0\). This is true for each chart \(U_i\), so we apply Lemma 2 to conclude that \(\cup _{i \geq 1} U_i = M\) is a Lebesgue null set.

        \medskip

        (b):

        Let \(p\) be a point in the topological boundary of \(M\). We will show that \(p \in \partial M\). Suppose for contradiction that \(p\) is in \(M^\circ\), meaning it is contained in a chart \((U, \varphi)\) that is diffeomorphic to an open set \(\hat{U}\) in \(\mathbb{R}^N\). Note that \(\varphi\) is a diffeomorphism with domain \(\hat{U}\) and codomain \(U\). Then it must be true that \(\varphi (\hat{U}) = U\) is an open subset of \(\mathbb{R}^N\). But this is a contradiction, as that would imply that the boundary point \(p\) is in the interior of \(M\). Therefore \(p \in \partial M\).

        Next, let \(p \in \partial M\) and again suppose for contradiction that \(p\) is not in the topological boundary of \(M\). Then it must be true that \(p\) is in the topological interior of \(M\). Recall that \(p \in \partial M\) implies that it is contained in a chart \((U, \varphi)\) that is diffeomorphic to \(\overline{\mathbb{H}^n}\) and \(p \in \mathrm{bd} (\mathbb{H}^n)\). Since \(p\) is in the topological interior of \(M\), we can find an open ball \(B(p, r) \subseteq M\) which is also open in \(\mathbb{R}^N\). Then \(\varphi ^{-1}(B(p,r))\) should also be open in \(\mathbb{R}^N\). But this implies that for small enough \(\delta\), \(\varphi (p) - (0, ..., \delta) \in \overline{\mathbb{H}^N}\), which cannot happen.

        \smallskip

        Therefore we can conclude that \(\partial M\) coincides with the topological boundary of \(M\).

        \medskip

        (c):

        First, we prove that a compact Lebesgue null set \(E\) has Jordan measure 0. It suffices to show that the upper measure \(\mu ^* (E) = 0\).

        Let \(\varepsilon > 0\). By definition, we can find a countable union of boxes \(B = \bigcup_{i=1}^{\infty} B_i\) such that \(E \subseteq B\) and \(\vol(B) < \varepsilon\). But since \(E\) is compact, it can be covered by finitely many boxes \(B_{n_i}\), \(0 < i \leq N\). Thus
        \[
            \vol\left( \bigcup_{i=1}^{N} B_{n_i} \right) = \sum_{i=1}^{N} \vol(B_{n_i}) \leq \vol(B) < \varepsilon.
        \]
        Since \(\varepsilon\) was chosen arbitrarily, we can conclude that \(\mu ^* (E) = 0\), and \(E\) has Jordan measure 0.

        Now, suppose that \(M\) is compact and its boundary is nonempty. If \(\dim M < N\), \(M\) is a Lebesgue null set and has Jordan measure 0, and therefore measurable. Otherwise, if \(\dim M = N\), since the boundary of \(M\) is non-empty, the topological boundary of \(M\) is actually a smooth manifold of dimension \((N-1)\), and therefore a Lebesgue null set. As well, the topological boundary of \(M\) is compact, so it is Jordan measure 0, which implies that \(M\) is Jordan measurable.
    \end{question}
    \newpage
    \begin{question}{34}
        Let $\Phi:\R^n\rightarrow\R^m$ be a $C^1$ mapping.
        \begin{enumerate}[label=(\alph*)]
            \item Suppose that $n>m = 1$. Show that $\Phi$ cannot be injective.
            
            \item Suppose that $n<m$. Show that if $K\subseteq\R^n$ is a compact set, then $\Phi(K)$ is a Jordan measurable set, and has Jordan measure zero.
        \end{enumerate}
        \tcblower
        \ 
        
        (a):

        Suppose for contradiction that \(n > m = 1\) and \(\Phi\) is a \(C^1\) injective function. Since \(\Phi\) cannot be a constant function, by the results of Big List \#26, there is a \(p \in \mathbb{R}^n\) so that \(\nabla \Phi(p) \neq 0\). In particular, we will say that \(\dfrac{\partial \Phi}{\partial x_j} \neq 0\). Define \(\alpha : \mathbb{R}^n \to \mathbb{R}\) by \(\alpha (x) = \Phi (x) - \Phi (p)\). Injectivity is translation-invariant, so \(\alpha\) is injective. Notice that \(\alpha (p) = 0\). We can apply the implicit function theorem to obtain an open set \(W \subseteq \mathbb{R}^{n-1}\) that contains \(p' = (p_1, ..., p_{j-1}, p_{j+1}, ..., p_n)\) and a \(C^1\) function \(\Psi : W \to \mathbb{R}\) such that for all \(x = (x_1, ..., x_{n-1}) \in W\),
        \[
            \alpha (x_1, ..., x_{j-1}, \Psi (x), x_{j}, ..., x_{n-1}) = 0
        \]
        Then, since \(W\) is open and contains \(p'\), we can find another distinct point \(q \in W\). We have
        \[
            \alpha (p_1, ..., p_{j-1}, \Psi (p'), p_{j+1}, ..., p_n) = 0 = \alpha (q_1, ..., q_{j-1}, \Psi (q), q_{j}, ..., q_{n-1})
        \]
        which contradicts the fact that \(\alpha\) is injective.

        \medskip

        (b):

        Since \(K\) is compact, and thus bounded, we can enclose it in a closed box \(B = [-L, L]^n\), for some positive \(L\). It suffices to show that \(\Phi (B)\) has measure 0, as we can apply the monotonicity of measure to conclude that \(\Phi (K)\) has measure 0.

        First, let \(\hat{\Phi}: \mathbb{R}^m \to \mathbb{R}^m\) be defined by \(\hat{\Phi}(x) = \Phi (\pi _{\mathbb{R}^n}(x))\). That is, \(\hat{\Phi}\) first projects vectors in \(\mathbb{R}^m\) onto \(\mathbb{R}^n\) and then composes it with \(\Phi\). Let \(\hat{B} = B \times \{ 0 \}^{m-n}\). Then note that \(\hat{\Phi}(\hat{B}) = \Phi (B)\). Trivially, \(\hat{B}\) has measure 0. We now show that \(\Phi (B)\) also has measure 0.

        Let \(\varepsilon > 0\). Since \(\hat{\Phi}\) is \(C^1\), its derivative is continuous. By the extreme value theorem, each component derivative attains a maximum on \(\hat{B}\). Let \(\alpha\) a positive number greater than all the maximums. Since \(\hat{B}\) has measure 0, we can find a finite number of cubes \(B_1, ..., B_k\) with side length \(d\) such that
        \[
            \hat{B} \subseteq \bigcup_{i=1}^{k} B_i \text{ and } \sum_{i=1}^{k} \vol(B_i) < \frac{\varepsilon}{m^m \alpha^m}
        \]
        Consider some cube \(B_i = \prod_{j=1}^{m} [a_{ij}, a_{ij} + d]\). For each component \(\hat{\Phi}_j: \mathbb{R}^m \to \mathbb{R}\), it must be true that \(\hat{\Phi}_j\) attains a maximum \(\hat{\Phi}_j(M_{ij})\) and minimum \(\hat{\Phi}_{ij}(m_{ij})\), for some \(M_{ij}, m_{ij} \in B_i\). Define \(g: [0,1] \to \mathbb{R}\) by
        \[
            g(t) = \hat{\Phi}_j(t M_{ij} + (1 - t)m_{ij})
        \]
        By the Mean Value Theorem, there exists some \(c \in (0,1)\) such that
        \[
            g(1) - g(0) = g'(c) \implies \hat{\Phi}_j (M) - \hat{\Phi}_j (m) = \hat{\Phi}_j '(c M_{ij} + (1 - c)m_{ij})(M_{ij} - m_{ij})
        \]
        Taking the 2-norm of both sides, it follows that
        \[
            \hat{\Phi}_j (M_{ij}) - \hat{\Phi}_j (m_{ij}) \leq \|\hat{\Phi}_j '(c M_{ij} + (1 - c)m_{ij})\| \|M_{ij} - m_{ij}\| \leq (\sqrt{m} \alpha)(\sqrt{m}d) = m \alpha d
        \]
        Note that we used the fact that each component of \(\hat{\Phi}_j '(c M_{ij} + (1 - c)m_{ij})\) is bounded by \(\alpha\) and \(\|M_{ij} - m_{ij}\|\) can be no bigger than the diagonal of \(B_i\).

        Finally, we see that the box \(C_i = \prod_{j=1}^m [\hat{\Phi}_j(m_{ij}), \hat{\Phi}_j(M_{ij})]\) covers \(\hat{\Phi} (B_i)\), and its volume is given by
        \[
            \prod_{j=1}^m (\hat{\Phi}_j (M_{ij}) - \hat{\Phi}_j (m_{ij})) \leq \prod_{j=1}^m m \alpha d = m^m \alpha^m \vol (B_i)
        \]
        From here, we know that the union \(\bigcup_{i=1}^{k} C_i \supseteq \hat{\Phi}(\hat{B})\) but
        \[
            \sum_{i=1}^{k} C_i < m^m \alpha^m \sum_{i=1}^{k} \vol (B_i) < \varepsilon.
        \]
        Therefore \(\Phi(B)\) has measure 0, so \(\Phi (K)\) has measure 0 and we are done.
    \end{question}
\end{document}