\documentclass{../../../tex-setup/eh-homework}

\begin{document}
    \begin{question}{39}
        Let $M\subseteq\R^N$ be a set and let $\varphi:U\rightarrow M$ be a smooth parametrization of $M$, defined on some open set $U\subseteq \R^n$. We say that $\varphi$ is a \textbf{regular parametrization} if $\varphi$ is a homeomorphism, and if $\varphi'(p)$ has rank $n$ at every point $p\in U$. Note that we require $\varphi^{-1}$ to be continuous, but not necessarily smooth!

    \begin{enumerate}[label=(\alph*)]
        \item Prove that if $\varphi$ is a regular parametrization of $M$, then $\varphi^{-1}$ is also smooth.
        
        \item Prove that if $\varphi:U\rightarrow M$ and $\psi:V\rightarrow M$ are two regular parametrizations of $M$, then $\vol_\varphi(M)=\vol_\psi(M)$.

        \item Use regular parametrizations to find the surface area of a sphere of radius $r$ in $\R^4$.
    \end{enumerate}
    \tcblower
    \ 

    (a):

    Suppose \(\varphi\) is a smooth regular parametrization of \(M\).

    \medskip

    (b):

    Suppose that \(\varphi : U \to M\) and \(\psi : V \to M\) are regular parametrizations of \(M\). We can rewrite
    \[
        \psi = \varphi \circ \left(\varphi^{-1} \circ \psi\right).
    \]
    Let \(\Phi : V \to U\) be defined by \(\Phi = \varphi^{-1} \circ \psi\). We claim that \(\Phi\) is a diffeomorphim. \(\psi\) is smooth, and by part (a), we know that \(\varphi^{-1}\) is smooth, so \(\Phi\) is also smooth. Furthermore, \(\Phi^{-1} = \psi^{-1} \circ \varphi\), which is smooth for similar reasons. Thus \(\Phi\) is indeed a diffeomorphism. This being the case, we know that parametrizations are diffeomorphism-invariant (done in class), so we can conclude that
    \[
        \vol _{\varphi}(M) = \vol _{\psi}(M).
    \]
    \end{question}
\end{document}