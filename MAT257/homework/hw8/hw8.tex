\documentclass{article}
\usepackage[margin=1.0in]{geometry}
\usepackage{amssymb,amsmath,amsthm,amsfonts}
\usepackage{enumitem}
\usepackage{xcolor}
\usepackage{mathtools}

% My boxes
\usepackage[breakable]{tcolorbox}

% \RequirePackage{background}
% \backgroundsetup{
%     scale=1,
%     color=black,
%     opacity=1,
%     angle=0,
%     contents={
%         \includegraphics[width=\paperwidth,height=\paperheight]{\nightmodebackground}
%     }
% }

\definecolor{pastelblue}{RGB}{96, 145, 245}
\definecolor{pastelgreen}{RGB}{106, 235, 135}
\definecolor{darkgray}{RGB}{60, 60, 60}
\definecolor{lightgray}{RGB}{180, 180, 180}
\definecolor{offwhite}{RGB}{225, 225, 245}


\pagecolor{darkgray}
\color{offwhite}

\newcommand{\Z}{\mathbf{Z}}
\newcommand{\N}{\mathbf{N}}
\newcommand{\R}{\mathbf{R}}
\newcommand{\Q}{\mathbf{Q}}
\newcommand{\C}{\mathbf{C}}

\newcommand{\id}{\mathrm{id}}
\newcommand{\op}{\mathrm{op}}
\newcommand{\diam}{\mathrm{diam}}
\newcommand{\GL}{\mathrm{GL}}
\newcommand{\Tr}{\mathrm{Tr}}
\newcommand{\im}{\mathrm{im}}
\newcommand{\rank}{\mathrm{rank}}

\newcommand{\cl}[1]{\overline{#1}}

\swapnumbers % places numbers before thm names

\theoremstyle{plain} % The "plain" style italicizes all body text.
	\newtheorem{thm}{Theorem}
		\numberwithin{thm}{section} % Theorem numbers are determined by section.
	\newtheorem{lemma}[thm]{Lemma}
	\newtheorem{prop}[thm]{Proposition}
	\newtheorem{cor}[thm]{Corollary}

\theoremstyle{definition}
    \newtheorem{defn}[thm]{Definition}
	\newtheorem{example}[thm]{Example}
	\newtheorem{exercise}[thm]{Exercise} %Exercise

\begin{document}
    \newtcolorbox{question}[2][]{fonttitle=\large, fontupper=\large, fontlower=\large, title=Question {#2}., oversize, arc=3mm, outer arc=2mm, opacityback=0.9, coltitle=offwhite, colframe=pastelblue, colback=darkgray, colupper=lightgray, collower=lightgray, leftrule=1mm, rightrule=1mm, toprule=1.5mm, titlerule=1mm, bottomrule=1mm, valign=center, add to natural height=5mm, lower separated=false, before lower=\begin{proof}, after lower= \\ \end{proof}, #1, breakable=true}

    \begin{question}{22}
        For a normed vector space $(X,\|\cdot\|)$, let $X^*=B(X,\R)$ denote the set of bounded linear mappings from $X$ to $\R$. Here $X^*$ is equipped with the operator norm $\|\cdot\|_{\mathrm{op}}$, and the normed vector space $(X^*,\|\cdot\|_{\mathrm{op}})$ is called the \textbf{topological dual} of $X$.

        Let $c_0$ denote the set of sequences converging to zero. Prove that $c_0^*\equiv \ell^1$.
        \tcblower
        Equip \(\ell ^1\) with the 1-norm and \(c_0\) with the sup-norm. Define the function \(f: \ell ^1 \to c_0^*\) by \(f(\vec{x})(\vec{a}) = \sum_{i=1}^{\infty} x_i \cdot a_i\). This sum is convergent because for sufficiently large \(N > 0\), \(|a_n| < 1\) for \(n > N\), so
        \[
            \sum_{i=1}^{\infty} x_i \cdot a_i = \sum_{i=1}^{N} x_i \cdot a_i + \sum_{i=N+1}^{\infty} x_i \cdot a_i < \sum_{i=1}^{N} x_i \cdot a_i + \sum_{i=N+1}^{\infty} x_i < \infty
        \]
        We claim that this is a bijective isometry.

        First, it can be quickly verified that this function is linear. Letting \(a_n \in c_0\), for \(x,y \in \ell ^1\), \(k > 0\),
        \[
            f(kx + y)(a_n) = \sum_{i=1}^{\infty} a_i \cdot (kx_i + y_i) = k \sum_{i=1}^{\infty} a_i \cdot x_i + \sum_{i=1}^{\infty} a_i \cdot y_i = kf(x)(a_n) + f(y)(a_n) \text{.} 
        \]
        To show that this function is an isometry, let \(x,y \in \ell ^1\).
        
        Let \(A = \{|f(x)(a) - f(y)(a)| : a \in c_0, \|a\| _{\infty} \leq 1\}\). It will be shown that \(\|x-y\| _1 = \sup A\).
        
        Let \(a \in c_0\) such that \(\|a\| _{\infty} \leq 1\). Then
        \[
            |f(x)(a)-f(y)(a)| = \left\vert \sum_{i=1}^{\infty} (x_i \cdot a_i - y_i \cdot a_i) \right\vert = \left\vert \sum_{i=1}^{\infty} a_i (x_i - y_i) \right\vert \leq \sum_{i=1}^{\infty} |a_i| |x_i - y_i| \leq \sum_{i=1}^{\infty} |x_i - y_i| 
        \]
        \[
            = \|x-y\| _1 \text{.} 
        \]
        This means that \(\|x-y\| _1\) is an upper bound for \(A\). To prove that this is the least upper bound, let \(\varepsilon > 0\). There is an \(N > 0\) such that \(\sum_{N+1}^{\infty} |x_i - y_i| < \varepsilon\). Define the sequence \((a_n)\) in \(c_0\) as the sequence of 1's until and including \(n=N\) and 0 afterwards. Then
        \[
            |f(x)(a) - f(y)(a)| = \left\vert \sum_{i=1}^{\infty} a_i (x_i - y_i) \right\vert = \left\vert \sum_{i=1}^{\infty} \left(1 - 1 + a_i \right)  (x_i - y_i) \right\vert
        \]
        \[
            \geq \left\vert \sum_{i=1}^{\infty} (x_i - y_i) \right\vert - \left\vert \sum_{i=1}^{\infty} \left( 1 - a_i \right)  (x_i - y_i) \right\vert
        \]
        \[
            =  \left\vert \sum_{i=1}^{\infty} (x_i - y_i) \right\vert - \sum_{i=N+1}^{\infty} |x_i - y_i| > \|x-y\| _1 - \varepsilon
        \]
        Thus we have that 
        \[
            \|f(x) - f(y)\| _{\op} = \sup {A} = \|x-y\| _1
        \]
        so \(f\) is an isometry.

        To prove that \(f\) is bijective, it suffices to show that \(f\) is surjective, since we already have injectivity from the isometry.

        Let \(T \in c_0^*\). Construct a sequence \((x_n)\) in \(\ell ^1\) defined by \(x_n = T(e_n)\), where \((e_n)\) is the sequence with the only non-zero term being \(e_n = 1\).

        First, we need to verify that \((x_n) \in \ell ^1\). To do this, we bound the absolute sum by a finite number. We have that
        \[
            \sum_{i=1}^{\infty} |x_i| = \sum_{i=1}^{\infty} |T(e_i)|
        \]
        For all non-zero terms, we can rewrite it as \(|T(e_i)| \cdot \dfrac{T(e_i)}{T(e_i)} = T\left(\dfrac{|T(e_i)|}{T(e_i)}e_i\right)\). Define \((b_n)\) to be a sequence defined by
        \[
            b_n = \begin{dcases}
                0, &\text{ if } T(e_i) = 0;\\
                \frac{|T(e_i)|}{T(e_i)}e_i, &\text{ otherwise} .
            \end{dcases}
        \]
        Then
        \[
            \sum_{i=1}^{\infty} |T(e_i)| = T(b_n)
        \]
        Notice that \(|b_n| \leq 1\) for all \(n \in \mathbb{N}\). Thus \(\|b_n\| _{\infty} = 1\), so
        \[
            \sum_{i=1}^{\infty} |x_i| = T(b_n) \leq \|T\| _{\op}
        \]
        By the monotone bounded convergence, \(\sum_{i=1}^{\infty} |x_i|\) converges. Therefore \((x_n) \in \ell ^1\).

        Now, we want to show that \(f(x_n) = T\). For \((a_n) \in c_0\), we have that
        \[
            f(x_n)(a_n) = \sum_{i=1}^{\infty} x_i \cdot a_i = \sum_{i=1}^{\infty} T(e_i) \cdot a_i = T\left(\sum_{i=1}^{\infty} a_i (e_i)\right) = T((a_n))
        \]
        so \(f\) is surjective.

        Thus \(f\) is a bijective isometry, so \(c_0^* \equiv \ell ^1\).
    \end{question}
    \pagebreak
    \begin{question}{23}
        Let $S^2$ denote the unit sphere in $\R^3$. Let $N=(0,0,1)$ denote the ``north pole''. In this problem, you will show that $S^2\setminus \{N\}$ is homeomorphic to $\R^2$. To do this, we define a function $\Phi:S^2\setminus \{N\}\rightarrow \R^2$ known as the \textbf{stereographic projection}: given a point $P$ in $S^2\setminus \{N\}$, draw a line between $P$ and $N$, and let $\Phi(P)$ denote the point where this line intersects the $xy$-plane in $\R^3$.
        \begin{enumerate}[label=(\alph*)]
            \item Given $P=(x,y,z)$, find an explicit formula for $\Phi(P)$ in terms of $x,y,z$.
            \item Deduce that $\Phi$ is continuous.
            \item Prove that $\Phi$ is a bijection; in fact, given $p=(s,t)\in \R^2$, find an explicit formula for $\Phi^{-1}(p)$.
            \item Deduce that $\Phi$ is a homeomorphism.
        \end{enumerate}
        \tcblower
        We will work off the assumption that both metric spaces are equipped with the max-norm. We can do this because all norms on \(\mathbb{R}^n\) are equivalent.

        (a):

        Let \(P=(x,y,z)\). First, we find the equation of the line that passes \(P\) and \(N\). Consider the equation of the line \(L(t) = (tx, ty, (z-1)t + 1)\). Notice that \(L(0) = N\) and \(L(1) = P\), so \(L\) satisfies what we were looking for. Now we find the point where \(L\) intersects with the \(xy\)-plane. This happens exactly when \((z-1)t+1=0\). Solving for \(t\) gives \(t=\dfrac{1}{1-z}\). This value is always defined as \(z\neq 1\). As a result, it turns out that
        \[
            L\left(\frac{1}{1-z}\right) = \left(\frac{x}{1-z}, \frac{y}{1-z}, 0\right) \text{.}
        \]
        Thus
        \[
            \Phi (P) = \frac{1}{1-z}\left(x,y\right) \text{.}
        \]

        (b):
        
        To show that \(\Phi\) is continuous, fix \((a,b,c) \in \mathbb{R}^3\). Let \(\varepsilon > 0\). Then let \(\delta = \min \{\frac{1-c}{2}, \frac{(1-c)\varepsilon}{4}, \frac{(1-c)^2 \varepsilon}{4\max \{|a|,|b|\}}\}\). For \((x,y,z) \in \mathbb{R}^3\) such that \(\|(x,y,z) - (a,b,c)\| _{\max} < \delta\), notice that
        \[
            z-c < |z-c| \leq \|(x,y,z) - (a,b,c)\| _{\max} < \frac{1-c}{2}
        \]
        Manipulating this, we obtain
        \[
            z < \frac{1+c}{2} \implies 1 - z > \frac{1-c}{2} \implies \frac{1}{1-z} < \frac{2}{1-c} \text{.} 
        \]
        So
        \[
            \|\Phi (x,y,z) - \Phi (a,b,c)\| _{\max} = \left\lVert \left(\frac{x}{1-z}, \frac{y}{1-z}\right) - \left(\frac{a}{1-c}, \frac{b}{1-c}\right) \right\rVert _{\max}
        \]
        \[
            =\frac{\|(x(1-c) - a(1-z), y(1-c) - b(1-z))\| _{\max}}{(1-z)(1-c)}
        \]
        \[
            < \frac{2\|(x(1-c) - a(1-z), y(1-c) - b(1-z))\| _{\max}}{(1-c)^2}
        \]
        Without loss of generality, suppose that \(|x(1-c) - a(1-z)| \geq |y(1-c) - b(1-z)|\). Then
        \[
            \frac{2\|(x(1-c) - a(1-z), y(1-c) - b(1-z))\| _{\max}}{(1-c)^2} = \frac{2|x(1-c) - a(1-z)|}{(1-c)^2}
        \]
        \[
            = \frac{2|(x-a)(1-c) + a(z-c)|}{(1-c)^2} \leq \frac{2(|x-a||1-c| + |a||z-c|)}{(1-c)^2}
        \]
        \[
            < \frac{2 \left(\frac{(1-c)^2}{4}\varepsilon + \frac{|a|(1-c)^2}{4\max \{|a|,|b|\}}\varepsilon\right)}{(1-c)^2} < \frac{\varepsilon}{2} + \frac{\varepsilon}{2} = \varepsilon
        \]
        Therefore \(\|\Phi (x,y,z) - \Phi (a,b,c)\| _{\max} <\varepsilon\), so \(\Phi\) is continuous.

        (c):

        Let \(p = (s,t) \in \mathbb{R}^2\). Our goal is to find \((x,y,z) \in S^2 \setminus \{N\}\) such that \(\Phi (x,y,z) = \left( \frac{x}{1-z}, \frac{y}{1-z} \right) = (s,t)\). Immediately, we obtain the following system of equations:
        \[
            \frac{x}{1-z} = s \text{,} 
        \]
        \[
            \frac{y}{1-z} = t \text{,} 
        \]
        \[
            x^2 + y^2 + z^2 = 1
        \]
        We also have the restriction \(z\neq 1\) because \((x,y,z) \neq N\). Isolating for \(x\) and \(y\) yields
        \[
            x=s(1-z)
        \]
        \[
            y=t(1-z)
        \]
        Then we substitute this into the third equation and get
        \[
            s^2(1-z)^2 + t^2(1-z)^2 + z^2 = 1 \implies (s^2 + t^2 + 1)z^2 - 2(s^2 + t^2)z + s^2 + t^2 - 1 = 0
        \]
        We can replace the term \(t^2 + s^2\) with \(\|p\| _2 ^2\), and the equation becomes
        \[
            (\|p\| _2 ^2 + 1)z^2 - 2\|p\| _2 ^2 z + \|p\| _2 ^2 - 1 = 0
        \]
        Using the quadratic formula:
        \[
            z = \frac{2(\|p\| _2 ^2) \pm \sqrt{4(\|p\| _2 ^2)^2 - 4(\|p\| _2 ^2 +1)(\|p\| _2 ^2 - 1)}}{2(\|p\| _2 ^2 + 1)}
        \]
        \[
            \implies z = \frac{\|p\| _2 ^2 \pm \sqrt{\|p\| _2 ^4 - (\|p\| _2 ^4 - 1)}}{\|p\| _2 ^2 + 1}
        \]
        \[
            z = \frac{\|p\| _2 ^2 \pm 1}{\|p\| _2 ^2 + 1}
        \]
        Notice that we cannot use the positive solution, for then
        \[
            z = \frac{\|p\| _2 ^2 + 1}{\|p\| _2 ^2 + 1} = 1
        \]
        Thus it must be true that
        \[
            z = \frac{\|p\| _2 ^2 - 1}{\|p\| _2 ^2 + 1}
        \]
        \[
            x = s(1-z) = \frac{2s}{\|p\| _2 ^2 + 1}
        \]
        \[
            y = \frac{2t}{\|p\| _2 ^2 + 1}
        \]
        It can be verified that these values of \(x,y,z\) result in \(\Phi (x,y,z) = (s,t)\), which show that \(\Phi\) is surjective. Additionally, it was also proven that these values are the only ones that work, so \(\Phi\) is bijective. In fact, using this, we obtain that the formula for \(\Phi ^{-1}\) is
        \[
            \Phi ^{-1} (s,t) = \left(\frac{2s}{\|p\| _2 ^2 + 1}, \frac{2t}{\|p\| _2 ^2 + 1}, \frac{\|p\| _2 ^2 - 1}{\|p\| _2 ^2 + 1}\right)
        \]

        (d):

        We know that \(\Phi ^{-1}\) is bijective because its inverse exists. It remains to show that it is continuous. This can be done by verifying pointwise continuity.

        Let \((a,b) \in \mathbb{R}^2\). Considering the first component, fix \(\varepsilon > 0\), and let \(\delta = \frac{\varepsilon}{2}\). Let \((s,t) \in \mathbb{R}^2\) so that \(\|(s - a, t - b)\|_2 < \delta\). Consider the case where \(\frac{2s}{\|(s,t)\| _2 ^2 + 1} - \frac{2a}{\|(a,b)\| _2 ^2 + 1} > 0\). Then
        \[
            \left\vert \frac{2s}{\|(s,t)\| _2 ^2 + 1} - \frac{2a}{\|(a,b)\| _2 ^2 + 1} \right\vert = \frac{2s}{\|(s,t)\| _2 ^2 + 1} - \frac{2a}{\|(a,b)\| _2 ^2 + 1} < \frac{2s(\|(s,t)\| _2^2 + 1)}{\|(s,t)\| _2 ^2 + 1} - 2a
        \]
        \[
            = 2(s-a) < \varepsilon
        \]
        The case when \(\frac{2s}{\|(s,t)\| _2 ^2 + 1} - \frac{2a}{\|(a,b)\| _2 ^2 + 1} \leq 0\) is the exact same but the variables are swapped. Thus the first component is continuous. We can do the same thing for the second component so we can accept that it is continuous as well.

        Finally, notice that since norms are continuous, so the third component is a composition of continuous functions, making it continuous. Therefore \(\Phi ^{-1}\) is continuous.

        We can conclude that \(\Phi\) is indeed a homeomorphism and we are done.
    \end{question}
    \pagebreak
    \begin{question}{24}
        Let $X$ be a normed vector space. Prove that the following statements are equivalent.
        \begin{enumerate}[label=(\roman*)]
            \item $X$ is finite-dimensional.
            \item The unit ball $\cl{B}(\vec{0},1)$ is compact.
            \item $X$ is \textbf{locally compact}: each point $p\in X$ is contained in some open set $U$ such that $\cl{U}$ is compact.
        \end{enumerate}
        \tcblower
        It will be proven that (i) \(\implies\) (ii) \(\iff\) (iii) \(\implies\) (i).

        (i) \(\implies\)  (ii):

        Suppose that \(X\) has finite dimension \(n\). Then there is a continuous linear isomorphism \(\Phi\) between \(X\) and \(\mathbb{R}^n\). Since \(\cl{B}(0,1)\) is closed and bounded, \(\Phi (\cl{B}(0,1))\) is also closed and bounded in \(\mathbb{R}^n\), so the set is compact. Since homeomorphisms preserve compactness, we can conclude that the closed unit ball in \(X\) is compact.

        (ii) \(\implies\) (iii):

        Suppose that the unit ball \(\cl{B}(\vec{0}, 1)\) is compact. Let \(p \in X\). We claim that \(U = B(p,1)\). Consider \(\cl{U} = \cl{B}(p,1)\). There is an isometry \(\Phi\) from \(\cl{B}(0,1)\) to \(\cl{B}(p,1)\) defined by \(\Phi (x) = p + x\). Since the closed unit ball is compact, it follows that \(\cl{B}(p,1)\) is compact. Thus \(X\) is locally compact.

        (iii) \(\implies\) (ii):

        Suppose that \(X\) is locally compact. Then \(\vec{0}\) is contained in an open set \(U\) such that \(\cl{U}\) is compact. Since \(U\) is open, \(B(0, \varepsilon) \subseteq U\) for some \(\varepsilon\)-ball centered around 0. It follows that \(\cl{B}(0,\varepsilon)\) is compact, as it is a closed subset of \(\cl{U}\). Since there is a homeomorphism from this closed ball to \(\cl{B}(0,1)\), the closed unit ball is also compact, as desired.

        (iii) \(\implies\) (i):

        Suppose that \(X\) is locally compact. We know previously that this implies that the closed unit ball is compact. Firstly, a quick lemma will be proven.

        \textbf{Lemma 1.} \(\forall x \in X\), \(r > 0\), \(B(x,r) = \{x\} + B(0,r)\).

        Let \(y \in B(x,r)\). Notice that \(y-x \in B(0,r)\). Thus \(y = x + (y-x)\) so \(y \in \{x\} + B(0,r)\).

        Then, let \(y \in \{x\} + B(0,r)\), so \(y = x + s\), for some \(s \in B(0,r)\). Since \(\|y-x\| = \|s\| < r\), it follows that \(y \in B(x,r)\) and we are done.
        
        Moving on, we construct a finite set of vectors in the following way:

        Since \(\cl{B}(0,1)\) is totally bounded, we can find a finite set of vectors \(\beta\) such that \(\bigcup_{x \in \beta} B\left(x,\frac{1}{2}\right)\).

        \textbf{Lemma 2.} \(B(0,1) \subseteq \operatorname{span}(\beta) + 2^{-n} B(0,1)\), \(\forall n \in \mathbb{N}\).

        To prove this, let \(n \in \mathbb{N}\) and \(y \in B(0,1)\). It follows that \(y \in B(x_1, \frac{1}{2})\) for some \(x_1 \in \beta\). Using Lemma 1,
        \[
            y \in \{x_1\} + B\left(0,\frac{1}{2}\right) = \{x_1\} + \frac{1}{2}B(0,1)
        \]
        We can repeatedly use this argument to obtain that
        \[
            y \in \left\{x_1 + \frac{1}{2}x_2 + \frac{1}{2^2}x_3 + \dots + \frac{1}{2^{n-1}}x_n\right\} + \frac{1}{2^n}B(0,1)
        \]
        Note that \(x_i\) are not necessarily distinct. This simplifies down to
        \[
            y \in \operatorname{span}(\beta) + \frac{1}{2^n}B(0,1) \implies B(0,1) \subseteq \operatorname{span}(\beta) + \frac{1}{2^n}B(0,1)
        \]
        which is what we wanted.

        Finally, we claim that \(\beta\) is the basis for \(X\).

        Let \(v \in X\). Since \(\dfrac{v}{2\|v\|} \in B(0,1)\), by Lemma 2,
        \[
            \frac{v}{2\|v\|} \in \operatorname{span}(\beta) + \frac{1}{2^n}B(0,1) \text{, for all } n \in \mathbb{N}
        \]
        Then we construct a sequence \(\{v_n\}_{n\geq 1}\) in \(\operatorname{span}(\beta)\) where \(v_n \in \operatorname{span}(\beta)\) such that \(y = v_n + p\), for some \(p \in \frac{1}{2^n}B(0,1)\). Notice that
        \[
            \|y - v_n\| = \|p\| < \frac{1}{2^n}
        \]
        which can become arbitrarily small. This implies that \(y\) is a limit point for \(\operatorname{span}(\beta)\). However, this set is closed because it is finite dimensional, so we also have that \(y \in \operatorname{span}(\beta)\). Therefore \(X \subseteq \operatorname{span}(\beta)\), and since \(X\) is spanned by a finite set, it is finite dimensional.

        These implications are sufficient for proving that (i), (ii), (iii) are equivalent.
    \end{question}
    
\end{document}