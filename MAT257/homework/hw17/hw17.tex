\documentclass{../../../tex-setup/eh-homework}

\begin{document}
    \begin{question}{40}
        Let $O_n(\R)$ be the set of all $n\times n$ real orthogonal matrices:
        \[ O_n(\R) = \{A\in M_n(\R) : A^tA=I_n\}. \]
        Show that $O_n$ is a smooth manifold, and find its dimension.
        \tcblower
        First, we note that \(O_n(\mathbb{R})\) is the zero set of the function \(f: M_n(\mathbb{R}) \to S^n\) defined by
        \[
            f(A) = A^t A - I_n
        \]
        where \(S^n\) is the set of symmetric \(n\times n\) matrices. Notice that \(f\) is smooth as it is constructed by smooth functions. Additionally, we show that \(Jf(X)(h) = X^t h + h^T X\). Indeed,
        \begin{align*}
            \lim_{h \to 0} \frac{f(X + h) - f(X) - X^t h - h^t X}{\|h\|} &= \lim_{h \to 0} \frac{(X + h)^t (X + h) - X^t X - X^t h - h^t X}{\|h\|} \\
            &= \lim_{h \to 0} \frac{h^t h}{\|h\|} \\
            &= 0
        \end{align*}
        Next, we want to show that \(\rank Jf(X) = \frac{1}{2}n(n+1)\) for all \(X \in O_n(\mathbb{R})\). It suffices to show that \(Jf(X)\) is surjective to \(S^n\).

        Let \(Y \in S^n\). Let \(h = \frac{1}{2}XY\). We see that
        \begin{align*}
            Jf(X)(h) &= X^t \left( \frac{1}{2}XY \right) + \left( \frac{1}{2}XY \right)^t X = \frac{1}{2}\left( X^t XY + Y^t X^t X \right) \\
            &= \frac{1}{2}(Y + Y^t) \tag{\(X\) is orthogonal} \\
            &= Y \tag{\(Y\) is symmetric}
        \end{align*}
        Thus \(R(Jf(X)) = S^n\) so \(\rank Jf(X) = \dim S^n = \frac{1}{2}n(n+1)\).

        We now prove that \(O_n(\mathbb{R})\) is a smooth manifold of dimension \(\frac{1}{2}n(n+1)\). Let \(p \in O_n(\mathbb{R})\). Then \(f(p) = 0\) and \(Jf(p)\) has the maximal rank of \(\frac{1}{2}n(n+1)\). We write
        \[
            Jf(p) = \left( \begin{array}{@{}c|c@{}}
                A & B
            \end{array} \right) 
        \]
        where \(A\) is a \(\frac{1}{2}n(n+1) \times \frac{1}{2}n(n-1)\) matrix and \(B\) is a \(\frac{1}{2}n(n+1) \times \frac{1}{2}n(n+1)\) matrix and assume without loss of generality that \(B\) is an invertible submatrix of \(Jf(p)\). We can do this because we can swap the components of \(f\), and therefore columns of \(Jf(p)\) without affecting the conclusion of the statement (because manifolds are invariant under diffeomorphims). Thus, we write \(p = (a,b)\) for \(a \in \mathbb{R}^{\frac{1}{2}n(n-1)}\), \(b \in \mathbb{R}^{\frac{1}{2}n(n+1)}\) and apply the Implicit Function Theorem and obtain an open set \(\hat{U} \subseteq \mathbb{R}^{\frac{1}{2}n(n-1)}\) containing \(a\) and a \(C^{\infty}\) function \(\Phi : \hat{U} \to \mathbb{R}^{\frac{1}{2}n(n+1)}\) so that
        \[
            f(x, \Phi(x)) = 0
        \]
        for all \(x \in \hat{U}\). We claim that \(\varphi: \hat{U} \to \Phi (\hat{U})\) defined by
        \[
            \varphi (x) = (x, \Phi (x))
        \]
        is our desired smooth regular embedding. It is fairly clear that \(\varphi\) is smooth. Additionally, \(J \varphi (x) = \left( \begin{array}{@{}cc@{}}
            I_{\frac{1}{2}n(n-1)} \\
            \hline
            J\Phi
        \end{array} \right) \) is a \(n^2 \times \frac{1}{2}n(n-1)\) matrix and has at least \(\frac{1}{2}n(n-1)\) linearly independent rows, so \(\varphi\) is regular. Finally, if we let \(\varphi (x) = \varphi (y)\), we have that \((x,\Phi (x)) = (y, \Phi (y))\), from which we get \(x=y\), so \(\varphi\) is injective, and therefore bijective to its image. In addition, we can explicitly find \(\varphi^{-1} (y) = \pi _{\mathbb{R}^{\frac{1}{2}n(n-1)}}(y)\), which is continuous because it is a linear map. Therefore \(\varphi\) is a homeomorphism onto its image, and we are done.
    \end{question}
    \newpage
    \begin{question}{41}
        Let $0<a<b$. In the $xz$-plane, draw a circle of radius $a$ centered at the point $(b,0,0)$; rotate this circle about the $z$-axis. The resulting subset of $\R^3$ is called a \textbf{torus}, denoted by $\mathbf{T}=\mathbf{T}_{a,b}$.
        \begin{enumerate}[label=(\alph*)]
            \item Find a smooth function $f:U\rightarrow \R$, defined on some open set $U\subseteq \R^2$, so that $\mathbf{T}$ is equal to the zero set of $f$.
            
            \item Show that $\mathbf{T}$ is a smooth manifold.
            
            \item Find the surface area of $\mathbf{T}$, in terms of $a$ and $b$.
        \end{enumerate}
        \tcblower
        \ 

        (a):

        Notice that in cylindrical coordinates, the torus can be defined by
        \[
            T = \{ (r, \theta, z) : (r - b)^2 + z^2 = a^2\}.
        \]
        If we map the polar part of the set back to cartesian coordinates, we see that \(T\) is actually the zero set of the function
        \[
            f(x,y,z) = (\sqrt{x^2 + y^2} - b)^2 + z^2 = a^2
        \]
    \end{question}
\end{document}