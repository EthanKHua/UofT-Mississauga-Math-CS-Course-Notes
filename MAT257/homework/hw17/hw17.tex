\documentclass{../../../tex-setup/eh-homework}

\begin{document}
    \begin{question}{40}
        Let $O_n(\R)$ be the set of all $n\times n$ real orthogonal matrices:
        \[ O_n(\R) = \{A\in M_n(\R) : A^tA=I_n\}. \]
        Show that $O_n$ is a smooth manifold, and find its dimension.
        \tcblower
        First, we will prove the Regular Level-Set Theorem:
        \smallbreak
        Let \(X,Y\) be normed vector spaces with dimensions \(n,m\) and ordered bases \(\alpha ,\beta\) respectively, where \(n > m\). Let \(F: X \to Y\) be a smooth function. Define \(M = F^{-1} (0_Y)\). If \(F'(p)\) is surjective for all \(p \in M\), then \(M\) is a smooth manifold of dimension \(n - m\). That is, if \(\phi _\alpha : X \to \mathbb{R}^n\) is the coordinate isomorphism corresponding to \(\alpha\), then \(M\) is a smooth \(k\)-manifold if \(\phi _\alpha (M)\) is a smooth \(k\)-manifold in the usual sense.
        \smallbreak

        This problem reduces to trying to prove that \(N \coloneqq \phi _\alpha (M)\) is a smooth manifold of dimension \(n-m\). Notice that \(N = \phi _\alpha (F^{-1} (0_Y)) = \phi _\alpha (F^{-1} (\phi _\beta (0_{\mathbb{R}^m})))\). Since \(\phi _\alpha , \phi _\beta\) are isomorphisms, we have that \(N = \phi _\alpha \circ F^{-1} \circ \phi _\beta (0_{\mathbb{R}^m})\). Let \(\hat{F} : \mathbb{R}^n \to \mathbb{R}^m\) be a function defined by \(\hat{F} = \phi _\alpha \circ F^{-1} \circ \phi _\beta\), and notice that \(N\) is the zero set of \(\hat{F}\). For any \(p \in N\), notice that \(\hat{F}'(p)\) is surjective, because
        \[
            \hat{F}'(p) = (\phi _\beta \circ F \circ \phi^{-1} _\alpha)'(p) = \phi _\beta \circ F'(\phi ^{-1}_{\alpha}) \circ \phi^{-1}_{\alpha} 
        \]
        is just \(F'(\phi^{-1}_{\alpha})\)--a surjective map--composed with linear isomorphisms. This implies that \(R(J\hat{F}(p)) = \mathbb{R}^m\) and \(\rank J\hat{F}(p) = m\). We can write
        \[
            J\hat{F}(p) = \left( \begin{array}{@{}c|c@{}}
                A & B
            \end{array} \right) 
        \]
        where \(A\) is a \(m\times n-m\) matrix and \(B\) is a \(m \times m\) matrix, and assume without loss of generality that \(B\) is invertible, for if not, \(J\hat{F}(p)\) is still guaranteed to contain an invertible \(m\times m\) submatrix, and we can perform column swaps to move the matrix to the right, which does not change the conclusion of our statement.

        Recall that since \(N\) is the zero set of \(\hat{F}\), \(\hat{F}(p) = 0\). Thus, we write \(p = (a,b)\) for \(a \in \mathbb{R}^{n-m}\), \(b \in \mathbb{R}^{m}\) and apply the Implicit Function Theorem and obtain an open set \(\hat{U} \subseteq \mathbb{R}^{n-m}\) containing \(a\) and a \(C^{\infty}\) function \(\Phi : \hat{U} \to \mathbb{R}^{m}\) so that
        \[
            \hat{F}(x, \Phi(x)) = 0
        \]
        for all \(x \in \hat{U}\). We claim that \(\varphi: \hat{U} \to \Phi (\hat{U})\) defined by
        \[
            \varphi (x) = (x, \Phi (x))
        \]
        is our desired smooth regular embedding. It is fairly clear that \(\varphi\) is smooth. Additionally, \(J \varphi (x) = \left( \begin{array}{@{}cc@{}}
            I_{n-m} \\
            \hline
            J\Phi
        \end{array} \right) \) is a \((n-m) \times n\) matrix and has at least \(n-m\) linearly independent rows, so \(\varphi\) is regular. Finally, if we let \(\varphi (x) = \varphi (y)\), we have that \((x,\Phi (x)) = (y, \Phi (y))\), from which we get \(x=y\), so \(\varphi\) is injective, and therefore bijective to its image. In addition, we can explicitly find \(\varphi^{-1} (y) = \pi _{\mathbb{R}^{n-m}}(y)\), which is continuous because it is a linear map. Therefore \(\varphi\) is a homeomorphism onto its image, so \(N\) is a smooth \((n-m)\)-manifold. It follows that \(\phi ^{-1}_{\alpha} (N) = M\) is also a smooth manifold of dimension \(n-m\) so we are done.

        \medskip

        Now, we may proceed to prove that \(O_n(\mathbb{R})\) is a smooth manifold of dimension \(\frac{1}{2}n(n-1)\).

        We note that \(O_n(\mathbb{R})\) is the zero set of the function \(f: M_n(\mathbb{R}) \to S^n\) defined by
        \[
            f(A) = A^t A - I_n
        \]
        where \(S^n\) is the set of symmetric \(n\times n\) matrices. Notice that \(f\) is smooth as it is constructed by smooth functions. Additionally, we show that \(f'(X)(h) = X^t h + h^T X\). Indeed,
        \begin{align*}
            \lim_{h \to 0} \frac{f(X + h) - f(X) - X^t h - h^t X}{\|h\|} &= \lim_{h \to 0} \frac{(X + h)^t (X + h) - X^t X - X^t h - h^t X}{\|h\|} \\
            &= \lim_{h \to 0} \frac{h^t h}{\|h\|} \\
            &= 0
        \end{align*}
        Next, we want to show that \(f'(X)\) is surjective on \(S^n\) for all \(X \in O_n(\mathbb{R})\).

        Let \(Y \in S^n\). Let \(h = \frac{1}{2}XY\). We see that
        \begin{align*}
            f'(X)(h) &= X^t \left( \frac{1}{2}XY \right) + \left( \frac{1}{2}XY \right)^t X = \frac{1}{2}\left( X^t XY + Y^t X^t X \right) \\
            &= \frac{1}{2}(Y + Y^t) \tag{\(X\) is orthogonal} \\
            &= Y \tag{\(Y\) is symmetric}
        \end{align*}
        Now, we have the hypotheses needed to apply the Regular Level-Set Theorem, and conclude that \(O_n(\mathbb{R})\) is a smooth manifold of dimension \(n^2 - \frac{1}{2}n(n+1) = \frac{1}{2}n(n-1)\).
    \end{question}
    \newpage
    \begin{question}{41}
        Let $0<a<b$. In the $xz$-plane, draw a circle of radius $a$ centered at the point $(b,0,0)$; rotate this circle about the $z$-axis. The resulting subset of $\R^3$ is called a \textbf{torus}, denoted by $\mathbf{T}=\mathbf{T}_{a,b}$.
        \begin{enumerate}[label=(\alph*)]
            \item Find a smooth function $f:U\rightarrow \R$, defined on some open set $U\subseteq \R^3$, so that $\mathbf{T}$ is equal to the zero set of $f$.
            
            \item Show that $\mathbf{T}$ is a smooth manifold.
            
            \item Find the surface area of $\mathbf{T}$, in terms of $a$ and $b$.
        \end{enumerate}
        \tcblower
        \ 

        (a):

        Notice that in cylindrical coordinates, the torus can be defined by
        \[
            T = \{ (r, \theta, z) : (r - b)^2 + z^2 = a^2\}.
        \]
        If we map the polar part of the set back to cartesian coordinates, we see that \(T\) is actually the zero set of the function
        \[
            f(x,y,z) = (\sqrt{x^2 + y^2} - b)^2 + z^2 - a^2
        \]
        This function is smooth everywhere except for when \(x = 0 = y\), so we let \(U = \mathbb{R}^3 \setminus \{ (x,y,z) : x = y = 0 \}\).

        \medskip

        (b):

        Notice that for all \(p = (x,y,z) \in T\), \(f'(p)\) is rank 1, because \(f'(p)\) is a \(1 \times 3\) matrix, so its rank is at most 1, but it cannot be rank 0 because
        \[
            \frac{\partial f}{\partial x} (x,y,z) = 2\frac{\sqrt{x^2 + y^2} - b}{\sqrt{x^2 + y^2}} = 2 - \frac{2b}{\sqrt{x^2 + y^2}}
        \]
        and
        \[
            \frac{\partial f}{\partial z} (x,y,z) = 2z
        \]
        so being rank 0 implies that
        \[
            f'(p) = \left( 2 - \frac{2b}{\sqrt{x^2 + y^2}}, 2 - \frac{2b}{\sqrt{x^2 + y^2}}, 2z \right) = (0,0,0) \implies \sqrt{x^2 + y^2} = b \text{ and } z = 0
        \]
        but if this is the case, \(f(p) = -a^2 \neq 0\), so \(p \notin T\), which is a contradiction.

        Thus \(Jf(p)\) is always rank 1, so according to the Regular Level-Set Theorem, \(T\) is a smooth manifold of dimension \(3 - 1 = 2\).

        \medskip

        (c):

        We split \(T\) into \(4\) quadrants of equal volume, but we will only do the computation for one quadrant. Let \(Q\) be the set
        \[
            Q = \{ (x,y,z) \in T : x > 0 , z > 0 \}.
        \]
        We parametrize \(Q\) with the function \(\varphi : (b -a, b + a) \times \left( -\frac{\pi}{2},\frac{\pi}{2} \right)\) defined by
        \[
            \varphi (r,\theta) = (r\cos \theta , r\sin \theta , \sqrt{a^2 - (r - b)^2}).
        \]
        To see that \(\varphi\) is a smooth regular embedding, we see that every component is smooth on \(Q\) (the third component is smooth when \(r \neq b \pm a\), but \(r \in (b - a, b + a)\)). As well, the first two components are the polar coordinate transform, so \(J \varphi(p)\) is rank \(2\) for all \(p \in (b - a, b + a)\times \left( -\frac{\pi}{2}, \frac{\pi}{2} \right)\). Finally, notice that \(\varphi^{-1}(x,y,z) = \left( \sqrt{x^2 + y^2} , \arctan \left( \frac{y}{x} \right)  \right) \) is continuous, so \(\varphi\) is a homeomorphism, thus confirming that \(\varphi\) is a smooth regular embedding. Now, we find an expression for \(V(J \varphi(r, \theta))\):
        \begin{align*}
            V(J \varphi(r, \theta)) &= \sqrt{\det ((J \varphi (r, \theta))^t J \varphi (r, \theta))} \\
            &= \sqrt{\det \left(\begin{pmatrix}
                \cos \theta  & \sin \theta & - \frac{r - b}{\sqrt{a^2 - (r-b)^2}} \\
                -r \sin \theta & r\cos \theta & 0  \\
            \end{pmatrix}
            \begin{pmatrix}
                \cos \theta & -r \sin \theta  \\
                \sin \theta & r\cos \theta  \\
                - \frac{r-b}{\sqrt{a^2 - (r - b)^2}}& 0  \\
            \end{pmatrix}
            \right)} \\
            &= \sqrt{\det \begin{pmatrix}
                \frac{a^2}{a^2 - (r-b)^2} & 0  \\
                0 & r^2  \\
            \end{pmatrix}}
            = \sqrt{\frac{a^2 r^2}{a^2 - (r - b)^2}} \\
            &= \frac{ar}{\sqrt{a^2 - (r - b)^2}}
        \end{align*}
        Now we can calculate the volume of \(Q\):
        \begin{align*}
            \vol (Q) &= \int _{-\frac{\pi}{2}}^{\frac{\pi}{2}} \int _{b-a}^{b+a} V(J \varphi (r, \theta))\ dr\ d \theta = \int _{-\frac{\pi}{2}}^{\frac{\pi}{2}} \int _{b-a}^{b+a} \frac{ar}{\sqrt{a^2 - (r-b)^2}}\ dr\ d \theta \\
            &= \int _{-\frac{\pi}{2}}^{\frac{\pi}{2}} \int _{-a}^a \frac{a(r + b)}{\sqrt{a^2 - r^2}}\ dr\ d\theta \tag{substitution \(r \mapsto r - b\)} \\
            &= \int _{-\frac{\pi}{2}}^{\frac{\pi}{2}} \int _{-a}^a \frac{ar}{\sqrt{a^2 - r^2}}\ dr\ d \theta + \int _{-\frac{\pi}{2}}^{\frac{\pi}{2}} \int _{-a}^a \frac{ab}{\sqrt{a^2 - r^2}}\ dr\ d \theta \\
            &= 0 + ab\int _{-\frac{\pi}{2}}^{\frac{\pi}{2}} \arcsin \left( \frac{r}{a}\right) \Big| _{-a}^{a}\ d \theta \tag{first integrand is odd} \\
            &= ab \int _{-\frac{\pi}{2}}^{\frac{\pi}{2}} \pi\ d \theta \\
            &= \pi ^2 ab
        \end{align*}
        Finally, we multiply the surface area of the quadrant by 4 to obtain that the total surface area is \(4\pi ^2 ab\).
    \end{question}
\end{document}