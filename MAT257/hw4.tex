\documentclass{article}
\usepackage[margin=1.0in]{geometry}
\usepackage{amssymb,amsmath,amsthm,amsfonts}
\usepackage{enumitem}
\usepackage{xcolor}
\usepackage{mathtools}

\newcommand{\Z}{\mathbf{Z}}
\newcommand{\N}{\mathbf{N}}
\newcommand{\R}{\mathbf{R}}
\newcommand{\Q}{\mathbf{Q}}
\newcommand{\C}{\mathbf{C}}

\newcommand{\id}{\mathrm{id}}
\newcommand{\op}{\mathrm{op}}
\newcommand{\diam}{\mathrm{diam}}
\newcommand{\Tr}{\mathrm{Tr}}
\newcommand{\im}{\mathrm{im}}
\newcommand{\rank}{\mathrm{rank}}

\newcommand{\cl}[1]{\overline{#1}}

\swapnumbers % places numbers before thm names

\theoremstyle{plain} % The "plain" style italicizes all body text.
	\newtheorem{thm}{Theorem}
		\numberwithin{thm}{section} % Theorem numbers are determined by section.
	\newtheorem{lemma}[thm]{Lemma}
	\newtheorem{prop}[thm]{Proposition}
	\newtheorem{cor}[thm]{Corollary}

\theoremstyle{definition}
    \newtheorem{defn}[thm]{Definition}
	\newtheorem{example}[thm]{Example}
	\newtheorem{exercise}[thm]{Exercise} %Exercise

\begin{document}
    \setcounter{section}{3}
    \section{Homework 4}
    \noindent\textbf{Question 11.} Let $(X,d)$ be a metric space. A function $f:X\rightarrow X$ is called a \textbf{contraction mapping} if there exists a constant $M\in (0,1)$ such that
    \[ d(f(x),f(y))\leq M d(x,y) \quad \text{for all $x,y\in X$.} \]
    \begin{enumerate}[label=(\alph*)]
        \item Suppose that $(X,d)$ is a complete metric space, and that $f:X\rightarrow X$ is a contraction mapping. Prove that $f$ has a unique fixed point; \textit{i.e.} there exists a unique point $x_0\in X$ such that $f(x_0)=x_0$.
        
        \begin{proof}
            Let \((X, d)\) be a complete metric space and \(f\) be a contraction mapping. In this proof, for \(n \in \mathbb{N}\), we denote \(f^n\) to be a composition of \(f\). First, we will prove a lemma:

            \textbf{Lemma.} \(\forall x \in X, k \in \mathbb{N}, d(x, f^n(x)) < C\), where \(C\) is a real constant.

            To prove this, we will use an induction argument on \(k\).

            Let \(k=1\).

            Now suppose that the claim holds true for \(k=l\), where \(l \in \mathbb{N}\). Then by the triangle inequality,
            \[
                d(x, f^{l+1} (x)) \leq d(x, f^l (x)) + d(f^l(x), f^{l+1}(x))
            \]
        \end{proof}

        \item Give an example of a normed vector space $(X,\|\cdot\|)$ and a contraction mapping $f:X\rightarrow X$ such that $f$ does \textbf{not} have a fixed point.
    \end{enumerate}
    \noindent\textbf{Question 12.} \textit{The Intermediate Value Theorem.}

    \begin{enumerate}[label=(\alph*)]
        \item A subset $I\subseteq \R$ is called an \textbf{interval} if $a,b\in I$ implies $[a,b]\subseteq I$.

        Let $I\subseteq \R$. Prove that $I$ is connected (with respect to the usual metric on $\R$) if and only if $I$ is an interval.
        
        \begin{proof}
            We prove the equivalent statement \(I\) is disconnected if and only if \(I\) is not an interval.

            Suppose \(I\) is disconnected. Then there exist disjoint, open, non-empty sets \(A,B \subseteq I\) such that \(A \cup B = I\). Take any \(a \in A\) and \(b \in B\). We can assume without loss of generality that \(a<b\) and consider the interval \([a,b]\).

            Conversely, suppose that \(I\) is not an interval. Then for \(p<q \in I\), there is a \(c \in [p,q]\) such that \(c \notin I\). Define subsets \(A\) and \(B\) in \(I\) as \(A = \{x\in I \colon x < c\}\) and \(B = \{x\in I \colon x > c\}\). \(A\) and \(B\) are non-empty because \(p \in A\) and \(q \in B\). The sets are also disjoint by construction. To show that \(A\) is open, take any \(a \in A\).
            
            For this value of \(a\), take \(\varepsilon = c - a > 0\). For all \(x \in B_I(a, \varepsilon)\), note that \(x \in I\). if \(x < a\), immediately we have \(x < a < c \implies x \in A\). If \(x >a\), since \(x\) is within the open ball surrounding \(a\), \(x-a = |x-a| < c-a \implies x<c \implies x\in A\). Thus every element of \(A\) is an interior point, so \(A\) is open.

        \end{proof}

        \item Let $(X,d_X)$ and $(Y,d_Y)$ be two metric spaces, and let $f:X\rightarrow Y$ be a continuous function. Prove that if $C$ is a connected subset of $X$, then $f(C)$ is a connected subset of $Y$.
        
        \item Recall the \textbf{Intermediate Value Theorem} from single-variable calculus. \textit{``Let $I\subseteq \R$ be an open interval and $f:I\rightarrow \R$ be a continuous function. Suppose that $a,b\in f(I)$ are two numbers such that $a<b$, and suppose that $a<y_0<b$. Then there exists $x_0\in I$ such that $f(x_0)=y_0$.''}

        Prove that this theorem immediately follows from (a) and (b). Thus, (b) is a generalization of the Intermediate Value Theorem.
    \end{enumerate}
\end{document}