\documentclass{article}
\usepackage[margin=1.0in]{geometry}
\usepackage{amssymb,amsmath,amsthm,amsfonts,mathtools}
\usepackage{enumitem}
\usepackage{xcolor}

\newlength\tindent
\setlength{\tindent}{\parindent}
\setlength{\parindent}{0pt}
\renewcommand{\indent}{\hspace*{\tindent}}

\newcommand{\Z}{\mathbf{Z}}
\newcommand{\N}{\mathbf{N}}
\newcommand{\R}{\mathbf{R}}
\newcommand{\Q}{\mathbf{Q}}
\newcommand{\C}{\mathbf{C}}

\newcommand{\id}{\mathrm{id}}
\newcommand{\op}{\mathrm{op}}
\newcommand{\diam}{\mathrm{diam}}
\newcommand{\GL}{\mathrm{GL}}
\newcommand{\Tr}{\mathrm{Tr}}

\newcommand{\cl}[1]{\overline{#1}}

\swapnumbers % places numbers before thm names

\theoremstyle{plain} % The "plain" style italicizes all body text.
	\newtheorem{thm}{Theorem}
		\numberwithin{thm}{section} % Theorem numbers are determined by section.
	\newtheorem{lemma}[thm]{Lemma}
	\newtheorem{prop}[thm]{Proposition}
	\newtheorem{cor}[thm]{Corollary}

\theoremstyle{definition}
    \newtheorem{defn}[thm]{Definition}
	\newtheorem{example}[thm]{Example}
	\newtheorem{exercise}[thm]{Exercise} %Exercise

\begin{document}
    \section*{Exercise 18.15}
    Recall the following definitions:
    \begin{enumerate}[label=(\alph*)]
        \item $F$ is \textbf{exact} if $F=\nabla f$ for some smooth function $f:U\rightarrow \R$. The function $f$ is called a \textbf{scalar potential function} for $F$.
        \item $F$ is \textbf{conservative} if $\displaystyle \oint_CF \cdot d\vec{x}=0$ for every loop $C$ contained in $U$. (A \textbf{loop} is the image of a piecewise smooth map $\gamma:[a,b]\rightarrow \R^n$ such that $\gamma$ is a regular embedding on $[a,b)$, and $\gamma(a)=\gamma(b)$. Or, you could think of it as a smooth, oriented curve $C$ which is closed as a subset of $\R^n$, and such that $\partial C = \varnothing$.)
    \end{enumerate}
    In this exercise, we will show that a conservative function is also exact.
    \begin{proof}
        Let \(F: U \subseteq \mathbb{R}^n \to \mathbb{R}^n\) be conservative. For each connected component \(U_i\) of \(U\), we pick an arbitrary \(q_i \in U_i\). Define a function \(f: U \to \mathbb{R}\) as follows. For each \(p \in U\), it is contained in some \(U_i\). Let \(\gamma : [0,1] \to U_i\) parametrize a path from \(p\) to \(p_i\), that is, \(\gamma\) is smooth, regular, and satisfies \(\gamma (0) = p\), \(\gamma (1) = p_i\). For this \(\gamma\), we define
        \[
            f(p) = \oint _{\gamma [0,1]} F \cdot \vec{dx} = \int _0^1 F (\gamma (t))\cdot \gamma '(t)\ dt
        \]
    \end{proof}
\end{document}