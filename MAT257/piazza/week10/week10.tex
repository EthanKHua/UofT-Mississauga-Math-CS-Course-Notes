\documentclass{article}
\usepackage[margin=1.0in]{geometry}
\usepackage{amssymb,amsmath,amsthm,amsfonts,mathtools}
\usepackage{enumitem}
\usepackage{xcolor}

\newcommand{\Z}{\mathbf{Z}}
\newcommand{\N}{\mathbf{N}}
\newcommand{\R}{\mathbf{R}}
\newcommand{\Q}{\mathbf{Q}}
\newcommand{\C}{\mathbf{C}}

\newcommand{\id}{\mathrm{id}}
\newcommand{\op}{\mathrm{op}}
\newcommand{\diam}{\mathrm{diam}}
\newcommand{\Tr}{\mathrm{Tr}}

\newcommand{\cl}[1]{\overline{#1}}

\swapnumbers % places numbers before thm names

\theoremstyle{plain} % The "plain" style italicizes all body text.
	\newtheorem{thm}{Theorem}
		\numberwithin{thm}{section} % Theorem numbers are determined by section.
	\newtheorem{lemma}[thm]{Lemma}
	\newtheorem{prop}[thm]{Proposition}
	\newtheorem{cor}[thm]{Corollary}

\theoremstyle{definition}
    \newtheorem{defn}[thm]{Definition}
	\newtheorem{example}[thm]{Example}
	\newtheorem{exercise}[thm]{Exercise} %Exercise

\begin{document}
    \section*{Exercise 8.12 (c) (iii)}
    \textbf{Solvers:} Ethan

    \noindent\textbf{Writeup:} Ethan

    Let \(U \subseteq \mathbb{R}^n\) be an open set and let \(f: U \to \mathbb{R}\) be a scalar function. Then if all partial derivatives exist and are continuous on \(U\), then \(f\) is continuously differentiable and \(f'(p)\) is given by \(f'(p)(v) = D_v f(p)\).

    \textcolor{red}{Recall the statement in (ii): There exists a point \(q_k \in U\) such that
    \[
        \frac{f(p_k) - f(p_{k-1})}{h_k} = \frac{\partial f}{\partial x_k} (q_k)
    \]
    We make the additional distinction that \(q_k\) is of the form \(p_{k-1} + \gamma e_k\), where \(|\gamma| < |h_k|\).}

    \begin{proof}
        Define \(\|\cdot\|\) on \(U\) to be the 1-norm.

        Let \(L_p = f'(p)\). We will show that \(\dfrac{|f(p+h) - f(p) - L_p(h)|}{\|h\|} \to 0\).

        Let \(\varepsilon > 0\). Utilising part (i), we define a sequence of points \(p_0, ..., p_n \in X\) by
        \[
            p_0 = p \text{ and } p_i = p_{i-1} + h_i e_i \text{.} 
        \]
        We know that for \(\|h\|\) smaller than some positive \(\delta _1\), \(p_i \in U\). By the uniform continuity of \(\dfrac{\partial f}{\partial x_i}\) there is also a \(\delta _2\) such that for all \(a,b \in U\) such that \(\|a-b\| <\delta _2\), \(\left\vert\dfrac{\partial f}{\partial x_i} (a-b)\right\vert < \varepsilon\).

        Let \(\delta = \min \{\delta _1, \delta _2 \}\). Let \(h \in U\) so that \(\|h\| < \delta\). Notice that \(p+h = p_n\) and \(p = p_0\). We have that
        \[
            \frac{|f(p+h) - f(p) - L_p(h)}{\|h\|} = \frac{|f(p_n) - f(p_0) - L_p(h)|}{\|h\|}
        \]
        We can expand the numerator by adding and subtracting every term \(p_i\) and substituting
        \[
            L_p(h) = D_h f (p) = \sum_{i=1}^n h_i \dfrac{\partial f}{\partial x_i}(p) \text{,} 
        \]
        which yields
        \[
            \frac{\left\vert \sum\limits_{i=1}^n \left(f(p_i) - f(p_{i-1}) - h_i\dfrac{\partial f}{\partial x_i}(p) \right)\right\vert}{\|h\|} \leq \sum_{i=1}^n \frac{\left\vert f(p_i) - f(p_{i-1}) - h_i \dfrac{\partial f}{\partial x_i} (p) \right\vert }{\|h\|} \text{.} 
        \]
        Now, by part (ii), we can rewrite \(f(p_i) - f(p_{i-1})\) as \(h_i \dfrac{\partial f}{\partial x_i} (q_i)\), for some \(q_i \in U\), so the expression becomes
        \[
            \sum_{i=1}^n \frac{\left\vert h_i \dfrac{\partial f}{\partial x_i} (q_i) - h_i \dfrac{\partial f}{\partial x_i} (p) \right\vert}{\|h\|} = \frac{1}{\|h\|}\sum_{i=1}^n |h_i| \left\vert\dfrac{\partial f}{\partial x_i} (q_i - p)\right\vert
        \]
        Note that \(p_i\) can also be written as \(p + \sum_{j=1}^i h_j e_j\). Thus we can say that \(q_i = p_{i-1} + \gamma e_i = p + \gamma e_i + \sum_{j=1}^{i-1} h_j e_j\). We get that
        \[
            \frac{1}{\|h\|}\sum_{i=1}^n |h_i| \left\vert\dfrac{\partial f}{\partial x_i} (q_i - p)\right\vert = \frac{1}{\|h\|}\sum_{i=1}^n |h_i| \left\vert\dfrac{\partial f}{\partial x_i} \left(\gamma e_i + \sum_{j=1}^{i-1} h_j e_j\right)\right\vert
        \]
        We see that the norm of the argument inside the partial derivative is
        \[
            \left\lVert \gamma e_i + \sum_{j=1}^{i-1} h_j e_j \right\rVert \leq |\gamma| + \sum_{j=1}^{i-1} |h_j| < \sum_{j=1}^i |h_j| \leq \sum_{j=1}^n |h_j| = \|h\| < \delta _2 \text{,} 
        \]
        so by the continuity of \(\dfrac{\partial f}{\partial x_i}\),
        \[
            \frac{1}{\|h\|}\sum_{i=1}^n |h_i| \left\vert\dfrac{\partial f}{\partial x_i} \left(\gamma e_i + \sum_{j=1}^{i-1} h_j e_j\right)\right\vert < \frac{1}{\|h\|}\sum_{i=1}^n |h_i| \cdot \varepsilon = \frac{\|h\|}{\|h\|} \cdot \varepsilon = \varepsilon
        \]
        and the proof is complete.

    \end{proof}
\end{document}