\documentclass{article}
\usepackage[margin=1.0in]{geometry}
\usepackage{amssymb,amsmath,amsthm,amsfonts,mathtools}
\usepackage{enumitem}
\usepackage{xcolor}

\newlength\tindent
\setlength{\tindent}{\parindent}
\setlength{\parindent}{0pt}
\renewcommand{\indent}{\hspace*{\tindent}}

\newcommand{\Z}{\mathbf{Z}}
\newcommand{\N}{\mathbf{N}}
\newcommand{\R}{\mathbf{R}}
\newcommand{\Q}{\mathbf{Q}}
\newcommand{\C}{\mathbf{C}}

\newcommand{\id}{\mathrm{id}}
\newcommand{\op}{\mathrm{op}}
\newcommand{\diam}{\mathrm{diam}}
\newcommand{\GL}{\mathrm{GL}}
\newcommand{\Tr}{\mathrm{Tr}}

\newcommand{\cl}[1]{\overline{#1}}

\swapnumbers % places numbers before thm names

\theoremstyle{plain} % The "plain" style italicizes all body text.
	\newtheorem{thm}{Theorem}
		\numberwithin{thm}{section} % Theorem numbers are determined by section.
	\newtheorem{lemma}[thm]{Lemma}
	\newtheorem{prop}[thm]{Proposition}
	\newtheorem{cor}[thm]{Corollary}

\theoremstyle{definition}
    \newtheorem{defn}[thm]{Definition}
	\newtheorem{example}[thm]{Example}
	\newtheorem{exercise}[thm]{Exercise} %Exercise

\begin{document}
    \section*{Exercise 15.14}
    \textbf{Solvers:} Emerald, Ethan

    \textbf{Writeup:} Ethan

    \begin{center}
        \textit{Special mention to Alan and Sanchit for helping us out a bit as well!}
    \end{center}
    Let \(M\) be a (smooth) \(n\)-manifold. Show that each point \(p \in M\) has a relatively open neighborhood \(U \subseteq M\) such that \(U\) homeomorphic to \(\mathbb{R}^n\). In fact, show that there is a smooth regular embedding \(\varphi : \mathbb{R}^n \to U\) such that \(\varphi (\vec{0}) = p\).

    \begin{proof}
        Let \(p \in M\). By our assumption, we can find a smooth regular embedding \(\hat{\varphi}: \hat{V} \to V\), where \(\hat{V} \subseteq \mathbb{R}^n\) is open and \(V\) is a relatively open subset of \(M\) containing \(p\). Furthermore, there exists \(q \in \hat{V}\) such that \(\hat{\varphi}(q) = p\) and an open ball \(B_{\max}(q, r) \subseteq \hat{V}\). Note that we make use of an open ball with respect to the max-norm, for reasons that will become clear soon. Let \(U = \hat{\varphi}(B_{\max}(q, r))\). We are guaranteed that \(U\) is open because \(\hat{\varphi}\) is a homeomorphism. Now, we define a smooth homeomorphism between \(B_{\max}(q, r)\) and \(\mathbb{R}^n\). Let \(\Phi : \mathbb{R}^n \to B_{\max}(q, r)\) be defined by
        \[
            \Phi (\vec{x}) = \left( \frac{2}{\pi r}\arctan (x_1) + q_1, ..., \frac{2}{\pi r}\arctan (x_n) + q_n\right) 
        \]
        Notice that \(B_{\max}(q, r)\) is a cartesian product of intervals \(\prod _{i=1}^n [q_i - r, q_i + r]\). The function \(\Phi_i(x) = \frac{2}{\pi r}\arctan (x) + q_i\) is a well known bijection between \(\mathbb{R}\) and \([q_i - r, q_i + r]\), so it is clear that \(\Phi\) is a bijection with a continuous inverse. Moreover, each component is analytic, and therefore smooth, so \(\Phi\) is a smooth homeomorphism. We claim that our desired function \(\varphi: \mathbb{R}^n \to U\) is given by
        \[
            \varphi (x) = \hat{\varphi}(\Phi (x)).
        \]
        \(\varphi\) is a composition of smooth functions, and therefore smooth. \(J \Phi\) is a diagonal matrix with non-zero diagonals, so is rank \(n\). It follows that the Jacobian \(J \varphi (x) = J \hat{\varphi}(\Phi (x)) \cdot J \Phi (x)\) is rank \(n\) for all \(x\). Finally, \(\varphi\) is a homeomorphism since it is a composition of homeomorphisms (so \(U \simeq \mathbb{R}^n\), as needed). Thus we can conclude that \(\varphi\) is a smooth regular embedding. As well,
        \[
            \varphi (0) = \hat{\varphi}(\Phi (0)) = \hat{\varphi}(q) = p.
        \]
    \end{proof}
\end{document}