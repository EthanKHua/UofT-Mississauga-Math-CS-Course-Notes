\documentclass{article}
\usepackage[margin=1.0in]{geometry}
\usepackage{amssymb,amsmath,amsthm,amsfonts,mathtools}
\usepackage{enumitem}
\usepackage{xcolor}

\newlength\tindent
\setlength{\tindent}{\parindent}
\setlength{\parindent}{0pt}
\renewcommand{\indent}{\hspace*{\tindent}}

\newcommand{\Z}{\mathbf{Z}}
\newcommand{\N}{\mathbf{N}}
\newcommand{\R}{\mathbf{R}}
\newcommand{\Q}{\mathbf{Q}}
\newcommand{\C}{\mathbf{C}}

\newcommand{\id}{\mathrm{id}}
\newcommand{\op}{\mathrm{op}}
\newcommand{\diam}{\mathrm{diam}}
\newcommand{\GL}{\mathrm{GL}}
\newcommand{\Tr}{\mathrm{Tr}}

\newcommand{\cl}[1]{\overline{#1}}

\swapnumbers % places numbers before thm names

\theoremstyle{plain} % The "plain" style italicizes all body text.
	\newtheorem{thm}{Theorem}
		\numberwithin{thm}{section} % Theorem numbers are determined by section.
	\newtheorem{lemma}[thm]{Lemma}
	\newtheorem{prop}[thm]{Proposition}
	\newtheorem{cor}[thm]{Corollary}

\theoremstyle{definition}
    \newtheorem{defn}[thm]{Definition}
	\newtheorem{example}[thm]{Example}
	\newtheorem{exercise}[thm]{Exercise} %Exercise

\begin{document}
    \section*{Fubini's Theorem + Exercise 12.15}
    \textbf{Solvers:} Ethan
    Let \(E\) be a closed rectangle in \(\mathbb{R}^2\), so \(E = [a,b]\times [c,d]\). Let \(f: E \to \mathbb{R}\) be a continuous function. Then
    \[
        \int _E f = \int _c^d \int _a^b f(x,y)\ dx\ dy
    \]
    \begin{proof}
        We make the quick note that \(f(\cdot,y)\) is integrable because it is continuous. Moreover, it is uniformly continuous because it is defined on the closed interval \([a,b]\). Define \(g: [c,d] \to \mathbb{R}\) by
        \[
            g(y) = \int _a^b f(x,y)\ dx.
        \]
        We can argue that \(g\) is continuous because for any \(y\),
        \[
            |g(y + h) - g(y)| = \left\vert \int _a^b f(x,y + h)\ dx - \int _a^b f(x,y)\ dx\right\vert < \int _a^b |f(x, y + h) - f(x,y)|\ dx,
        \]
        which we can make arbitrarily small using the uniform continuity of \(f\).

        Let \(P\) partition \([a,b]\) and \(Q\) partition \([c,d]\), and let \(n,m\) represent the number of intervals in \(P\) and \(Q\) respectively. Define
        \[
            M_{ij} = \sup \{f(x,y) : (x,y) \in [p_{i-1}, p_i] \times [q_{j-1}, q_j]\}.
        \]
        We claim that for all \(j\), \(y \in [q_{j-1}, q_j]\),
        \[
            \sum_{i=1}^{n} M_{ij} (p_i - p_{i-1}) \geq U(f(\cdot,y), P).
        \]
        Define \(M_{iy} = \sup \{f(x,y) : x \in [p_{i-1}, p_i]\}\). We notice that the set in \(M_{iy}\) is a subset of the one in \(M_{ij}\), so we immediately have that \(M_{iy} < M_{ij}\). Thus
        \[
            U(f(\cdot, y), P) = \sum_{i=1}^{n} M_{iy} (p_{i-1}, p_i) \leq \sum_{i=1}^{n} M_{ij} (p_{i-1}, p_i).
        \]
        From this, we get that for any \(j\),
        \[
            g(y_j) = \int _a^b f(x,y)\ dx \leq \sum_{i=1}^{n} M_{ij} (p_{i-1}, p_i), \text{ where } y_j \in [q_{j-1}, q_j],
        \]
        which implies that
        \[
            \sum_{i=1}^{n} M_{ij} (p_{i-1}, p_i) \geq M_j \coloneqq \sup \{g(y) : y \in [q_{j-1}, q_j]\}.
        \]

        It follows that
        \[
            U(f,(P,Q)) = \sum_{j=1}^{m} \sum_{i=1}^{n} M_{ij} (p_i - p_{i-1})(q_j - q_{j-1}) \geq \sum_{j=1}^{m} M_j (q_j - q_{j-1}) = U(g, Q) \geq \int _c^d g(y)\ dy,
        \]
        \[
            \implies U(f,(P,Q)) \geq \int _c^d \int _a^b f(x,y)\ dx\ dy.
        \]
        We can follow the same steps from above but reversing the inequality signs to get that
        \[
            L(f,(P,Q)) \leq \int _c^d \int _a^b f(x,y)\ dx\ dy.
        \]
        Since we did this with arbitrary partitions \(P,Q\), we actually know that
        \[
            L(f) \leq \int _c^d \int _a^b f(x,y)\ dx\ dy \leq U(f),
        \]
        but \(L(f) = U(f)\) because of the integrability of \(f\), so we conclude that
        \[
            \int _E f = \int _c^d \int _a^b f(x,y)\ dx\ dy.
        \]
    \end{proof}
    \textbf{Example:} Integrate \(f(x,y) = x^y\) on \(E = [0,1]^2\).

    Note that \(f\) is undefined at \((0,0)\), but we can construct a function \(g\) that is the same as \(f\) but with \(g(0,0) = 1\) to solve this issue. It is also continuous, which I invite you to do as an exercise.

    Using Fubini's Theorem, we have that
    \[
        \int _E f = \int _0^1 \int _0^1 x^y\ dx\ dy = \int _0^1 \frac{x^{y+1}}{y + 1}\Big| _0^1\ dy = \int _0^1 \frac{1}{y + 1}\ dy = \ln |y + 1| \Big| _0^1 = \ln 2.
    \]
    Doing this from the definition is kind of painful, so having Fubini in your closet is quite useful.
\end{document}