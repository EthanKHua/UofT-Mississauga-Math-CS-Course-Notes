\documentclass{article}
\usepackage[margin=1.0in]{geometry}
\usepackage{amssymb,amsmath,amsthm,amsfonts,mathtools}
\usepackage{enumitem}
\usepackage{xcolor}

\newlength\tindent
\setlength{\tindent}{\parindent}
\setlength{\parindent}{0pt}
\renewcommand{\indent}{\hspace*{\tindent}}

\newcommand{\Z}{\mathbf{Z}}
\newcommand{\N}{\mathbf{N}}
\newcommand{\R}{\mathbf{R}}
\newcommand{\Q}{\mathbf{Q}}
\newcommand{\C}{\mathbf{C}}

\newcommand{\id}{\mathrm{id}}
\newcommand{\op}{\mathrm{op}}
\newcommand{\diam}{\mathrm{diam}}
\newcommand{\GL}{\mathrm{GL}}
\newcommand{\Tr}{\mathrm{Tr}}

\newcommand{\cl}[1]{\overline{#1}}

\swapnumbers % places numbers before thm names

\theoremstyle{plain} % The "plain" style italicizes all body text.
	\newtheorem{thm}{Theorem}
		\numberwithin{thm}{section} % Theorem numbers are determined by section.
	\newtheorem{lemma}[thm]{Lemma}
	\newtheorem{prop}[thm]{Proposition}
	\newtheorem{cor}[thm]{Corollary}

\theoremstyle{definition}
    \newtheorem{defn}[thm]{Definition}
	\newtheorem{example}[thm]{Example}
	\newtheorem{exercise}[thm]{Exercise} %Exercise

\begin{document}
    \section*{Exercise 13.14}
    \textbf{Solvers:} Ethan

    Recall the statement of Theorem 13.6 and the definition of a simple mapping:

    \textbf{Theorem 13.6.} Let $U,V\subseteq\R^n$ be two open sets and let $\Phi:U\rightarrow V$ be a diffeomorphism. Let $E\subseteq U$ be a measurable set. Then $\Phi(E)$ is also measurable, and its measure is
        \[ \mu(\Phi(E)) = \int_E |\det J\Phi|. \]

    \textbf{Definition 13.12.} Let $\Phi:U\rightarrow \R^n$ be a function defined on some set $U\subseteq \R^n$, and let $\varphi_1,\ldots,\varphi_n$ be the component functions of $\Phi$:
        \[ \Phi(p) = (\varphi_1(p), \ldots, \varphi_n(p)) \quad \text{for all $p\in U$.} \]
    If the $i$th component function is just $x_i$ --- \textit{i.e.} $\varphi_i(x_1,\ldots,x_n)=x_i$ for some $i$ --- then we say that $\Phi$ is \textbf{$x_i$-simple}. We say that $\Phi$ is \textbf{simple} if it is $x_i$-simple for all but one $i$.

    \textbf{Exercise 13.14.} Show that Theorem 13.6 holds if \(\Phi\) is simple.

    \begin{proof}
        Let \(U,V \subseteq \mathbb{R}^n\) be open, and let \(\Phi : U \to V\) be a simple diffeomorphism. Without loss of generality, let the only non-simple component of \(\Phi\) be \(\varphi _n\) (I will explain why this reduction is valid below!), and let \(E = \prod_{i=1}^n [a_i, b_i] \subseteq U\) be a closed box. Since \(\Phi\) is a diffeomorphism, this implies that \(\Phi (E)\) is measurable, and as well, because \(\Phi\) is simple,
        \[
            \Phi (E) = \left\{ (x_1, ..., x_{n-1}, \varphi (x)) \in V : x = (x_1, ..., x_n) \in \prod_{i=1}^{n} [a_i, b_i] \right\} 
        \]
        Let \(E' = \prod_{i=1}^{n-1} [a_i, b_i]\), and define \(\Psi : E' \to \mathbb{R}\) by
        \[
            \Psi (x') = \int_{\varphi (x', [a_n, b_n])} 1\ dx_n.
        \]
        \(\Phi\) is a diffeomorphism, so the \(x'\)-slice \(\varphi(x', \cdot)\) is \(C^1\) and injective. It follows that the image \(\varphi(x', [a_n, b_n])\) is an interval. Now, we want to apply single-variable change of variables with \(\varphi (x', u) = x_n\). Notice that
        \[
            dx_n = \frac{\partial \varphi }{\partial u}(x', u) du
        \]
        and
        \[
            \det J \Phi = \begin{array}{|cccc|}
                1 & 0 & \cdots &  0 \\
                0 & 1 & \cdots &  0 \\
                \vdots & \vdots & \ddots &  \vdots \\
                \dfrac{\partial \varphi }{\partial x_1}(x', u) & \dfrac{\partial \varphi }{\partial x_2}(x', u) & \cdots & \dfrac{\partial \varphi }{\partial u}(x', u) \\
            \end{array} = \frac{\partial \varphi }{\partial u}(x', u)
        \]
        so it follows that
        \[
            \Psi (x') = \int_{\varphi (x', [a_n, b_n])} 1\ dx_n = \int _{a_n}^{b_n} |\det J \Phi |
        \]
        Finally, using Fubini's Theorem, we get that
        \[
            \mu (\Phi (E)) = \int _{\Phi (E)} 1 = \int _{E'} \Psi = \int _{E'} \int_{a_n}^{b_n} |\det J \Phi | = \int _E |\det J \Phi |
        \]
        as needed.

        \textbf{Explanation of Reduction.} It, in fact, does not matter in which component \(\Phi\) is simple in.
        
        Let \(\Phi: U \to V\) be a simple diffeomorphism, where the \(i\)th component is non-simple, and we will denote this component as the function \(\varphi : U \to \mathbb{R}\). Let \(L\) be the linear map that swaps the \(i\)th and \(n\)th component. Notice that \(L = L^{-1}\) and \(\det L = \det L^{-1} = 1\) (this will be important later). We would like to apply the theorem to \(L \circ \Phi\), but this is not simple! However, we have a workaround for this issue. Let \(\Psi : U \to V\) be the function defined by \(\Psi (x_1, ..., x_n) = (x_1, ..., x_{n-1}, \varphi \circ L(x_1, ..., x_n))\). We claim that \(L \circ \Phi (E) = \Psi (E)\).

        Let \(x \in E\) and consider \(L \circ \Phi (x) = (x_1, ..., x_n, ..., \varphi (x))\). Let \(x' = L(x)\). We have that
        \[
            \Psi (x') = \Psi (x_1, ..., x_n, ..., x_i) = (x_1, ..., x_n, ..., \varphi (x)) = L \circ \Phi (x)
        \]
        We will not show the converse because it is the same argument. As well, convince yourself that \(\det J \Phi = \det J \Psi\). Now, we can apply the statement to the simple function \(\Psi\). Here is everything we need:
        \begin{itemize}
            \item \(\mu (L(R)) = |\det L| \mu (R)\): This is Exercise 13.11/12.18
            \item \(\det L = 1\)
            \item \(L \circ \Phi (E) = \Psi (E)\)
            \item The exercise above
            \item \(\det J \Psi = \det J \Phi\) 
        \end{itemize}
        We put it all together and see that
        \[
            \mu (\Phi (E)) = \mu (L^{-1} \circ L \circ \Phi (E)) = |\det L^{-1}| \mu (L \circ \Phi (E)) = \mu (\Psi (E)) = \int _E |\det J \Psi | = \int _E |\det J \Phi |
        \]
    \end{proof}
\end{document}