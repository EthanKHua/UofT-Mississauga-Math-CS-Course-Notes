\documentclass{article}
\usepackage[margin=1.0in]{geometry}
\usepackage{amssymb,amsmath,amsthm,amsfonts,mathtools}
\usepackage{enumitem}
\usepackage{xcolor}

\newcommand{\Z}{\mathbf{Z}}
\newcommand{\N}{\mathbf{N}}
\newcommand{\R}{\mathbf{R}}
\newcommand{\Q}{\mathbf{Q}}
\newcommand{\C}{\mathbf{C}}

\newcommand{\id}{\mathrm{id}}
\newcommand{\op}{\mathrm{op}}
\newcommand{\diam}{\mathrm{diam}}
\newcommand{\Tr}{\mathrm{Tr}}

\newcommand{\cl}[1]{\overline{#1}}

\swapnumbers % places numbers before thm names

\theoremstyle{plain} % The "plain" style italicizes all body text.
	\newtheorem{thm}{Theorem}
		\numberwithin{thm}{section} % Theorem numbers are determined by section.
	\newtheorem{lemma}[thm]{Lemma}
	\newtheorem{prop}[thm]{Proposition}
	\newtheorem{cor}[thm]{Corollary}

\theoremstyle{definition}
    \newtheorem{defn}[thm]{Definition}
	\newtheorem{example}[thm]{Example}
	\newtheorem{exercise}[thm]{Exercise} %Exercise

\begin{document}
    \setcounter{section}{5}
    \section{Week 6? i think}
    \noindent\textbf{Solvers:} Sanchit, Ethan
    
    \noindent\textbf{Writeup:} Ethan
    
    \textbf{Exercise 6.33.} Prove that $[0,1]^2$ is homeomorphic to the closed unit ball $\cl{B}(0,1)$ in $\R^2$.

    \begin{proof}
        Now, we continue to the main result. It is pretty easy to see that the closed box \([0,1]^2\) is homeomorphic to \([-1,1]^2\). We define the function \(h:[-1,1]^2 \to \cl{B}(0,1)\) by
        \[
            h(x,y)=\begin{dcases}
                \frac{\|(x,y)\| _{\max}}{\|(x,y)\| _2}(x, y), &\text{ if } (x,y) \neq (0,0) ;\\
                (0,0), &\text{ if } (x,y)=(0,0).
            \end{dcases}
        \]
        We can verify that this function is indeed well defined because for \((x,y) \in [-1,1]^2\),
        \[
            \left\lVert h(x,y) \right\rVert _2 \leq \left\lVert \frac{\|(x,y)\| _{\max}}{\|(x,y)\| _2}(x, y) \right\rVert _2 = \frac{\|(x,y)\| _{\max}}{\|(x,y)\| _2} \left\lVert (x,y) \right\rVert _2 = \|(x,y)\| _{\max} \leq 1
        \]
        which implies that \(h(x,y) \in \cl{B}(0,1)\).

        We show that this is a homeomorphism by first showing continuity, and then showing that the inverse is continuous.

        \textit{Fact!} Norms are continuous. Therefore \(h\) is continuous when \((x,y) \neq 0\). It remains to show continuity at \((0,0)\).

        Let \(\varepsilon > 0\). By the strong equivalence of norms on \(\mathbb{R}^2\), there is an \(M >0\) such that \(\dfrac{\|(x,y)\| _{\max}}{\|(x,y)\| _2} \leq M\) for all \((x,y) \in \mathbb{R}^2\). Let \(\delta = \frac{\varepsilon}{M}\). Let \((x,y) \in \mathbb{R}^2\) such that \(\|(x,y)\| < \delta\).
        Then
        \[
            \left\lVert h(x,y) \right\rVert = \left\lVert \frac{\|(x,y)\| _{\max}}{\|(x,y)\| _2}(x, y) \right\rVert = \frac{\|(x,y)\| _{\max}}{\|(x,y)\| _2} \left\lVert (x,y) \right\rVert \leq M \left\lVert (x,y) \right\rVert < \varepsilon
        \]
        Thus \(h\) is continuous everywhere. We can explicitly define the inverse \(h^{-1}\) as
        \[
            h^{-1} (x,y) = \begin{dcases}
                \frac{\|(x,y)\| _2}{\|(x,y)\| _{\max}} (x,y), &\text{ if } (x,y) \neq 0 ;\\
                (0,0), &\text{ if } (x,y) = (0,0).
            \end{dcases}
        \]
        By a similar argument, we can prove that \(h^{-1}\) is continuous. Therefore \(h\) is indeed a homeomorphism.

        Since \([0,1]^2 \cong [-1,1]^2 \cong \cl{B}(0,1)\), \([0,1]^2 \cong \cl{B}(0,1)\) by transitivity.
        
    \end{proof}
\end{document}