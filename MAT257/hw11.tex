\documentclass{article}
\usepackage[margin=1.0in]{geometry}
\usepackage{amssymb,amsmath,amsthm,amsfonts}
\usepackage{enumitem}
\usepackage{xcolor}
\usepackage{mathtools}

% My boxes
\usepackage[breakable]{tcolorbox}

% \RequirePackage{background}
% \backgroundsetup{
%     scale=1,
%     color=black,
%     opacity=1,
%     angle=0,
%     contents={
%         \includegraphics[width=\paperwidth,height=\paperheight]{\nightmodebackground}
%     }
% }

\definecolor{pastelblue}{RGB}{96, 145, 245}
\definecolor{pastelgreen}{RGB}{106, 235, 135}
\definecolor{darkgray}{RGB}{60, 60, 60}
\definecolor{lightgray}{RGB}{180, 180, 180}
\definecolor{offwhite}{RGB}{225, 225, 245}


\pagecolor{darkgray}
\color{offwhite}

\newcommand{\Z}{\mathbf{Z}}
\newcommand{\N}{\mathbf{N}}
\newcommand{\R}{\mathbf{R}}
\newcommand{\Q}{\mathbf{Q}}
\newcommand{\C}{\mathbf{C}}

\newcommand{\id}{\mathrm{id}}
\newcommand{\op}{\mathrm{op}}
\newcommand{\diam}{\mathrm{diam}}
\newcommand{\GL}{\mathrm{GL}}
\newcommand{\Tr}{\mathrm{Tr}}
\newcommand{\im}{\mathrm{im}}
\newcommand{\rank}{\mathrm{rank}}

\newcommand{\cl}[1]{\overline{#1}}

\swapnumbers % places numbers before thm names

\theoremstyle{plain} % The "plain" style italicizes all body text.
	\newtheorem{thm}{Theorem}
		\numberwithin{thm}{section} % Theorem numbers are determined by section.
	\newtheorem{lemma}[thm]{Lemma}
	\newtheorem{prop}[thm]{Proposition}
	\newtheorem{cor}[thm]{Corollary}

\theoremstyle{definition}
    \newtheorem{defn}[thm]{Definition}
	\newtheorem{example}[thm]{Example}
	\newtheorem{exercise}[thm]{Exercise} %Exercise

\begin{document}
    \newtcolorbox{question}[2][]{fonttitle=\large, fontupper=\large, fontlower=\large, title=Question {#2}., oversize, arc=3mm, outer arc=2mm, opacityback=0.9, coltitle=offwhite, colframe=pastelblue, colback=darkgray, colupper=lightgray, collower=lightgray, leftrule=1mm, rightrule=1mm, toprule=1.5mm, titlerule=1mm, bottomrule=1mm, valign=center, add to natural height=5mm, lower separated=false, before lower=\begin{proof}, after lower= \smallbreak \end{proof}, #1, breakable=true}

    \begin{question}{30}
        Let $U\subseteq \R^n$ be an open set in $\R^n$, and let $K$ be a compact subset of $U$. Prove that there exists an \textit{infinitely differentiable} function $\varphi:\R^n\rightarrow [0,1]$ such that $\varphi(p)=1$ for all $p\in K$, and $\varphi(p)=0$ for all $p\in \R^n\setminus U$. This is called a \textbf{bump function} supported on $U$.

        (For a function $f:U\rightarrow Y$, the \textbf{$n$th total derivative} $f^{(n)}$ is defined as follows: for $n=0$, we set $f^{(0)}=f$; for $n\geq 1$, if $f^{(n-1)}$ is totally differentiable, we set $f^{(n)} = (f^{(n-1)})'$. We say that $f$ is \textbf{infinitely differentiable} if $f^{(n)}$ exists for all $n\geq 0$.)

		\tcblower
		First, we notice that the bump function in Big List \#4 can be generalised to arbitrary intervals by simply performing horizontal translations. Now, we show that bump functions for closed rectangles within open rectangles in \(\mathbb{R}^n\) can be constructed.

		Let \(R = \prod _{i=1}^n [a_i, b_i]\) be a closed rectangle that is inside an open rectangle \(S = \prod _{i=1}^n (c_i, d_i)\) in \(\mathbb{R}^n\). This implies that for all \(i\), \(c_i < a_i \leq b_i < d_i\). Considering this as an interval in \(\mathbb{R}\), we can find a bump function \(\varphi _i\) such that \(\varphi _i([a_i, b_i]) = \{1\}\) and \(\varphi _i(\mathbb{R}\setminus (c_i, d_i)) = \{0\}\). Notice that this is a function from \(\mathbb{R}\) to \(\mathbb{R}\). We define \(\alpha _i : \mathbb{R}^n \to \mathbb{R}\) by \(\alpha _i (x) = \varphi _i (x_i)\). It will be shown that the bump function supported in \(S\) is
		\[
			\beta (x) = \prod _{i=1}^n \alpha _i(x)
		\]
		If \(p = (p_1, ..., p_n) \in R\), then for all \(i \in \{1, ..., n\}\), \(p_i \in [a_i, b_i]\), so \(\alpha _i (p) = 1\). We have
		\[
			\beta (p) = \prod _{i=1}^n \alpha _i(p) = 1
		\]
		Using a similar argument, if \(p \in \mathbb{R}^n \setminus S\), there is at least one component of \(p\) such that \(p_i \notin (c_i, d_i)\), and so \(\alpha _i(p) = 0\), which implies that
		\[
			\beta (p) = 0
		\]
		as desired.

		Next, let \(\mathbb{R}^n\) be equipped with the max-norm. We claim that \(\prod _{i=1}^n (p-r,p+r) \subseteq B(p,r)\), for \(p \in \mathbb{R}^n\) and \(r > 0\).

		Let \(x \in \prod _{i=1}^n (p-r,p+r)\). For all \(i\), \(|p_i - x_i| < r\), so \(\|x - p\| _{\max} < r\). Hence \(x \in B(p, r)\).

		Thus \(\prod _{i=1}^n (p-r,p+r) \subseteq B(p,r)\), and moreover by taking the closure of both sets,
		
		\(\prod _{i=1}^n [p-r,p+r] \subseteq \cl{B}(p,r)\).
		
		Now, we can proceed proving the main result.

		Let \(U\) be an open subset of \(\mathbb{R}^n\), and let \(K \subseteq U\) be compact. For every \(y \in K\), there exists an open ball centered around \(y\) so that \(B(y, \delta) \subseteq U\). Furthermore, we have that
		\[
			\prod _{i=1}^n \left[y - \frac{\delta}{2}, y + \frac{\delta}{2}\right] \subseteq \prod _{i=1}^n (y - \delta , y + \delta) \subseteq B(y, \delta) \subseteq U
		\]
		Notice that 
    \end{question}
	\newpage
	\begin{question}{31}
		\begin{enumerate}[label=(\alph*)]
			\item Let $A\in M_n(\R)$ be a symmetric matrix and let $Q(\vec{x})=\vec{x}^TA\vec{x}$ be the correpsonding quadratic form. Prove that the following two statements are equivalent:
			\begin{enumerate}[label=(\roman*)]
				\item $Q(\vec{x})> 0$ for all $\vec{x}\neq \vec{0}$.
		
				\item All eigenvalues of $A$ are strictly positive.
			\end{enumerate}
		
			In this case, we say that $Q$ is a \textbf{positive definite} quadratic form, and that $A$ is a \textbf{positive definite} matrix.
		
			\item Prove the following ``stay away'' lemma, which you will need for part (c): if $Q:\R^n\rightarrow \R$ is a positive definite quadratic form, then there exists a constant $\eta>0$ such that $Q(\vec{x})\geq \eta \|\vec{x}\|^2$ for all $\vec{x}\in \R^n$.
		
			\item Let $U\subseteq\R^n$ be an open set, let $f:U\rightarrow \R$ be a twice continuously differentiable function, and let $p_0\in U$ be a point at which $\nabla f(p_0)=\vec{0}$. Prove that if the Hessian matrix $Hf(p_0)$ is positive definite, then $f$ achieves a \textbf{local minimum} at $p_0$; \textit{i.e.} $p_0$ has an open neighborhood $U_0$, contained in $U$, such that $f(p)\geq f(p_0)$ for all $p\in U_0$.
		\end{enumerate}
		\tcblower
		(a):

		Suppose that \(Q(\vec{x}) > 0\) for all \(\vec{x} \neq 0\). Since \(A\) is symmetric, it is orthogonally diagonalizable, so there exists an orthogonal matrix \(P \in M_n(\mathbb{R})\) such that \(B = P^{\top} A P\) is diagonal. Letting \(Q_B(\vec{x}) = \vec{x}^{\top} B \vec{x}\), we have that
		\[
			Q(\vec{x}) = Q_B \circ P (\vec{x})
		\]
		We will show that \(Q_B(\vec{x}) > 0\) for all \(\vec{x} \neq 0\).

		Let \(\vec{x} \in \mathbb{R}^n\) so that \(\vec{x} \neq 0\). Since \(Q(\vec{x}) > 0\), it follows that \(Q_B(\vec{x}) > 0\) as well, which implies that its eigenvalues are strictly positive. Since \(A\) and \(B\) share the same eigenvalues, we are done with this direction.
	\end{question}
\end{document}