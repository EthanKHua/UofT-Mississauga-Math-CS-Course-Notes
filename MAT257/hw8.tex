\documentclass{article}
\usepackage[margin=1.0in]{geometry}
\usepackage{amssymb,amsmath,amsthm,amsfonts}
\usepackage{enumitem}
\usepackage{xcolor}
\usepackage{mathtools}

% My boxes
\usepackage[breakable]{tcolorbox}

% \RequirePackage{background}
% \backgroundsetup{
%     scale=1,
%     color=black,
%     opacity=1,
%     angle=0,
%     contents={
%         \includegraphics[width=\paperwidth,height=\paperheight]{\nightmodebackground}
%     }
% }

\definecolor{pastelblue}{RGB}{96, 145, 245}
\definecolor{pastelgreen}{RGB}{106, 235, 135}
\definecolor{darkgray}{RGB}{60, 60, 60}
\definecolor{lightgray}{RGB}{180, 180, 180}
\definecolor{offwhite}{RGB}{225, 225, 245}


\pagecolor{darkgray}
\color{offwhite}

\newcommand{\Z}{\mathbf{Z}}
\newcommand{\N}{\mathbf{N}}
\newcommand{\R}{\mathbf{R}}
\newcommand{\Q}{\mathbf{Q}}
\newcommand{\C}{\mathbf{C}}

\newcommand{\id}{\mathrm{id}}
\newcommand{\op}{\mathrm{op}}
\newcommand{\diam}{\mathrm{diam}}
\newcommand{\GL}{\mathrm{GL}}
\newcommand{\Tr}{\mathrm{Tr}}
\newcommand{\im}{\mathrm{im}}
\newcommand{\rank}{\mathrm{rank}}

\newcommand{\cl}[1]{\overline{#1}}

\swapnumbers % places numbers before thm names

\theoremstyle{plain} % The "plain" style italicizes all body text.
	\newtheorem{thm}{Theorem}
		\numberwithin{thm}{section} % Theorem numbers are determined by section.
	\newtheorem{lemma}[thm]{Lemma}
	\newtheorem{prop}[thm]{Proposition}
	\newtheorem{cor}[thm]{Corollary}

\theoremstyle{definition}
    \newtheorem{defn}[thm]{Definition}
	\newtheorem{example}[thm]{Example}
	\newtheorem{exercise}[thm]{Exercise} %Exercise

\begin{document}
    \newtcolorbox{question}[2][]{fonttitle=\large, fontupper=\large, fontlower=\large, title=Question {#2}., oversize, arc=3mm, outer arc=2mm, opacityback=0.9, coltitle=offwhite, colframe=pastelblue, colback=darkgray, colupper=lightgray, collower=lightgray, leftrule=1mm, rightrule=1mm, toprule=1.5mm, titlerule=1mm, bottomrule=1mm, valign=center, add to natural height=5mm, lower separated=false, before lower=\begin{proof}, after lower= \\ \end{proof}, #1, breakable=true}
    
    \begin{question}{23}
        Let $S^2$ denote the unit sphere in $\R^3$. Let $N=(0,0,1)$ denote the ``north pole''. In this problem, you will show that $S^2\setminus \{N\}$ is homeomorphic to $\R^2$. To do this, we define a function $\Phi:S^2\setminus \{N\}\rightarrow \R^2$ known as the \textbf{stereographic projection}: given a point $P$ in $S^2\setminus \{N\}$, draw a line between $P$ and $N$, and let $\Phi(P)$ denote the point where this line intersects the $xy$-plane in $\R^3$.
        \begin{enumerate}[label=(\alph*)]
            \item Given $P=(x,y,z)$, find an explicit formula for $\Phi(P)$ in terms of $x,y,z$.
            \item Deduce that $\Phi$ is continuous.
            \item Prove that $\Phi$ is a bijection; in fact, given $p=(s,t)\in \R^2$, find an explicit formula for $\Phi^{-1}(p)$.
            \item Deduce that $\Phi$ is a homeomorphism.
        \end{enumerate}
        \tcblower
        (a):

        Let \(P=(x,y,z)\). First, we find the equation of the line that passes \(P\) and \(N\). Consider the equation of the line \(L(t) = (tx, ty, (z-1)t + 1)\). Notice that \(L(0) = N\) and \(L(1) = P\), so \(L\) satisfies what we were looking for. Now we find the point where \(L\) intersects with the \(xy\)-plane. This happens exactly when \((z-1)t+1=0\). Solving for \(t\) gives \(t=\dfrac{1}{1-z}\). This value is always defined as \(z\neq 1\). As a result, it turns out that
        \[
            L\left(\frac{1}{1-z}\right) = \left(\frac{x}{1-z}, \frac{y}{1-z}, 0\right) \text{.}
        \]
        Thus
        \[
            \Phi (P) = \frac{1}{1-z}\left(x,y\right) \text{.}
        \]

        (b):

        (c):

        Let \(p = (s,t) \in \mathbb{R}^2\). Our goal is to find \((x,y,z) \in S^2 \setminus \{N\}\) such that \(\Phi (x,y,z) = \left( \frac{x}{1-z}, \frac{y}{1-z} \right) = (s,t)\). Immediately, we obtain the following system of equations:
        \[
            \frac{x}{1-z} = s \text{,} 
        \]
        \[
            \frac{y}{1-z} = t \text{,} 
        \]
        \[
            x^2 + y^2 + z^2 = 1
        \]
        We also have the restriction \(z\neq 1\) because \((x,y,z) \neq N\). Isolating for \(x\) and \(y\) yields
        \[
            x=s(1-z)
        \]
        \[
            y=t(1-z)
        \]
        Then we substitute this into the third equation and get
        \[
            s^2(1-z)^2 + t^2(1-z)^2 + z^2 = 1 \implies (s^2 + t^2 + 1)z^2 - 2(s^2 + t^2)z + s^2 + t^2 - 1 = 0
        \]
        We can replace the term \(t^2 + s^2\) with \(\|p\| _2 ^2\), and the equation becomes
        \[
            (\|p\| _2 ^2 + 1)z^2 - 2\|p\| _2 ^2 z + \|p\| _2 ^2 - 1 = 0
        \]
        Using the quadratic formula:
        \[
            z = \frac{2(\|p\| _2 ^2) \pm \sqrt{4(\|p\| _2 ^2)^2 - 4(\|p\| _2 ^2 +1)(\|p\| _2 ^2 - 1)}}{2(\|p\| _2 ^2 + 1)}
        \]
        \[
            \implies z = \frac{\|p\| _2 ^2 \pm \sqrt{\|p\| _2 ^4 - (\|p\| _2 ^4 - 1)}}{\|p\| _2 ^2 + 1}
        \]
        \[
            z = \frac{\|p\| _2 ^2 \pm 1}{\|p\| _2 ^2 + 1}
        \]
        Notice that we cannot use the positive solution, for then
        \[
            z = \frac{\|p\| _2 ^2 + 1}{\|p\| _2 ^2 + 1} = 1
        \]
        Thus it must be true that
        \[
            z = \frac{\|p\| _2 ^2 - 1}{\|p\| _2 ^2 + 1}
        \]
        \[
            x = s(1-z) = \frac{2s}{\|p\| _2 ^2 + 1}
        \]
        \[
            y = \frac{2t}{\|p\| _2 ^2 + 1}
        \]
        It can be verified that these values of \(x,y,z\) result in \(\Phi (x,y,z) = (s,t)\). In fact, using this, we obtain that the formula for \(\Phi ^{-1}\) is
        \[
            \Phi ^{-1} (s,t) = \left(\frac{2s}{\|p\| _2 ^2 + 1}, \frac{2t}{\|p\| _2 ^2 + 1}, \frac{\|p\| _2 ^2 - 1}{\|p\| _2 ^2 + 1}\right)
        \]
    \end{question}
    \pagebreak
    \begin{question}{24}
        Let $X$ be a normed vector space. Prove that the following statements are equivalent.
        \begin{enumerate}[label=(\roman*)]
            \item $X$ is finite-dimensional.
            \item The unit ball $\cl{B}(\vec{0},1)$ is compact.
            \item $X$ is \textbf{locally compact}: each point $p\in X$ is contained in some open set $U$ such that $\cl{U}$ is compact.
        \end{enumerate}
        \tcblower
        It will be proven that (i) \(\implies\) (ii) \(\iff\) (iii) \(\implies\) (i).

        (i) \(\implies\)  (ii):

        Suppose that \(X\) has finite dimension \(n\). Then there is a continuous linear isomorphism \(\Phi\) between \(X\) and \(\mathbb{R}^n\). Since \(\cl{B}(0,1)\) is closed and bounded, \(\Phi (\cl{B}(0,1))\) is also closed and bounded in \(\mathbb{R}^n\), so the set is compact. Since homeomorphisms preserve compactness, we can conclude that the closed unit ball in \(X\) is compact.

        (ii) \(\implies\) (iii):

        Suppose that the unit ball \(\cl{B}(\vec{0}, 1)\) is compact. Let \(p \in X\). We claim that \(U = B(p,1)\). Consider \(\cl{U} = \cl{B}(p,1)\). There is an isometry \(\Phi\) from \(\cl{B}(0,1)\) to \(\cl{B}(p,1)\) defined by \(\Phi (x) = p + x\). Since the closed unit ball is compact, it follows that \(\cl{B}(p,1)\) is compact. Thus \(X\) is locally compact.

        (iii) \(\implies\) (ii):

        Suppose that \(X\) is locally compact. Then \(\vec{0}\) is contained in an open set \(U\) such that \(\cl{U}\) is compact. Since \(U\) is open, \(B(0, \varepsilon) \subseteq U\) for some \(\varepsilon\)-ball centered around 0. It follows that \(\cl{B}(0,\varepsilon)\) is compact, as it is a closed subset of \(\cl{U}\). Since there is a homeomorphism from this closed ball to \(\cl{B}(0,1)\), the closed unit ball is also compact, as desired.

        (iii) \(\implies\) (i):

        Suppose that \(X\) is locally compact. We know previously that this implies that the closed unit ball is compact. Firstly, a quick lemma will be proven.

        \textbf{Lemma 1.} \(\forall x \in X\), \(r > 0\), \(B(x,r) = \{x\} + B(0,r)\).

        Let \(y \in B(x,r)\). Notice that \(y-x \in B(0,r)\). Thus \(y = x + (y-x)\) so \(y \in \{x\} + B(0,r)\).

        Then, let \(y \in \{x\} + B(0,r)\), so \(y = x + s\), for some \(s \in B(0,r)\). Since \(\|y-x\| = \|s\| < r\), it follows that \(y \in B(x,r)\) and we are done.
        
        Moving on, we construct a finite set of vectors in the following way:

        Since \(\cl{B}(0,1)\) is totally bounded, we can find a finite set of vectors \(\beta\) such that \(\bigcup_{x \in \beta} B\left(x,\frac{1}{2}\right)\).

        \textbf{Lemma 2.} \(B(0,1) \subseteq \operatorname{span}(\beta) + 2^{-n} B(0,1)\), \(\forall n \in \mathbb{N}\).

        To prove this, let \(n \in \mathbb{N}\) and \(y \in B(0,1)\). It follows that \(y \in B(x_1, \frac{1}{2})\) for some \(x_1 \in \beta\). Using Lemma 1,
        \[
            y \in \{x_1\} + B\left(0,\frac{1}{2}\right) = \{x_1\} + \frac{1}{2}B(0,1)
        \]
        We can repeatedly use this argument to obtain that
        \[
            y \in \left\{x_1 + \frac{1}{2}x_2 + \frac{1}{2^2}x_3 + \dots + \frac{1}{2^{n-1}}x_n\right\} + \frac{1}{2^n}B(0,1)
        \]
        Note that \(x_i\) are not necessarily distinct. This simplifies down to
        \[
            y \in \operatorname{span}(\beta) + \frac{1}{2^n}B(0,1) \implies B(0,1) \subseteq \operatorname{span}(\beta) + \frac{1}{2^n}B(0,1)
        \]
        which is what we wanted.

        Finally, we claim that \(\beta\) is the basis for \(X\).

        Let \(v \in X\). Since \(\dfrac{v}{2\|v\|} \in B(0,1)\), by Lemma 2,
        \[
            \frac{v}{2\|v\|} \in \operatorname{span}(\beta) + \frac{1}{2^n}B(0,1) \text{, for all } n \in \mathbb{N}
        \]
        Then we construct a sequence \(\{v_n\}_{n\geq 1}\) in \(\operatorname{span}(\beta)\) where \(v_n \in \operatorname{span}(\beta)\) such that \(y = v_n + p\), for some \(p \in \frac{1}{2^n}B(0,1)\). Notice that
        \[
            \|y - v_n\| = \|p\| < \frac{1}{2^n}
        \]
        which can become arbitrarily small. This implies that \(y\) is a limit point for \(\operatorname{span}(\beta)\). However, this set is closed because it is finite dimensional, so we also have that \(y \in \operatorname{span}(\beta)\). Therefore \(X \subseteq \operatorname{span}(\beta)\), and since \(X\) is spanned by a finite set, it is finite dimensional.

        These implications are sufficient for proving that (i), (ii), (iii) are equivalent.
    \end{question}
    
\end{document}