\documentclass{article}
\usepackage[margin=1.0in]{geometry}
\usepackage{amssymb,amsmath,amsthm,amsfonts}
\usepackage{enumitem}
\usepackage{xcolor}
\usepackage{mathtools}

\newcommand{\Z}{\mathbf{Z}}
\newcommand{\N}{\mathbf{N}}
\newcommand{\R}{\mathbf{R}}
\newcommand{\Q}{\mathbf{Q}}
\newcommand{\C}{\mathbf{C}}

\newcommand{\id}{\mathrm{id}}
\newcommand{\op}{\mathrm{op}}
\newcommand{\diam}{\mathrm{diam}}
\newcommand{\GL}{\mathrm{GL}}
\newcommand{\Tr}{\mathrm{Tr}}
\newcommand{\im}{\mathrm{im}}
\newcommand{\rank}{\mathrm{rank}}

\newcommand{\cl}[1]{\overline{#1}}

\swapnumbers % places numbers before thm names

\theoremstyle{plain} % The "plain" style italicizes all body text.
	\newtheorem{thm}{Theorem}
		\numberwithin{thm}{section} % Theorem numbers are determined by section.
	\newtheorem{lemma}[thm]{Lemma}
	\newtheorem{prop}[thm]{Proposition}
	\newtheorem{cor}[thm]{Corollary}

\theoremstyle{definition}
    \newtheorem{defn}[thm]{Definition}
	\newtheorem{example}[thm]{Example}
	\newtheorem{exercise}[thm]{Exercise} %Exercise

\begin{document}
    \setcounter{section}{4}
    \section{Homework 5}
    \noindent\textbf{13.} Let $(X,\|\cdot\|_X)$, $(Y,\|\cdot\|_Y)$ be two normed vector spaces and let $T:X\rightarrow Y$ be a linear mapping. %\due{10/4}
    \begin{enumerate}[label=(\alph*)]
        \item Prove that $T$ is continuous if and only if $T$ is a bounded linear mapping.
        
        \begin{proof}
            Suppose that \(T\) is continuous. Then at \(x_0 = 0\), there exists a \(\delta >0\) such that
            \[
                \|x\| _X < \delta \implies \|T(x)\| _2 \leq 1
            \]
            We claim that our \(M = \frac{2}{\delta}\).Let \(x \in X\). Notice that \(\left\lVert\dfrac{\delta \cdot x}{2\|x\| _X}\right\rVert _X <= \delta\). By the continuity of \(T\) we have that
            \[
                \left\lVert T\left(\dfrac{\delta \cdot x}{2\|x\| _X} \right)\right\rVert _Y \leq 1 \implies \dfrac{\delta}{2\cdot \|x\| _X}\left\lVert T\left(x\right) \right\rVert _Y \leq 1 \implies \left\lVert T\left(x\right) \right\rVert _Y\leq \dfrac{2}{\delta}\|x\| _X
            \]
            Thus \(T\) is a bounded linear mapping.

            Next, suppose that \(T\) is a bounded linear mapping. Then there is an \(M > 0\) such that for all \(x \in X\),
            \[
                \|T(x)\|_Y \leq M\|x\| _X
            \]
            We will show that \(T\) is continuous everywhere. Fix \(a \in X\). Let \(\varepsilon > 0\). Let \(\delta = \dfrac{\varepsilon}{M}\). Let \(x \in X\) and suppose that \(\|x - a\| _X < \delta\). Then
            \[
                \|T(x) - T(a)\| _Y = \|T(x-a)\| _Y \leq M\|x-a\| _X < \varepsilon
            \]
            and we are done. 
        \end{proof}
        
        \item Suppose that $(Y,\|\cdot\|_Y)=(\R^n,\|\cdot\|_2)$. Prove that $T$ is continuous if and only if $\ker(T)$ is closed.
        \begin{proof}
            Suppose that \(T\) is continuous. Let \(a\) be a limit point for \(\ker (T)\). We want to show that \(T(a)=0\). 
            
            By definition, \(T\) being continuous implies that
            \[
                \lim_{x \to a} T(x) = T(a)
            \]
            or more formally, letting \(\varepsilon\) be arbitrary, there exists a \(\delta\) such that
            \[
                \|x - a\| _X < \delta \implies \|T(x) - T(a)\| _2
            \]

            Since \(a\) is a limit point of \(\ker (T)\), there exists \(x \in \ker (T)\) such that
            \[
                x \in B(a, \delta) \implies \|x -a\| _X \implies \|T(x)-T(a)\| _2 < \varepsilon \implies \|T(a)\| _2 < \varepsilon
            \]
            By properties of norms \(\|T(a)\| _2 \geq 0\), so we must have that \(T(a) = 0 \implies a \in \ker (T)\). Thus \(\ker(T)\) is closed.

            I have no clue how to prove the converse so yeah.

        \end{proof}
    \end{enumerate}
    \pagebreak
    \noindent\textbf{16.} Let $(X,\|\cdot\|)$ be a normed vector space, let $B(X)=B(X,X)$ be the space of bounded linear operators on $X$, and equip $B(X)$ with the operator norm $\|\cdot\|_{\op}$. Let $\GL(X)$ be the set of invertible bounded linear operators:
        \[ \GL(X) = \{T\in B(X) : \text{$T$ is invertible}\}. \]
    (The notation $\GL$ means ``general linear group.'')
    
    Prove that if $(X,\|\cdot\|)$ is complete, then $\GL(X)$ is an open subset of $(B(X),\|\cdot\|_{\mathrm{op}})$.
    
    \begin{proof}
        First, we prove a couple lemmas.

        \noindent\textbf{Lemma 1.}
            For all \(S\), \(T \in B(X,X)\),
            \[
                \|ST\| _{\op} \leq \|S\| _{\op} \|T\| _{\op}
            \]
            To prove this, let \(x \in X\) so that \(\|x\| _X \leq 1\). Then
            \[
                \|S(T(x))\| _X = \left\lVert S\left(\dfrac{\|T(x)\| _X}{\|T(x)\| _X} T(x)\right)\right\rVert _X= \|T(x)\| _X \left\lVert S\left(\dfrac{T(x)}{\|T(x)\| _X}\right)\right\rVert _X
            \]
            Notice that \(\left\lVert\dfrac{T(x)}{\|T(x)\| _X}\right\rVert _X = 1\). We obtain that
            \[
                \|S(T(x))\| _X = \|T(x)\| _X \left\lVert S\left(\dfrac{T(x)}{\|T(x)\| _X}\right)\right\rVert _X \leq \|T\| _{\op} \|S\| _{\op}
            \]
            Since \(\|T\| _{\op} \|S\| _{\op}\) is an upper bound for \(\|S(T(x))\| _X\) when \(\|x\| _X \leq 1\), by definition of supremum we can conclude that \(\|ST\| _{\op} \leq \|S\| _{\op} \|T\| _{\op}\), proving the first lemma.

            Next, the following lemma will help in proving Lemma 3.

            \noindent\textbf{Lemma 2.} If \(T,U \in B(X)\), then for \(x \in X\),
            \[
                \|T(x)-U(x)\| _X \leq \|X\| _X \|T-U\| _{\op}
            \]
            Let \(T,U \in B(X)\). Then
            \[
                \|(T-U)(x)\| _X = \left\lVert \|x\| _X (T-U)\left( \dfrac{x}{\|x\| _X} \right)  \right\rVert = \|x\| _X \left\lVert (T-U)\left( \dfrac{x}{\|x\| _X} \right)  \right\rVert 
            \]
            Since \(\dfrac{x}{\|x\| _X} =1\),
            \[
                \|x\| _X \left\lVert (T-U)\left( \dfrac{x}{\|x\| _X} \right)  \right\rVert \leq \|x\| _X \|T-U\| _{\op}
            \]
            which is what we wanted. 

            We will use the proceeding lemma to help us prove Lemma 4.

            \noindent\textbf{Lemma 3.} If \(X\) is complete then \(B(X,X)\) is complete.

            For every Cauchy sequence \((T_n)\) in \(B(X,X)\), the sequence \((T_n (x_i))\), where \(x_i\) is an arbitrary element in \(X\), is a Cauchy sequence in \(X\). By the completeness of \(X\), \((T_n(x_i))\) converges to some \(L_i\). Let \(T\) be a linear operator on \(X\) defined by \(T(x_i) = L_i\). We can see that this is indeed a linear operator because
            \[
                cT(x_i) + T(x_j) = cL_i + L_j = c \lim_{n \to \infty} T_n(x_i) + \lim_{n \to \infty} T_n(x_j) = \lim_{n \to \infty} cT_n(x_i) + T_n(x_j) = \lim_{n \to \infty} T_n(cx_i + x_j)
            \]
            \(X\) is closed, so \(cx_i + x_j\) is equal to some \(x_k \in X\). Thus
            \[
                cT(x_i) + T(x_j) = \lim_{n \to \infty} T_n(cx_i + x_j) = \lim_{n \to \infty} T_n(x_k) = L_k = T(x_k)
            \]
            which implies that \(T\) is a linear operator.

            To show that \(T\) is bounded, we know that \(T_n\) is bounded, so \(\|T_n(x)\| _X < M_n \|x\| _X\), for some \(M_n > 0\). As well, we have that \(T_n\) is uniformly continuous. Thus for a sufficiently large \(n \in \mathbb{N}\), for all \(x \in X\), using Lemma 2,
            \[
                \|T(x)\| _X = \lim_{m \to \infty} \|T_m(x)\| _X \leq \lim_{m \to \infty} \|T_m(x) - T_n(x)\| _X + \|T_n(x)\| _X < \lim_{m \to \infty} \|T_n-T_m\| _{\op} \|x\| _X + \|T_n(x)\| _X
            \]
            \[
                \leq \|x\| _X + M_n \|x\| _X
            \]
            Thus \(T \in B(X)\). To show that \(T_n\) converges to \(T\), fix \(\varepsilon > 0\). For all \(x\in X\) such that \(\|x\| _X \leq 1\), there exists a sufficiently large \(N\) such that for all \(x \in X\), if \(n > N\) then
            \[
                \|T(x) - T_n(x)\| _X < \varepsilon \implies \|T-T_n\| _{\op} < \varepsilon
            \]
            Thus we can conclude that \(B(X,X)\) is complete.

            \noindent\textbf{Lemma 4.} If \(T \in B(X,X)\) such that \(\|I - T\| _{\op} < 1\), then \(T\) is invertible and
            \[
                T^{-1} = \sum_{i=0}^{\infty}(I-T)^i
            \]
            Suppose that \(T \in B(X,X)\) and \(\|I - T\| _{\op} < 1\). Let \((a_n)_{n\geq1}\) be a Cauchy sequence in \(B(X,X)\) defined by
            \[
                a_n = \sum_{i=0}^n (I-T)^i
            \]
            We verify that this sequence is Cauchy. Let \(\varepsilon > 0\). Since \(\|I - T\| _{\op} < 1\), there exists natural \(N\) such that
            \[
                \frac{\|I-T\| _{\op} ^N}{1-\|I-T\| _{\op}} < \varepsilon
            \]
            Let \(m,n > N\) and suppose that \(m < n\). Then
            \[
                \left\lVert a_n - a_m\right\rVert _{\op} = \left\lVert\sum_{i=0}^{n} (I-T)^i - \sum_{i=0}^{m} (I-T)^i \right\rVert _{\op} = \left\lVert\sum_{i=m+1}^n (I-T)^i \right\rVert _{\op}
            \]
            By the triangle inequality and Lemma 1,
            \[
                \left\lVert\sum_{i=m+1}^n (I-T)^i \right\rVert \leq \sum_{i=m+1}^n \left\lVert (I-T)^i \right\rVert \leq \sum_{i=m+1}^n \left\lVert I-T \right\rVert ^i = \left\lVert I-T \right\rVert ^{m+1} \sum_{i=0}^{n-m-1} \left\lVert I-T \right\rVert ^i
            \]
            \[
                = \left\lVert I-T \right\rVert ^{m+1} \dfrac{1-\|I-T\|^{n-m}}{1-\|I-T\|}
            \]
            \(1-\|I-T\|^{n-m}<1\), so
            \[
                \left\lVert I-T \right\rVert ^{m+1} \dfrac{1-\|I-T\|^{n-m}}{1-\|I-T\|} < \dfrac{\|I-T\|^{m+1}}{1-\|I-T\|} < \dfrac{\|I-T\|^{N}}{1-\|I-T\|} < \varepsilon
            \]
            We see that \((a_n)\) is indeed a Cauchy sequence.

            By Lemma 4, \(B(X,X)\) is complete, so \((a_n)\) converges, meaning that \(\sum_{i=0}^{\infty}(I-T)^i\) exists. Now, notice that
            \[
                T\left( \sum_{i=0}^{\infty}(I-T)^i \right) = (I - (I-T))\left( \sum_{i=0} ^{\infty} (I-T)^i \right) = \sum_{i=0} ^{\infty} (I-T)^i - \sum_{i=0} ^{\infty} (I-T)^{i+1} = \sum_{i=0} ^{\infty} \left( (I-T)^i - (I-T)^{i+1} \right)
            \]
            This is a telescoping series. As \(i \to \infty\), \((I-T)^i \to 0\). Thus
            \[
                T\left( \sum_{i=0}^{\infty}(I-T)^i \right) = \sum_{i=0} ^{\infty} \left( (I-T)^i - (I-T)^{i+1} \right) = I
            \]
            Thus \(\sum_{i=0}^{\infty}(I-T)^i\) is the inverse of \(T\), which implies that \(T^{-1}\) exists.
            \\\\
            Now we can show that \(B(X)\) is open.

            Let \(T \in B(X)\). Consider the open ball \(B(T, \frac{1}{\|T^{-1}\| _{\op}})\). For all \(S \in B(T, \frac{1}{\|T^{-1}\| _{\op}})\),
            \[
                \|T - S\| _{\op} < \frac{1}{\|T^{-1}\| _{\op}} \implies \|T-S\| _{\op} \|T^{-1}\| _{\op} < 1
            \]
            By Lemma 1, we have that
            \[
                1 > \|T-S\| _{\op} \|T^{-1}\| _{\op} \geq \|(T-S)T^{-1}\| _{\op} = \|I - ST^{-1}\| _{\op}
            \]
            By Lemma 4, \(ST^{-1}\) is invertible. If we let \(S^{-1} = T^{-1}(ST^{-1})^{-1}\), we see that
            \[
                S S^{-1} = S (T^{-1} (ST^{-1})^{-1}) = (ST^{-1})(ST^{-1})^{-1} = I
            \]
            Thus \(S\) is invertible, which means that \(S \in B(X)\). Therefore \(B(X)\) is open.

    \end{proof}

    \pagebreak
    \noindent\textbf{17.} \textit{The magic number lemma.}

    Let $(X,d)$ be a metric space and let $\{U_i\}_{i\in I}$ be an open cover of $X$; this means that each $U_i$ is an open subset of $X$, and that $X=\bigcup_{i\in I} U_i$. A \textbf{magic number} for $\{U_i\}_{i\in I}$ is a number $\delta>0$ with the following property: if $A\subseteq X$ is a set with $\diam(A)<\delta$, then $A\subseteq U_i$ for at least one index $i\in I$.

    Suppose that $(X,d)$ is a clustering metric space. Prove that every open cover has a magic number.

    \begin{proof}
        Suppose that \((X,d)\) is a clustering metric space. Suppose for the sake of contradiction that there exists an open cover \(\{U_i\}_{i\in I}\) that doesnt have a magic number.

        For \(n \in \mathbb{N}\), there is \(A_n \subseteq X\) with \(\diam A_n < \frac{2}{n}\) so that \(A_n \nsubseteq U_i\) for all indices \(i \in I\). Since \(\diam A_n < \frac{2}{n}\), we can cover \(A_n\) with an open ball \(B(a_n, \frac{1}{n})\), where \(a_n\) is some element in \(X\).

        Define a sequence \((a_n)_{n\in \mathbb{N}}\) in \(X\) such that \(a_n\) is equal to the one above.

        By the clustering property of \(X\), \((a_n)\) has a convergent subsequence, which will be redefined as \((a_n)\). Denote the limit of \((a_n)\) as \(p\).

        \(p\) is an element of \(X\), so it is contained in some \(U_i\) in the open cover. Since \(U_i\) is open, we can find \(\varepsilon > 0\) such that \(B(p, \varepsilon) \subseteq U_i\). As well, since \((a_n)\) converges to \(p\) we can find infinitely many entries of the sequence within the open ball \(B(p, \frac{\varepsilon}{2})\). Thus we can find a large enough \(n\) such that \(n > \frac{2}{\varepsilon}\), which gives \(\frac{1}{n} < \frac{\varepsilon}{2}\), and still have that \(a_n\) is \(\frac{\varepsilon}{2}\)-close to \(p\).

        Now, consider the open ball \(B(a_n, \frac{1}{n})\). We will show that \(B(a_n, \frac{1}{n}) \subseteq B(p, \varepsilon)\).

        Let \(x \in B(a_n, \frac{1}{n})\). Then \(d(x, a_n) < \frac{1}{n} < \frac{\varepsilon}{2}\), so we have
        \[
            d(x,p) \leq d(x,a_n) + d(a_n,p) < \frac{\varepsilon}{2} + \frac{\varepsilon}{2} = \varepsilon \implies x \in B(p, \varepsilon)
        \]
        which is what we wanted.

        Recall that the set \(A_n\) is covered by \(B(a_n, \frac{1}{n})\). Then we have
        \[
            A_n \subseteq B(a_n, \frac{1}{n}) \subseteq B(p, \varepsilon) \subseteq U_i
        \]
        contradicting the fact that \(A_n \nsubseteq U_i\). Thus every open cover has a magic number.

    \end{proof}
\end{document}