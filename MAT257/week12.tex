\documentclass{article}
\usepackage[margin=1.0in]{geometry}
\usepackage{amssymb,amsmath,amsthm,amsfonts,mathtools}
\usepackage{enumitem}
\usepackage{xcolor}

\newcommand{\Z}{\mathbf{Z}}
\newcommand{\N}{\mathbf{N}}
\newcommand{\R}{\mathbf{R}}
\newcommand{\Q}{\mathbf{Q}}
\newcommand{\C}{\mathbf{C}}

\newcommand{\id}{\mathrm{id}}
\newcommand{\op}{\mathrm{op}}
\newcommand{\diam}{\mathrm{diam}}
\newcommand{\GL}{\mathrm{GL}}
\newcommand{\Tr}{\mathrm{Tr}}

\newcommand{\cl}[1]{\overline{#1}}

\swapnumbers % places numbers before thm names

\theoremstyle{plain} % The "plain" style italicizes all body text.
	\newtheorem{thm}{Theorem}
		\numberwithin{thm}{section} % Theorem numbers are determined by section.
	\newtheorem{lemma}[thm]{Lemma}
	\newtheorem{prop}[thm]{Proposition}
	\newtheorem{cor}[thm]{Corollary}

\theoremstyle{definition}
    \newtheorem{defn}[thm]{Definition}
	\newtheorem{example}[thm]{Example}
	\newtheorem{exercise}[thm]{Exercise} %Exercise

\begin{document}
    \section*{Exercise 10.9}
    \textbf{Solvers:} Ethan

    \noindent\textbf{Writeup:} Ethan

    Let \(U \subseteq \mathbb{R}^n\) be open, and let \(\Phi : \mathbb{R}^n \to \mathbb{R}^n\) be a \(C^1\) mapping. Suppose that there is a point \(p_0 \in U\) such that \(\Phi '(p_0) : \mathbb{R}^n \to \mathbb{R}^n\) is an isomorphism.

    \medskip

    Define the function \(\Psi = \Phi ^{-1}\). From the previous exercises, we know that \(\Psi '(q) = \Phi '(\Psi (q)) ^{-1}\). It remains to show that \(\Psi '\) is continuous.

    \noindent\textbf{Exercise 10.9} \(\Psi '\) is continuous.

    \begin{proof}
        Consider the inversion map \(T : \GL(\mathbb{R}^n) \to \GL(\mathbb{R}^n)\). That is, for an invertible linear mapping \(A\) in \(\mathbb{R}^n\), \(T(A) = A^{-1}\). It will be shown that \(T\) is continuous.

        Fix \(A \in \GL(\mathbb{R}^n)\) and let \(\varepsilon > 0\). Since \(\GL(\mathbb{R}^n)\) is open, for some \(\delta _0\), \(B(A, \delta _0) \subseteq \GL(\mathbb{R}^n)\). Let \(\delta = \min \{\delta _0, \|A\|^{-2}\}\). Let \(h \in \GL(\mathbb{R}^n)\) so that \(\|h\| _{\op} < \delta\). Fix \(x \in \mathbb{R}^n\) such that \(\|x\| = 1\). Since \((A + h)^{-1}\) is surjective, we have \(x = (A+h)^{-1} (y)\), for some \(y \in \mathbb{R}^n\). We see that
        \[
            fasdfg
        \]
        \[
            \|T(A + h) - T(A)\|_{\op} = \|(A+h)^{-1} \circ (A+h) \circ ((A+h)^{-1} - A^{-1})\|_{\op}
        \]
        \[
            = \|(A+h)^{-1} \circ ((A+h)\circ (A+h)^{-1} - (A+h) \circ A^{-1})\|_{\op} = \|(A+h)^{-1} \circ (- h \circ A^{-1})\|_{\op}
        \]
        \[
            \leq \|(A+h)^{-1}\|_{\op} \|h\|_{\op} \|A^{-1}\|_{\op}
        \]
    \end{proof}
\end{document}