\documentclass{article}
\usepackage[margin=1.0in]{geometry}
\usepackage{amssymb,amsmath,amsthm,amsfonts}
\usepackage{enumitem}
\usepackage{xcolor}
\usepackage{mathtools}
\usepackage{systeme}

% My boxes
\usepackage[breakable]{tcolorbox}

% \RequirePackage{background}
% \backgroundsetup{
%     scale=1,
%     color=black,
%     opacity=1,
%     angle=0,
%     contents={
%         \includegraphics[width=\paperwidth,height=\paperheight]{\nightmodebackground}
%     }
% }

\definecolor{pastelblue}{RGB}{96, 145, 245}
\definecolor{pastelgreen}{RGB}{106, 235, 135}
\definecolor{darkgray}{RGB}{60, 60, 60}
\definecolor{lightgray}{RGB}{180, 180, 180}
\definecolor{offwhite}{RGB}{225, 225, 245}


\pagecolor{darkgray}
\color{offwhite}

\newcommand{\Z}{\mathbf{Z}}
\newcommand{\N}{\mathbf{N}}
\newcommand{\R}{\mathbf{R}}
\newcommand{\Q}{\mathbf{Q}}
\newcommand{\C}{\mathbf{C}}

\newcommand{\id}{\mathrm{id}}
\newcommand{\op}{\mathrm{op}}
\newcommand{\diam}{\mathrm{diam}}
\newcommand{\GL}{\mathrm{GL}}
\newcommand{\Tr}{\mathrm{Tr}}
\newcommand{\im}{\mathrm{im}}
\newcommand{\rank}{\mathrm{rank}}

\newcommand{\cl}[1]{\overline{#1}}

\swapnumbers % places numbers before thm names

\theoremstyle{plain} % The "plain" style italicizes all body text.
	\newtheorem{thm}{Theorem}
		\numberwithin{thm}{section} % Theorem numbers are determined by section.
	\newtheorem{lemma}[thm]{Lemma}
	\newtheorem{prop}[thm]{Proposition}
	\newtheorem{cor}[thm]{Corollary}

\theoremstyle{definition}
    \newtheorem{defn}[thm]{Definition}
	\newtheorem{example}[thm]{Example}
	\newtheorem{exercise}[thm]{Exercise} %Exercise

\begin{document}
    \newtcolorbox{question}[2][]{fonttitle=\large, fontupper=\large, fontlower=\large, title=Question {#2}., oversize, arc=3mm, outer arc=2mm, opacityback=0.9, coltitle=offwhite, colframe=pastelblue, colback=darkgray, colupper=lightgray, collower=lightgray, leftrule=1mm, rightrule=1mm, toprule=1.5mm, titlerule=1mm, bottomrule=1mm, valign=center, add to natural height=5mm, lower separated=false, before lower=\begin{proof}, after lower= \smallbreak \end{proof}, #1, breakable=true}

    \begin{question}{36}
        In Handout \#7, we defined differentiability for functions on open sets. Now we give a definition that works over arbitrary sets. \textcolor{blue}{For this problem, you will need to read Piazza Post @274 and use Theorem 1.1.}
    
        Let $A\subseteq \R^n$ be an arbitrary set, let $f:A\rightarrow \R$ be a function, and let $p\in A$ be a point. We say that $f$ is \textbf{differentiable} at $p$ if there exists an open neighborhood $U$ of $p$ and a function $\widehat{f}:U\rightarrow \R$ such that $\widehat{f}$ is differentiable at $p$ (in the sense of Handout \#7) and $\widehat{f}\vert_{U\cap A}=f\vert_{U\cap A}$.

        \begin{enumerate}[label=(\alph*)]
            \item Prove that $f$ is differentiable at every point of $A$ if and only if $f$ extends to a differentiable function defined on an open set containing $A$.

            \item Suppose further that $A$ is closed. Prove that $f$ is differentiable at every point of $A$ if and only if $f$ extends to a differentiable function on $\R^n$.
        \end{enumerate}

        \tcblower
        \ 

        (a):

        Suppose that \(f\) extends to a differentiable function \(\hat{f}\) on an open set \(U \supseteq A\). That is, \(\hat{f}\vert _A = f\). Let \(x \in A\). Since \(U\) is open, we can find an open ball such that \(B(x, \varepsilon) \subseteq U\). Immediately, we get that the function \(\hat{f} \vert _{B(x,\varepsilon)}\) is the desired extension of \(f\) at \(x\), as \(\hat{f}\) is differentiable at \(x\) and \(\hat{f}\vert _{B(x,\varepsilon) \cap A} = f_{B(x,\varepsilon) \cap A}\).

        % \smallbreak

        Conversely, 
    \end{question}
    \newpage
    \begin{question}{37}
        The following set is called the \textbf{$n$-simplex}:
        \[ \Delta_n := \{ \vec{x}=(x_1,\ldots,x_n)\in \R^n : x_1,\ldots,x_n\geq 0 \text{ and } x_1+\cdots+x_n\leq 1\}. \]
        You can assume, without proof, that $\Delta_n$ is Jordan measurable.
        
        Find, with proof, an explicit formula for $\mu(\Delta_n)$ in terms of $n$.

        \tcblower

        First, we show that \(\Delta _n\) is the same as the set
        \[
            S = \left\{ (x_1, ..., x_n) \in \mathbb{R}^n : 0\leq x_1 \leq 1, 0 \leq x_2 \leq 1 - x_1, ..., 0 \leq x_n \leq 1 - \sum_{i=1}^{n-1} x_i \right\}
        \]
        Let \(x \in \Delta _n\). We want to show that \(0 \leq x_i \leq 1 - \sum_{j=1}^{i - 1} x_j\). We get that \(x_i \geq 0\) immediately. As well, since \(\sum_{j=1}^{n} x_j \leq 1\) and every component is non-negative, we have that
        \[
            x_i \leq 1 - \sum_{j=1}^{i - 1} x_j - \sum_{j=i+1}^{n} x_j \leq 1 - \sum_{j=1}^{i - 1} x_j
        \]
        which shows that \(\Delta _n \subseteq S\).

        Now, let \(x \in S\). We know every \(x_i\) is non-negative and additionally
        \[
            x_n \leq 1 - \sum_{i=1}^{n-1} x_i \implies \sum_{i=1}^{n} x_i \leq 1
        \]
        so \(S \subseteq \Delta _n\).

        Now, we proceed to find \(\mu (S) = \mu (\Delta _n)\). Using Fubini's Theorem, we get
        \[
            \mu (S) = \int _S 1 = \int _0^1 \int _0^{1-x_1} \cdots \int _0^{1 - \sum_{i=1}^{n-1} x_i} 1\ dx_n \cdots dx_2\ dx_1
        \]
        Let \(I: \mathbb{N} \times [0,1] \to \mathbb{R}\) be defined recursively as follows:
        % \[
        %     \begin{dcases*}
        %         I(1, \alpha) = \int _0^{1-\alpha} 1\ dt,\\
        %         I(n, \alpha) = \int _0^{1-\alpha} I(n-1, \alpha + t)\ dt, &\text{ for } n > 1 .
        %     \end{dcases*}
        % \]
        \begin{align*}
            I(1, \alpha) &= \int _0^{1-\alpha} 1\ dt &\ \\
            I(k, \alpha) &= \int _0^{1-\alpha} I(k-1, \alpha + t)\ dt, &\text{ for } > 1.
        \end{align*}
        Notice that if we continue applying the definition, we get that
        \[
            I(n, 0) = \mu (S)
        \]
        Now, we will prove using induction on \(n\) that for all \(\alpha \in [0,1]\), \(I(n, \alpha) = \dfrac{1}{n!}(1 - \alpha)^n\).

        Let \(n = 1\). Then
        \[
            I(1, \alpha) = \int _0^{1- \alpha}1\ dt = 1 - \alpha
        \]
        Now, suppose that \(I(k, \alpha) = \dfrac{1}{k!}(1 - \alpha)^k\) holds for all \(\alpha \in [0,1]\) and some \(k \in \mathbb{N}\). We want to show that the same holds for \(k + 1\) as well. For an arbitrary \(\alpha\), we get
        \begin{align*}
            I(k+1, \alpha ) &= \int _0^{1 - \alpha} I(k, \alpha + t)\ dt \\
            &= \int _0^{1-\alpha} \frac{1}{k!}(1 - \alpha - t)^k\ dt \\
            &= -\frac{1}{(k+1)!}(1 - \alpha - t)^{k+1} \Big| _0^{1-\alpha} \\
            &= \frac{1}{(k+1)!}(1 - \alpha)^{k+1}
        \end{align*}
        as desired. Thus we get that
        \[
            \mu (S) = I(n, 0) = \frac{1}{n!}
        \]
    \end{question}
\end{document}