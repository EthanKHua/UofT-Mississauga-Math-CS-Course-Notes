\documentclass{article}
\usepackage[margin=1.0in]{geometry}
\usepackage{amssymb,amsmath,amsthm,amsfonts}
\usepackage{enumitem}
\usepackage{xcolor}
\usepackage{mathtools}
\usepackage{systeme}

% My boxes
\usepackage[breakable]{tcolorbox}

% \RequirePackage{background}
% \backgroundsetup{
%     scale=1,
%     color=black,
%     opacity=1,
%     angle=0,
%     contents={
%         \includegraphics[width=\paperwidth,height=\paperheight]{\nightmodebackground}
%     }
% }

\definecolor{pastelblue}{RGB}{96, 145, 245}
\definecolor{pastelgreen}{RGB}{106, 235, 135}
\definecolor{darkgray}{RGB}{60, 60, 60}
\definecolor{lightgray}{RGB}{180, 180, 180}
\definecolor{offwhite}{RGB}{225, 225, 245}


\pagecolor{darkgray}
\color{offwhite}

\newcommand{\Z}{\mathbf{Z}}
\newcommand{\N}{\mathbf{N}}
\newcommand{\R}{\mathbf{R}}
\newcommand{\Q}{\mathbf{Q}}
\newcommand{\C}{\mathbf{C}}

\newcommand{\id}{\mathrm{id}}
\newcommand{\op}{\mathrm{op}}
\newcommand{\diam}{\mathrm{diam}}
\newcommand{\GL}{\mathrm{GL}}
\newcommand{\Tr}{\mathrm{Tr}}
\newcommand{\im}{\mathrm{im}}
\newcommand{\rank}{\mathrm{rank}}

\newcommand{\cl}[1]{\overline{#1}}

\swapnumbers % places numbers before thm names

\theoremstyle{plain} % The "plain" style italicizes all body text.
	\newtheorem{thm}{Theorem}
		\numberwithin{thm}{section} % Theorem numbers are determined by section.
	\newtheorem{lemma}[thm]{Lemma}
	\newtheorem{prop}[thm]{Proposition}
	\newtheorem{cor}[thm]{Corollary}

\theoremstyle{definition}
    \newtheorem{defn}[thm]{Definition}
	\newtheorem{example}[thm]{Example}
	\newtheorem{exercise}[thm]{Exercise} %Exercise

\begin{document}
    \newtcolorbox{question}[2][]{fonttitle=\large, fontupper=\large, fontlower=\large, title=Question {#2}., oversize, arc=3mm, outer arc=2mm, opacityback=0.9, coltitle=offwhite, colframe=pastelblue, colback=darkgray, colupper=lightgray, collower=lightgray, leftrule=1mm, rightrule=1mm, toprule=1.5mm, titlerule=1mm, bottomrule=1mm, valign=center, add to natural height=5mm, lower separated=false, before lower=\begin{proof}, after lower= \smallbreak \end{proof}, #1, breakable=true}

    \begin{question}{36}
        In Handout \#7, we defined differentiability for functions on open sets. Now we give a definition that works over arbitrary sets. \textcolor{blue}{For this problem, you will need to read Piazza Post @274 and use Theorem 1.1.}
    
        Let $A\subseteq \R^n$ be an arbitrary set, let $f:A\rightarrow \R$ be a function, and let $p\in A$ be a point. We say that $f$ is \textbf{differentiable} at $p$ if there exists an open neighborhood $U$ of $p$ and a function $\widehat{f}:U\rightarrow \R$ such that $\widehat{f}$ is differentiable at $p$ (in the sense of Handout \#7) and $\widehat{f}\vert_{U\cap A}=f\vert_{U\cap A}$.

        \begin{enumerate}[label=(\alph*)]
            \item Prove that $f$ is differentiable at every point of $A$ if and only if $f$ extends to a differentiable function defined on an open set containing $A$.

            \item Suppose further that $A$ is closed. Prove that $f$ is differentiable at every point of $A$ if and only if $f$ extends to a differentiable function on $\R^n$.
        \end{enumerate}
    \end{question}

    \begin{question}{37}
        The following set is called the \textbf{$n$-simplex}:
        \[ \Delta_n := \{ \vec{x}=(x_1,\ldots,x_n)\in \R^n : x_1,\ldots,x_n\geq 0 \text{ and } x_1+\cdots+x_n\leq 1\}. \]
        You can assume, without proof, that $\Delta_n$ is Jordan measurable.
        
        Find, with proof, an explicit formula for $\mu(\Delta_n)$ in terms of $n$.

        \tcblower

        
    \end{question}
\end{document}