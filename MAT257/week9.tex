\documentclass{article}
\usepackage[margin=1.0in]{geometry}
\usepackage{amssymb,amsmath,amsthm,amsfonts,mathtools}
\usepackage{enumitem}
\usepackage{xcolor}

\newcommand{\Z}{\mathbf{Z}}
\newcommand{\N}{\mathbf{N}}
\newcommand{\R}{\mathbf{R}}
\newcommand{\Q}{\mathbf{Q}}
\newcommand{\C}{\mathbf{C}}

\newcommand{\id}{\mathrm{id}}
\newcommand{\op}{\mathrm{op}}
\newcommand{\diam}{\mathrm{diam}}
\newcommand{\Tr}{\mathrm{Tr}}

\newcommand{\cl}[1]{\overline{#1}}

\swapnumbers % places numbers before thm names

\theoremstyle{plain} % The "plain" style italicizes all body text.
	\newtheorem{thm}{Theorem}
		\numberwithin{thm}{section} % Theorem numbers are determined by section.
	\newtheorem{lemma}[thm]{Lemma}
	\newtheorem{prop}[thm]{Proposition}
	\newtheorem{cor}[thm]{Corollary}

\theoremstyle{definition}
    \newtheorem{defn}[thm]{Definition}
	\newtheorem{example}[thm]{Example}
	\newtheorem{exercise}[thm]{Exercise} %Exercise

\begin{document}
    \section*{Chain Rule!!}
    \textbf{Solvers:} Ali, Ethan, Ryan

    \noindent\textbf{Writeup:} Ethan

    Let \(X,Y,Z\) be normed vector spaces, and \(g: X \to Y\), \(f: Y \to Z\) be functions. For some \(p \in X\), suppose that \(g\) is totally differentiable at \(p\) and \(f\) is totally differentiable at \(g(p)\). Then the function \(f \circ g\) is totally differentiable at \(p\), and \((f \circ g)' = f'(g(p)) \circ g'(p)\).

	\begin{proof}
		Let \(L_f = f'(g(p))\) and \(L_g = g'(p)\). First, let's prove that \(L_f \circ L_g\) is a linear bounded operator. For \(a,b \in X\), \(c \in \mathbb{R}\),
		\[
			L_f(L_g(ca + b)) = L_f(cL_g(a) + L_g(b)) = cL_f(L_g(a)) + L_f(L_g(b))
		\]
		Thus \(L_f \circ L_g\) is linear. Now, since \(L_f, L_g\) are bounded linear operators, there exist constants \(M_f, M_g > 0\) such that for all \(x \in X\), \(y \in Y\),
		\[
			\|L_f(y)\| _Z \leq M_f \|y\| _Y \text{ and } \|L_g(x)\| _Y \leq M_g \|x\| _X
		\]
		Then we have that for all \(x \in X\),
		\[
			\|L_f(L_g(x))\| _Z \leq M_f\|L_g(x)\| _Y \leq M_f \cdot M_g \|x\| _X
		\]
		so it is bounded as well. Now we can move on to the long part of the proof.

		\textcolor{red}{Consider the case if \(g(p+t) - g(p) = 0\) for some \(t \in X\) in every open neighbourhood around \(0\). Then it is probably true that \(g(x)\) is constant when \(x\) is sufficiently close to \(p\)}. We can quickly verify that \(L_g = 0\) because
		\[
			\lim_{h \to 0} \frac{\|g(p+h) - g(p)\| _Y}{\|h\| _X} = \lim_{h \to 0} 0 = 0
		\]
		From this, we can see that
		\[
			\lim_{h \to 0} \frac{\|f(g(p+h)) - f(g(p))\| _Z}{\|h\| _X} = 0 \text{,} 
		\]
		so \((f \circ g)' = 0 = L_f(L_g)\).

		Otherwise, there is some open ball \(B(0, r)\) such that \(g(p + h) - g(p) \neq 0\) for all \(h \in B(0, r)\).

		Let \(L_f = f'(g(p))\) and \(L_g = g'(p)\). Since these are bounded linear operators, there exist constants \(M_f, M_g > 0\) such that for all \(x \in X\), \(y \in Y\),
		\[
			\|L_f(y)\| _Z \leq M_f \|y\| _Y \text{ and } \|L_g(x)\| _Y \leq M_g \|x\| _X
		\]
		Let \(\varepsilon > 0\). By our assumption, there exists positive numbers \(\delta _f , \delta _g\) such that for \(h_f, h_g\) with \(\|h_f\| _X < \delta _f\), \(\|h_g\| _X < \delta _g\),
		\[
			\frac{\|f(g(p) + h_f) - f(g(p)) - L_f(h_f)\| _Z}{\|h_f\| _X} < \min \left\{\sqrt{\varepsilon}, \frac{\varepsilon}{4M_g}\right\} \tag{1}
		\]
		and
		\[
			\frac{\|g(p+h_g) - g(p) - L_g(h_g)\| _Y}{\|h_g\| _X} < \min \left\{\frac{\sqrt{\varepsilon}}{4}, \frac{\varepsilon}{2M_f}\right\} \tag{2}
		\]
		As well, since \(g\) being totally differentiable at \(p\) implies continuity at \(p\), then for all \(\|h_c\| _X < \delta _c\),
		\[
			\|g(p + h_c) - g(p)\| < \delta _f \tag{3}
		\]
		Define \(\varepsilon _f = \min \left\{\sqrt{\varepsilon}, \frac{\varepsilon}{4M_g}\right\}\) and \(\varepsilon _g = \min \left\{\frac{\sqrt{\varepsilon}}{4}, \frac{\varepsilon}{2M_f}\right\}\), and let \(\delta = \min \{\delta_g, \delta_c, r\}\). For \(h \in X\) with \(\|h\| _X < \delta\),
		\[
			\frac{\|f(g(p+h)) - f(g(p)) - L_f(L_g(h))\| _Z}{\|h\| _X}
		\]	
		\[
			\leq \frac{\|f(g(p+h)) - f(g(p)) - L_f(g(p + h) - g(p))\| _Z + \|L_f(g(p+h) - g(p) - L_g(h))\| _Z}{\|h\| _X}
		\]
		Consider the first term. Because of (3), we can apply (1) with \(h_f = g(p + h) - g(p)\) and get that
		\[
			\frac{\|f(g(p+h)) - f(g(p)) - L_f(g(p + h) - g(p))\| _Z}{\|h\| _X} < \frac{\|g(p+h) - g(p)\|}{\|h\| _X}\varepsilon _f
		\]
		Then applying (2) with \(h_g = h\),
		\[
			\leq \frac{\|g(p+h) - g(p) - L_g(h)\| _Y + \|L_g(h)\| _Y}{\|h\| _X}\varepsilon _f \leq \frac{\sqrt{\varepsilon}}{4}\varepsilon _f + M_g \varepsilon _f \leq \frac{\sqrt{\varepsilon}}{4}\sqrt{\varepsilon} + M_g \frac{\varepsilon}{4 M_g}
		\]
		\[
			= \frac{\varepsilon}{2}
		\]
		Now, we examine the second term. Using the fact that \(L_f\) is a bounded linear operator and (2),
		\[
			\frac{\|L_f(g(p+h) - g(p) - L_g(h))\| _Z}{\|h\| _X} \leq \frac{M_f \|g(p+h) - g(p) - L_g(h)\| _Y}{\|h\| _X} < \frac{M_f \varepsilon}{2M_f} = \frac{\varepsilon}{2}
		\]
		Adding the two together we conclude that
		\[
			\frac{\|f(g(p+h)) - f(g(p)) - L_f(g(p + h) - g(p))\| _Z + \|L_f(g(p+h) - g(p) - L_g(h))\| _Z}{\|h\| _X}
		\] 
		\[
			< \frac{\varepsilon}{2} + \frac{\varepsilon}{2} = \varepsilon
		\]
		and we are done.

	\end{proof}
\end{document}