\documentclass{article}
\usepackage[margin=1.0in]{geometry}
\usepackage{amssymb,amsmath,amsthm,amsfonts,mathtools}
\usepackage{enumitem}
\usepackage{xcolor}

\newcommand{\Z}{\mathbf{Z}}
\newcommand{\N}{\mathbf{N}}
\newcommand{\R}{\mathbf{R}}
\newcommand{\Q}{\mathbf{Q}}
\newcommand{\C}{\mathbf{C}}

\newcommand{\id}{\mathrm{id}}
\newcommand{\op}{\mathrm{op}}
\newcommand{\diam}{\mathrm{diam}}
\newcommand{\Tr}{\mathrm{Tr}}

\newcommand{\cl}[1]{\overline{#1}}

\swapnumbers % places numbers before thm names

\theoremstyle{plain} % The "plain" style italicizes all body text.
	\newtheorem{thm}{Theorem}
		\numberwithin{thm}{section} % Theorem numbers are determined by section.
	\newtheorem{lemma}[thm]{Lemma}
	\newtheorem{prop}[thm]{Proposition}
	\newtheorem{cor}[thm]{Corollary}

\theoremstyle{definition}
    \newtheorem{defn}[thm]{Definition}
	\newtheorem{example}[thm]{Example}
	\newtheorem{exercise}[thm]{Exercise} %Exercise

\begin{document}
    \section*{Exercise 9.10}
    \textbf{Solvers:} Ethan

    \noindent\textbf{Writeup:} Ethan

    Suppose that $D=\mathrm{diag}(\lambda_1,\ldots,\lambda_n)$ is a diagonal matrix, and let $Q_D$ be the corresponding quadratic form. \textit{$Q_D$ achieves a minimum at the origin if and only if \(\lambda _i \geq 0\) for all \(i \in \{1, ..., n\}\).}

    \begin{proof}
        First note that \(Q_D(x) = x^T D x = \begin{pmatrix}
            x_1 & \cdots &  x_n \\
        \end{pmatrix}
        \begin{pmatrix}
            \lambda _1 & 0 & \cdots & 0 \\
            0 & \lambda_2 & \cdots & 0 \\
            \vdots & \vdots & \ddots & \vdots  \\
            0 & 0 & \cdots & \lambda_n
        \end{pmatrix}
        \begin{pmatrix}
             x_1 \\
             \vdots \\
             x_n \\
        \end{pmatrix}
        = \sum\limits_{i=1}^n \lambda_i x_i^2\).
        
        Suppose that every \(\lambda _i \geq 0\). It follows that \(\lambda _i x_i^2 \geq 0\). Thus
        \[
            Q_D(x) = \sum_{i=1}^n \lambda_i x_i^2 \geq 0
        \]
        Since \(Q_D(0) = 0\), \(Q_D\) achieves a minimum at the origin.

        Conversely, suppose that \(\lambda_j < 0\), for some \(j \in \{1, ..., n\}\). \(Q_D(0) = 0\), but \(Q_D(e_j) = \lambda_j < 0\), so the origin is not a minimum.
        \smallbreak
    \end{proof}
\end{document}