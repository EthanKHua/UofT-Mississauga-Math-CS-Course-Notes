\documentclass{article}
\usepackage[margin=1.0in]{geometry}
\usepackage{amssymb,amsmath,amsthm,amsfonts}
\usepackage{enumitem}
\usepackage{xcolor}
\usepackage{mathtools}

\newcommand{\Z}{\mathbf{Z}}
\newcommand{\N}{\mathbf{N}}
\newcommand{\R}{\mathbf{R}}
\newcommand{\Q}{\mathbf{Q}}
\newcommand{\C}{\mathbf{C}}

\newcommand{\id}{\mathrm{id}}
\newcommand{\op}{\mathrm{op}}
\newcommand{\diam}{\mathrm{diam}}
\newcommand{\Tr}{\mathrm{Tr}}
\newcommand{\im}{\mathrm{im}}
\newcommand{\rank}{\mathrm{rank}}

\newcommand{\cl}[1]{\overline{#1}}

\swapnumbers % places numbers before thm names

\theoremstyle{plain} % The "plain" style italicizes all body text.
	\newtheorem{thm}{Theorem}
		\numberwithin{thm}{section} % Theorem numbers are determined by section.
	\newtheorem{lemma}[thm]{Lemma}
	\newtheorem{prop}[thm]{Proposition}
	\newtheorem{cor}[thm]{Corollary}

\theoremstyle{definition}
    \newtheorem{defn}[thm]{Definition}
	\newtheorem{example}[thm]{Example}
	\newtheorem{exercise}[thm]{Exercise} %Exercise

\begin{document}
    \section{Seperablility Stuff}
    \begin{thm}
        Let \((X, d)\) be a seperable metric space. Then \(A \subseteq X\) is seperable with respect to the metric \(d\).
    \end{thm}
    \begin{proof}
        Since \(X\) is seperable, there exists a countable dense set \(D \subseteq X\).

        For any \(a_i \in A\) and any \(k \in \mathbb{N}\), there exists an element \(d_{ik} \in D\) such that \(d_{ik} \in B(a_i, \frac{1}{k})\).

        Consider the set
        \[
            D^\prime = \bigcup_{i \in I} \left(\bigcup_{k \in \mathbb{N}} d_{ik}\right)
        \]
        Notice that \(D^\prime \subseteq D\) so it is at most countable.

        For all \(a \in A\), \(\varepsilon > 0\), consider the open ball \(B(a, \varepsilon)\). By the Archimedian property, there exists \(k \in \mathbb{N}\) such that \(k > \frac{1}{\varepsilon} \implies \frac{1}{k} < \varepsilon\). It follows that \(B(a, \frac{1}{k}) \subseteq B(a, \varepsilon)\), and we know we can find a \(d_{ik} \in D^\prime\) within this open ball. Thus \(D^\prime\) is dense in \(A\).

        Since \(D^\prime\) is dense and at most countable, \(A\) is seperable.

    \end{proof}

    \begin{thm}
        Let \(X,Y\) be metric spaces let and \(f \colon C \to X\) be a continuous function. If \(A \subseteq X\) is seperable, then \(f(A)\) is seperable.
    \end{thm}
    \begin{proof}
        Define \(d_X\) and \(d_Y\) to be metrics on \(X\) and \(Y\) respectively. Suppose that \(A\) is seperable. We will show that \(f(A)\) is seperable by equivalently showing that it is pre-totally bounded.

        Let \(\varepsilon > 0\). By the continuity of \(f\), for every \(a \in A\), there exists \(\delta > 0\) such that for all \(x \in A\), \(d_X(x, a) < \delta \implies d_Y(f(x), f(a)) < \varepsilon\). Keep this value of \(\delta\).

        Since \(A\) is seperable, by Theorem y.x \textcolor{red}{(replace this with the thm number idk what it is)}, \(A\) is pre-totally bounded. By definition, \(A\) is covered by a countable subcover \(\{B_X(a_i, \delta)\}_{i\in\mathbb{N}}\).

        Consider the countable collection \(\{B_Y(f(a_i), \varepsilon)\}_{i\in\mathbb{N}}\). We will show that this collection covers \(f(A)\). Let \(y \in f(A)\). Then \(y = f(x)\) for some \(x \in A \subseteq \{B_X(a_i, \delta)\}_{i\in\mathbb{N}}\). This implies that \(x\) is an some open ball \(B_X(a_k, \delta) \implies d_X(x, a_k) < \delta\). By the continuity of \(f\), this implies that \(d_Y(y, f(a_k)) = d_Y(f(x), f(a_k)) < \varepsilon \implies y \in B_Y(f(a_k), \varepsilon) \subseteq \{B_Y(f(a_i), \varepsilon)\}_{i\in\mathbb{N}}\).

        We see that \(f(A)\) is pre-totally bounded, which implies by \textcolor{red}{thm 29384qwurjhq} that \(f(A)\) is seperable.
        
    \end{proof}
\end{document}