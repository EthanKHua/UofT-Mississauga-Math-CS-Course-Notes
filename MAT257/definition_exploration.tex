\documentclass{article}
\usepackage[margin=1.0in]{geometry}
\usepackage{amssymb,amsmath,amsthm,amsfonts}
\usepackage{enumitem}
\usepackage{xcolor}
\usepackage{mathtools}

\newcommand{\Z}{\mathbf{Z}}
\newcommand{\N}{\mathbf{N}}
\newcommand{\R}{\mathbf{R}}
\newcommand{\Q}{\mathbf{Q}}
\newcommand{\C}{\mathbf{C}}

\newcommand{\id}{\mathrm{id}}
\newcommand{\op}{\mathrm{op}}
\newcommand{\diam}{\mathrm{diam}}
\newcommand{\Tr}{\mathrm{Tr}}
\newcommand{\im}{\mathrm{im}}
\newcommand{\rank}{\mathrm{rank}}

\newcommand{\cl}[1]{\overline{#1}}

\swapnumbers % places numbers before thm names

\theoremstyle{plain} % The "plain" style italicizes all body text.
	\newtheorem{thm}{Theorem}
		\numberwithin{thm}{section} % Theorem numbers are determined by section.
	\newtheorem{lemma}[thm]{Lemma}
	\newtheorem{prop}[thm]{Proposition}
	\newtheorem{cor}[thm]{Corollary}

\theoremstyle{definition}
    \newtheorem{defn}[thm]{Definition}
	\newtheorem{example}[thm]{Example}
	\newtheorem{exercise}[thm]{Exercise} %Exercise

\begin{document}
    \section{Seperablility Stuff}
    \begin{thm}
        Let \((X, d)\) be a separable metric space. Then \(A \subseteq X\) is separable with respect to the metric \(d\).
    \end{thm}
    \begin{proof}
        Since \(X\) is separable, it is pre-totally bounded. We will show that a metric subspace \(A\) is separable by showing that it is pre-totally bounded.

        Let \(\varepsilon > 0\). Since \(X\) is pre-totally bounded, there exists a finite collection of open balls \(\{ B(x_i, \frac{\varepsilon}{2})\}_{i\leq n}\) that covers \(X\). We define the desired collection of open balls in \(A\) with the following method:

        If \(B(x_i, \frac{\varepsilon}{2})\) contains some element \(a_i \in A\), we add \(B_A(a_i, \varepsilon)\) to the collection.

        The number of open balls in this collection is at most the number of open balls in the original collection in \(X\), so \(\{B_A(a_i, \varepsilon)\}\) is finite.

        It remains to show that the collection covers \(A\). Let \(a \in A\). Then \(a \in B(x_i, \frac{\varepsilon}{2})\) for some \(i\). It follows that \(a \in B(x_i,\frac{\varepsilon}{2})\cap A \subseteq B_A(a_i, \varepsilon)\). Thus \(A\) is covered by this collection, so \(A\) is pre-totally bounded.

        Thus we can conclude that \(A\) is separable.

    \end{proof}

    \begin{thm}
        Let \(X,Y\) be metric spaces let and \(f \colon C \to X\) be a continuous function. If \(A \subseteq X\) is seperable, then \(f(A)\) is seperable.
    \end{thm}
    \begin{proof}
        Define \(d_X\) and \(d_Y\) to be metrics on \(X\) and \(Y\) respectively. Suppose that \(A\) is seperable. We will show that \(f(A)\) is seperable by equivalently showing that it is pre-totally bounded.

        Let \(\varepsilon > 0\). By the continuity of \(f\), for every \(a \in A\), there exists \(\delta > 0\) such that for all \(x \in A\), \(d_X(x, a) < \delta \implies d_Y(f(x), f(a)) < \varepsilon\). Keep this value of \(\delta\).

        Since \(A\) is seperable, by Theorem y.x \textcolor{red}{(replace this with the thm number idk what it is)}, \(A\) is pre-totally bounded. By definition, \(A\) is covered by a countable subcover \(\{B_X(a_i, \delta)\}_{i\in\mathbb{N}}\).

        Consider the countable collection \(\{B_Y(f(a_i), \varepsilon)\}_{i\in\mathbb{N}}\). We will show that this collection covers \(f(A)\). Let \(y \in f(A)\). Then \(y = f(x)\) for some \(x \in A \subseteq \{B_X(a_i, \delta)\}_{i\in\mathbb{N}}\). This implies that \(x\) is an some open ball \(B_X(a_k, \delta) \implies d_X(x, a_k) < \delta\). By the continuity of \(f\), this implies that \(d_Y(y, f(a_k)) = d_Y(f(x), f(a_k)) < \varepsilon \implies y \in B_Y(f(a_k), \varepsilon) \subseteq \{B_Y(f(a_i), \varepsilon)\}_{i\in\mathbb{N}}\).

        We see that \(f(A)\) is pre-totally bounded, which implies by \textcolor{red}{thm 29384qwurjhq} that \(f(A)\) is seperable.
        
    \end{proof}
\end{document}