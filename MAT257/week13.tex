\documentclass{article}
\usepackage[margin=1.0in]{geometry}
\usepackage{amssymb,amsmath,amsthm,amsfonts,mathtools}
\usepackage{enumitem}
\usepackage{xcolor}

\newcommand{\Z}{\mathbf{Z}}
\newcommand{\N}{\mathbf{N}}
\newcommand{\R}{\mathbf{R}}
\newcommand{\Q}{\mathbf{Q}}
\newcommand{\C}{\mathbf{C}}

\newcommand{\id}{\mathrm{id}}
\newcommand{\op}{\mathrm{op}}
\newcommand{\diam}{\mathrm{diam}}
\newcommand{\GL}{\mathrm{GL}}
\newcommand{\Tr}{\mathrm{Tr}}

\newcommand{\cl}[1]{\overline{#1}}

\swapnumbers % places numbers before thm names

\theoremstyle{plain} % The "plain" style italicizes all body text.
	\newtheorem{thm}{Theorem}
		\numberwithin{thm}{section} % Theorem numbers are determined by section.
	\newtheorem{lemma}[thm]{Lemma}
	\newtheorem{prop}[thm]{Proposition}
	\newtheorem{cor}[thm]{Corollary}

\theoremstyle{definition}
    \newtheorem{defn}[thm]{Definition}
	\newtheorem{example}[thm]{Example}
	\newtheorem{exercise}[thm]{Exercise} %Exercise

\begin{document}
    \section*{Exercise 12.8}
    \textbf{Solvers:} Ethan, Ali, Mitya

    \noindent\textbf{Writeup:} Ethan

    \noindent\textbf{Lemma 12.4.} Let \(P\) be a polybox in \(\mathbb{R}^n\). Then
    \begin{enumerate}
        \item \(P\) is a finite disjoint union of boxes.
        \item If \(\{ E_1, ..., E_k \} \) and \(\{ F_1, ..., F_m \}\) are two pairwise disjoint collections of boxes, and \(\bigcup_{i=1}^{k} E_i = \bigcup_{j=1}^{m} F_j\), then
        \[
            \sum_{i=1}^{k} \mathrm{vol} (E_i) = \sum_{j=1}^{m} \mathrm{vol} (F_j)
        \]
    \end{enumerate}
    \begin{proof}
        We start with a proof of part (a). Since \(P\) is a polybox, we can write it as a finite union of boxes not necessarily disjoint. We have
        \[
            P = \bigcup_{i=1}^{p} P_i, \text{ where } P_i \text{ is a box in } \mathbb{R}^n
        \]
        We proceed using induction on \(p\) to prove (a). When \(p = 1\), \(P\) itself is a box and is considered a finite union of disjoint boxes.

        Suppose that the claim holds for some \(p - 1\). We can rewrite \(P\) as
        \[
            P = P_p \cup \bigcup_{i=1}^{p-1} P_i
        \]
        From our assumption, the polybox \(\bigcup_{i=1}^{p-1} P_i\) can be expressed as a finite union of disjoint boxes, which we will denote as \(B_i\), so we have
        \[
            P = P_p \cup \bigcup_{i=1}^{q} B_i = \bigcup_{i=1}^{q} (B_i \cup P_p)
        \]
        Consider each pair of boxes \(B_i\) and \(P_p\). By Lemma 12.2,
        \begin{enumerate}
            \item \(P_p \cap B_i\) is a box
            \item \(P_p \setminus B_i\) and \(B_i \setminus P_p\) are finite unions of disjoint boxes.
        \end{enumerate}
        Additionally, all these sets are disjoint from each other. We can take the union of all the sets above and see that \((P_p \cap B_i) \cup (P_p \setminus B_i) \cup (B_i \setminus P_p) = B_i \cup P_p\) is also a finite union of disjoint boxes. Thus \(P\) is a finite union of disjoint boxes.

        By induction, part (a) has been proven and we move on to prove part (b).

        \medbreak

        \noindent Let \(\{ E_1, ..., E_k \}, \{F_1, ..., F_m \}\) be pairwise disjoint collections of boxes such that
        \[
            P = \bigcup_{i=1}^{k} E_i = \bigcup_{j=1}^{m} F_j
        \]
        For \(i \in \{ 1, ..., k \}, j \in \{1, ..., m\}\), define \(A_{ij} = E_i \cap F_j\). We will show that \(A \coloneqq \bigcup_{i=1}^{k} \bigcup_{j=1}^{m} A_{ij} = P\). It is clear that \(A \subseteq P\), so only \(P \subseteq A\) will be shown.

        Let \(x \in P\). Then \(x\) must belong to some boxes \(E_i\) and \(F_j\), so \(x \in A_{ij} \subseteq A\). This shows that \(A\) is also a valid disjoint decomposition of \(P\).

        Finally, notice that for all \(E_i, F_j\),
        \[
            E_i = \bigcup_{j=1}^{m} A_{ij} \text{ and } F_j = \bigcup_{i=1}^{k} A_{ij}
        \]
        By part (b) of Lemma 12.2, we have
        \[
            \mathrm{vol} (E_i) = \sum_{j=1}^m \mathrm{vol} (A_{ij}) \text{ and } \mathrm{vol} (F_j) = \sum_{i=1}^{k} \mathrm{vol} (A_{ij})
        \]
        Therefore
        \[
            \sum_{i=1}^{k} \mathrm{vol} (E_i) = \sum_{i=1}^{k} \sum_{j=1}^{m} \mathrm{vol} (A_{ij}) = \sum_{j=1}^{m} \sum_{i=1}^{k} \mathrm{vol} (A_{ij}) = \sum_{j=1}^{m} \mathrm{vol} (F_j)
        \]
        what was what we wanted.
        \smallbreak
    \end{proof}
\end{document}