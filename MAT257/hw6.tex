\documentclass{article}
\usepackage[margin=1.0in]{geometry}
\usepackage{amssymb,amsmath,amsthm,amsfonts}
\usepackage{enumitem}
\usepackage{xcolor}
\usepackage{mathtools}

% My boxes
\usepackage[breakable]{tcolorbox}

% \RequirePackage{background}
% \backgroundsetup{
%     scale=1,
%     color=black,
%     opacity=1,
%     angle=0,
%     contents={
%         \includegraphics[width=\paperwidth,height=\paperheight]{\nightmodebackground}
%     }
% }

\definecolor{pastelblue}{RGB}{96, 145, 245}
\definecolor{pastelgreen}{RGB}{106, 235, 135}
\definecolor{darkgray}{RGB}{60, 60, 60}
\definecolor{lightgray}{RGB}{180, 180, 180}
\definecolor{offwhite}{RGB}{225, 225, 245}


\pagecolor{darkgray}
\color{offwhite}

\newcommand{\Z}{\mathbf{Z}}
\newcommand{\N}{\mathbf{N}}
\newcommand{\R}{\mathbf{R}}
\newcommand{\Q}{\mathbf{Q}}
\newcommand{\C}{\mathbf{C}}

\newcommand{\id}{\mathrm{id}}
\newcommand{\op}{\mathrm{op}}
\newcommand{\diam}{\mathrm{diam}}
\newcommand{\GL}{\mathrm{GL}}
\newcommand{\Tr}{\mathrm{Tr}}
\newcommand{\im}{\mathrm{im}}
\newcommand{\rank}{\mathrm{rank}}

\newcommand{\cl}[1]{\overline{#1}}

\swapnumbers % places numbers before thm names

\theoremstyle{plain} % The "plain" style italicizes all body text.
	\newtheorem{thm}{Theorem}
		\numberwithin{thm}{section} % Theorem numbers are determined by section.
	\newtheorem{lemma}[thm]{Lemma}
	\newtheorem{prop}[thm]{Proposition}
	\newtheorem{cor}[thm]{Corollary}

\theoremstyle{definition}
    \newtheorem{defn}[thm]{Definition}
	\newtheorem{example}[thm]{Example}
	\newtheorem{exercise}[thm]{Exercise} %Exercise

\begin{document}
    \newtcolorbox{question}[2][]{fonttitle=\large, fontupper=\large, fontlower=\large, title=Question {#2}., oversize, arc=3mm, outer arc=2mm, opacityback=0.9, coltitle=offwhite, colframe=pastelblue, colback=darkgray, colupper=lightgray, collower=lightgray, leftrule=1mm, rightrule=1mm, toprule=1.5mm, titlerule=1mm, bottomrule=1mm, valign=center, add to natural height=5mm, lower separated=false, before lower=\begin{proof}, after lower= \\ \end{proof}, #1, breakable=true}
    \setcounter{section}{5}
    \section{HOMEWORK 6 HAAHHAHAHAHAHAA}
    \begin{question}{19}
        Let $X$ be a metric space and let $A\subseteq X$. A \textbf{compact exhaustion} for $A$ is a sequence of compact sets $K_1,K_2,K_3,\ldots$ such that $U=\bigcup_{i\geq 1}K_i$ and $K_i\subseteq K_{i+1}^\circ$.
        \begin{enumerate}[label=(\alph*)]
            \item Let $U\subseteq \R^n$ be a bounded open set. Show that $U$ has a compact exhaustion.
            \item Now show that every open set $U\subseteq \R^n$ has a compact exhaustion.
        \end{enumerate}
        \tcblower
        (a):
        
        Let \(U \subseteq \mathbb{R}^n\) be a bounded open set. Let \(N\) be smallest natural number such that the set \(S = \{x \in U : \mathrm{dist}(\{x\}, U^c) \geq \frac{1}{N}\}\) is non-empty. Define the sequence of sets \((S_n)_{n\geq 1}\) by
        \[
            S_n = \left\{x \in U : \mathrm{dist}(\{x\}, U^c) \geq \frac{1}{n+N}\right\}
        \]
        We will show that \(S_n\) is closed. Let \(s\) be limit point of \(S_n\). Then for \(y \in U^c\),
        \[
            d(s,y) \geq d(x,y) - d(x,s) \implies d(s,y) \geq \sup _{x \in S_n} \left(d(x,y) - d(x,s)\right)
        \]
        We will show that \(\sup _{x \in S_n} \left(d(x,y) - d(x,s)\right) = \frac{1}{N+n}\). For all \(\varepsilon > 0\), there is an \(x \in S_n\) such that \(d(x,s) < \varepsilon\). Thus
        \[
            d(x,y) - d(x,s) > \frac{1}{N+n} - \varepsilon
        \]
        which means that \(\sup _{x \in S_n} \left(d(x,y) - d(x,s)\right) = \frac{1}{N+n}\). Therefore we have that
        \[
            d(s,y) \geq \frac{1}{N+n} \implies s \in S_n
        \]
        so it follows that \(S_n\) is closed. Since \(S_n \subseteq U\), it is also bounded, so because we are working in \(\mathbb{R}^n\), \(S_n\) is compact.

        As well, we need to have that \(S_n \subseteq S_{n+1}^{\circ}\). This result is quite quick, as we can notice that
        \[
            S_{n+1}^\circ = \left\{x \in U : \mathrm{dist}(\{x\}, U^c) > \frac{1}{N+n}\right\}
        \]
        From this, the inclusion follows nicely.

        Then, as \(n\) tends to infinity, \(S_n = \left\{x \in U : \mathrm{dist}(\{x\}, U^c) > 0\right\}\) which is exactly \((U^c)^c = U\). Therefore \(U\) has compact exhaustion.

        (b):

        Now, let \(U \subseteq \mathbb{R}^n\) be open. Define the sequence of bounded open sets \(A_n = U \cap B(\vec{0},  n)\). As \(n\) tends to infinity, \(\bigcup_{n=1}^{\infty} A_n = U\). By the previous part, \(A_n\) has a compact exhaustion \((K_{nk})_{k\geq 1}\). Take the sequence \((K_k)_{k\geq 1}\) to be \(K_k = \bigcup_{n=1}^{\infty} K_{nk}\). This sequence of sets satisfies the conditions for a compact exhaustion. Now we attempt to prove that the sequence converges to \(U\). We have
        \[
            \bigcup_{k=1}^{\infty} K_k = \bigcup_{k=1}^{\infty} \bigcup_{n=1}^{\infty} K_{nk}
        \]
        We swap the order of union and get that
        \[
            \bigcup_{k=1}^{\infty} \bigcup_{n=1}^{\infty} K_{nk} = \bigcup_{n=1}^{\infty} \bigcup_{k=1}^{\infty} K_{nk} = \bigcup_{n=1}^{\infty} A_n = U
        \]
        Thus every open subset of \(\mathbb{R}^n\) has a compact exhaustion.
    \end{question}
    \pagebreak
    \begin{question}{20}
        Let $x,y\in \ell^\infty$ be two sequences. Let us say that $y$ is \textbf{dominated} by $x$, denoted $x\geq y$, if $|x_n|\geq |y_n|$ for all $n\in \N$. Let $D_x$ denote the set of all sequences which are dominated by $x$:
        \[ D_x = \{y\in \ell^\infty: |y_n|\leq |x_n| \text{ for all $n\in \N$}\}. \]
        Prove that $D_x$ is compact if and only if $x_n\rightarrow 0$.

        \tcblower

        Suppose that \(D_x\) is compact. Suppose for contradiction that \(x_n \nrightarrow 0\). For some \(\varepsilon > 0\), \(|x_{N_k}| \geq \varepsilon\) for an infinite number of \(N_k\). Consider the open cover \(\{B(\vec{y}_i, \frac{\varepsilon}{2})\}_{i\in I}\), which is the collection of \(\frac{\varepsilon}{2}\)-balls centered around every \(\vec{y}_i \in D_x\). By compactness of \(D_x\), there is a finite subcover \(\{B(\vec{y}_i, \frac{\varepsilon}{2})\}_{i \leq m}\). Now, we construct a \(y \in D_x\) as follows:

        For every sequence \(\vec{y}_i\), let
        \[
            y_{N_i} = \begin{dcases}
                \varepsilon, &\text{ if } (\vec{y}_i)_{N_i} < \frac{\varepsilon}{2};\\
                0, &\text{ if } (\vec{y}_i)_{N_i} \geq \frac{\varepsilon}{2} ;\\
            \end{dcases}
        \]
        For all other terms in \(y\), make it 0. Notice that for all \(B(\vec{y}_i, \frac{\varepsilon}{2})\),
        \[
            \|y-\vec{y}_i\|_\infty \geq |y_{N_i} - (\vec{y}_i)_{N_i}| \geq \frac{\varepsilon}{2} \implies y \notin D_x
        \]
        which is a contradiction.

        Conversely, suppose that \(x_n \rightarrow 0\). We will show that \(D_x\) is complete and totally bounded, which is equivalent to compactness.

        To show completeness, notice that the ambient space \(\ell ^{\infty}\) is complete. Thus if we can show that \(D_x\) is closed, it will follow that \(D_x\) is complete.

        Let \(a \notin D_x\). We will show that \(a\) is not a limit point of \(D_x\), which means that \(D_x\) is closed. We know that there is a \(k \in \mathbb{N}\) such that \(|a_k| > |x_k|\). Define \(\varepsilon = |a_k| - |x_k|\). Fix \(y \in D_x\). Then
        \[
            \|a_k - y_k\| _{\infty} \geq |a_k - y_k| \geq |a_k| - |y_k| \geq |a_k| - |x_k| = \varepsilon
        \]
        which implies that \(y \notin B(a_k, \varepsilon)\), so \(a\) is not a limit point of \(D_x\). Thus \(D_x\) is closed. It follows that \(D_x\) is complete.

        To show that \(D_x\) is totally bounded, first let \(\varepsilon > 0\). Since \(x_n \to 0\), there is a large enough \(N\) such that for \(n >N\), \(|x_n| < \frac{\varepsilon}{2}\). For all \(y \in D_x\), \(|y_n| < |x_n| < \frac{\varepsilon}{2}\). Consider the set of elements in \(D_x\) such that their terms are 0 for \(n > N\). This set is totally bounded, so there is a \(\varepsilon\)-ball cover \(\{B(y_i, \varepsilon)\}_{i \leq n}\). We show that this collection also covers \(D_x\).

        For \(y \in D_x\), there is an open ball \(B(y_i, \varepsilon)\) such that \(\sup_{n\leq N} {|y_n - (y_i)_n|} < \varepsilon\). But also notice that for \(n>N\), \(|(y_i)_n - y_n| \leq |(y_i)_n| + |y_n| < |y_n| < \frac{\varepsilon}{2}\). Thus \(\|y-y_i\| _{\infty} < \varepsilon\) so \(y \in B(y_i, \varepsilon)\). Thus \(D_x\) is totally bounded.

        Since \(D_x\) is both complete and totally bounded, we can conclude that \(D_x\) is compact.
    \end{question}
    % \question{20}
    % \end{proof}
\end{document}