\documentclass{article}
\usepackage[margin=1.0in]{geometry}
\usepackage{amssymb,amsmath,amsthm,amsfonts}
\usepackage{enumitem}
\usepackage{xcolor}
\usepackage{mathtools}

% My boxes
\usepackage[breakable]{tcolorbox}

% \RequirePackage{background}
% \backgroundsetup{
%     scale=1,
%     color=black,
%     opacity=1,
%     angle=0,
%     contents={
%         \includegraphics[width=\paperwidth,height=\paperheight]{\nightmodebackground}
%     }
% }

\definecolor{pastelblue}{RGB}{96, 145, 245}
\definecolor{pastelgreen}{RGB}{106, 235, 135}
\definecolor{darkgray}{RGB}{60, 60, 60}
\definecolor{lightgray}{RGB}{180, 180, 180}
\definecolor{offwhite}{RGB}{225, 225, 245}


\pagecolor{darkgray}
\color{offwhite}

\newcommand{\Z}{\mathbf{Z}}
\newcommand{\N}{\mathbf{N}}
\newcommand{\R}{\mathbf{R}}
\newcommand{\Q}{\mathbf{Q}}
\newcommand{\C}{\mathbf{C}}

\newcommand{\id}{\mathrm{id}}
\newcommand{\op}{\mathrm{op}}
\newcommand{\diam}{\mathrm{diam}}
\newcommand{\GL}{\mathrm{GL}}
\newcommand{\Tr}{\mathrm{Tr}}
\newcommand{\im}{\mathrm{im}}
\newcommand{\rank}{\mathrm{rank}}

\newcommand{\cl}[1]{\overline{#1}}

\swapnumbers % places numbers before thm names

\theoremstyle{plain} % The "plain" style italicizes all body text.
	\newtheorem{thm}{Theorem}
		\numberwithin{thm}{section} % Theorem numbers are determined by section.
	\newtheorem{lemma}[thm]{Lemma}
	\newtheorem{prop}[thm]{Proposition}
	\newtheorem{cor}[thm]{Corollary}

\theoremstyle{definition}
    \newtheorem{defn}[thm]{Definition}
	\newtheorem{example}[thm]{Example}
	\newtheorem{exercise}[thm]{Exercise} %Exercise

\begin{document}
    \newtcolorbox{question}[2][]{fonttitle=\large, fontupper=\large, title=Question {#2}., oversize, arc=3mm, outer arc=2mm, opacityback=0.9, coltitle=offwhite, colframe=pastelblue, colback=darkgray, colupper=lightgray, collower=lightgray, leftrule=1mm, rightrule=1mm, toprule=1.5mm, titlerule=1mm, bottomrule=1mm, valign=center, add to natural height=5mm, lower separated=false, before lower=\begin{proof}, after lower= \\ \end{proof}, #1, breakable=true}
    \setcounter{section}{5}
    \section{HOMEWORK 6 HAAHHAHAHAHAHAA}
    \begin{question}{19}
        Let $X$ be a metric space and let $A\subseteq X$. A \textbf{compact exhaustion} for $A$ is a sequence of compact sets $K_1,K_2,K_3,\ldots$ such that $U=\bigcup_{i\geq 1}K_i$ and $K_i\subseteq K_{i+1}^\circ$.
        \begin{enumerate}[label=(\alph*)]
            \item Let $U\subseteq \R^n$ be a bounded open set. Show that $U$ has a compact exhaustion.
            \item Now show that every open set $U\subseteq \R^n$ has a compact exhaustion.
        \end{enumerate}
        \tcblower
        Suppose Question 19 is true. Then the result immediately follows.
    \end{question}
    \pagebreak
    \begin{question}{19}
        Let $x,y\in \ell^\infty$ be two sequences. Let us say that $y$ is \textbf{dominated} by $x$, denoted $x\geq y$, if $|x_n|\geq |y_n|$ for all $n\in \N$. Let $D_x$ denote the set of all sequences which are dominated by $x$:
        \[ D_x = \{y\in \ell^\infty: |y_n|\leq |x_n| \text{ for all $n\in \N$}\}. \]
        Prove that $D_x$ is compact if and only if $x_n\rightarrow 0$.

        \tcblower

        Suppose that \(D_x\) is compact. Suppose for contradiction that \(x_n \nrightarrow 0\). For some \(\varepsilon > 0\), \(|x_{N_k}| \geq \varepsilon\) for an infinite number of \(N_k\). Consider the open cover \(\{B(\vec{y}_i, \frac{\varepsilon}{2})\}_{i\in I}\), which is the collection of \(\frac{\varepsilon}{2}\)-balls centered around every \(\vec{y}_i \in D_x\). By compactness of \(D_x\), there is a finite subcover \(\{B(\vec{y}_i, \frac{\varepsilon}{2})\}_{i \leq m}\). Now, we construct a \(y \in D_x\) as follows:

        For every sequence \(\vec{y}_i\), let
        \[
            y_{N_i} = \begin{dcases}
                \varepsilon, &\text{ if } (\vec{y}_i)_{N_i} < \frac{\varepsilon}{2};\\
                0, &\text{ if } (\vec{y}_i)_{N_i} \geq \frac{\varepsilon}{2} ;\\
            \end{dcases}
        \]
        For all other terms in \(y\), make it 0. Notice that for all \(B(\vec{y}_i, \frac{\varepsilon}{2})\),
        \[
            \|y-\vec{y}_i\|_\infty \geq |y_{N_i} - (\vec{y}_i)_{N_i}| \geq \frac{\varepsilon}{2} \implies y \notin D_x
        \]
        which is a contradiction.

        Conversely, suppose that \(x_n \rightarrow 0\). Let \(\{U_i\}_{i \in I}\) be an open cover of \(D_x\).
    \end{question}
    % \question{20}
    % \end{proof}
\end{document}